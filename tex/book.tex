\documentclass{tufte-book}

\hypersetup{colorlinks}

\usepackage{xspace}
\newcommand{\PETSCVERSION}{3.6\xspace}
\newcommand{\PETSCMINORVERSION}{3.6.2\xspace}
\newcommand{\PETSCMONTH}{November 2015\xspace}

\newcommand{\CODELOC}{}  % each Chapter redefines this

\let\STUBSONLY\relax  %                                <--- UNCOMMENT TO GENERATE STUB CHAPTERS
\ifx\STUBSONLY\undefined
  \newcommand{\stubinput}[2]{\input{#1}}
\else
  \newcommand{\stubinput}[2]{\vspace{5cm} \centerline{\LARGE Percent completed:  \Huge #2\%.} \vfill}
\fi

\renewcommand{\maketitlepage}[0]{%
  \cleardoublepage%
  {%
  \sffamily%
  \begin{fullwidth}%
  \fontsize{18}{20}\selectfont\par\noindent\textcolor{darkgray}{\allcaps{\thanklessauthor}}%
  \vspace{9pc}%
  \fontsize{32}{40}\selectfont\par\noindent\textcolor{darkgray}{\allcaps{\thanklesstitle}}%
  \vfill%
    \fontsize{18}{40}\selectfont\par\noindent\textcolor{darkgray}{\allcaps{numerical solutions in C using}}%
    \fontsize{18}{40}\selectfont\par\noindent\textcolor{darkgray}{\allcaps{the Portable, Extensible Toolkit for Scientific computation version \PETSCVERSION}}%
  \vfill%
  \fontsize{14}{16}\selectfont\par\noindent\allcaps{\thanklesspublisher}%
  \end{fullwidth}%
  }
  \thispagestyle{empty}%
  \clearpage%
}

\title[PETSc for PDEs]{PETSc \\ for \\ Partial \\ Differential \\ Equations}

\author{Ed Bueler}
\publisher{Maybe Someday a Publisher of This Book}

\date{\today}

\usepackage{booktabs} % for nicely-typeset tabular material
\usepackage{verbatim} % for "comment" environment
%\usepackage{underscore} % causes "Missing \endcsname inserted" error?
\usepackage{graphicx}
\usepackage{amsmath,amssymb,amsthm,bm}
\usepackage{tikz}

\usetikzlibrary{arrows,decorations.markings,decorations.pathreplacing,fadings}

\usepackage{pgfplots}

\usepackage{newfloat,caption}

\usepackage{fancyvrb}
% an alternative which does not seem to have "start string" and
% "stop string" capabilities:
%\usepackage{listings}

\newtheorem*{theorem}{Theorem}
\theoremstyle{definition}
\newtheorem*{example}{Example}
\newtheorem*{examplecont}{Example, continued}

% Prints an epigraph and speaker in sans serif, all-caps type.
\newcommand{\openepigraph}[2]{%
  %\sffamily\fontsize{14}{16}\selectfont
  \begin{fullwidth}
  \sffamily\large
  \begin{doublespace}
  \noindent\allcaps{#1}\\% epigraph
  \noindent \Large \allcaps{#2}% author
  \end{doublespace}
  \end{fullwidth}
}

\newcommand{\monthyear}{%
  \ifcase\month\or January\or February\or March\or April\or May\or June\or
  July\or August\or September\or October\or November\or
  December\fi\space\number\year
}


% the goal here is to have a display of code with a separate "Code x.y" numbering
% and with the caption in the margin
\DeclareFloatingEnvironment[placement={!ht},name=Code]{mycodeenv}
\captionsetup[mycodeenv]{labelfont=bf}
% next line duplicates declaration of environment "marginfigure" in tufte-common.def
\newenvironment{margincode}[1][-1.2ex]%
  {\begin{@tufte@margin@float}[#1]{mycodeenv}}
  {\end{@tufte@margin@float}}

\newcommand{\trueinput}[2]{
\VerbatimInput[frame=single,framesep=3mm,label=\fbox{\normalsize \textsl{\,#1\,}},fontfamily=courier,fontsize=\footnotesize]{#2}
}

%\inputfromline{FULLPATH}{FILENAME}{CAPTION}{FIRSTLINE}{LABEL}
\newcommand{\inputfromline}[5]{
\vspace{0.8cm}
\let\FancyVerbStartString\relax
\let\FancyVerbStopString\relax
\begin{minipage}[l]{1.25\textwidth}
\VerbatimInput[frame=single,%
               framesep=3mm,%
               label=\fbox{\small \textsl{\,#2\,}},%
               fontfamily=courier,%
               fontsize=\footnotesize,%
               firstline=#4]{#1}
\end{minipage}
\vspace{0.5cm}
\begin{margincode}[1.0cm]
\caption{#3}
\label{#5}
\end{margincode}
\vspace{1.5cm}
}

\newcommand{\inputwhole}[4]{\inputfromline{#1}{#2}{#3}{1}{#4}}

%\cinputraw{FULLPATH}{FILENAMESHOWN}{CAPTION}{PARTSTRING}{STARTSTRING}{STOPSTRING}{LABEL}
\newcommand{\cinputraw}[7]{
\newcommand*\FancyVerbStartString{#5}
\newcommand*\FancyVerbStopString{#6}
\vspace{0.8cm}
\begin{minipage}[l]{1.25\textwidth}
\VerbatimInput[frame=single,%
               framesep=3mm,%
               label=\fbox{\small \textsl{\,#2\,}#4},%
               fontfamily=courier,%
               fontsize=\footnotesize]{#1}
\end{minipage}
\vspace{0.5cm}
\begin{margincode}[1.0cm]
\caption{#3}
\label{#7}
\end{margincode}
\vspace{1.5cm}
\let\FancyVerbStartString\relax
\let\FancyVerbStopString\relax
}

\newcommand{\cinput}[6]{%
    \cinputraw{cstrip/#1}{#2#1}{#3}{}{#4}{#5}{#6}}

\newcommand{\cinputpart}[7]{%
    \cinputraw{cstrip/#1}{#2#1}{#3}{\quad \textbf{part #4}}{#5}{#6}{#7}}

\newcommand{\cinputpartnostrip}[7]{%
    \cinputraw{#1}{#2}{#3}{\quad \textbf{part #4}}{#5}{#6}{#7}}

\DefineVerbatimEnvironment{code}{Verbatim}
{fontsize=\small,frame=lines,framerule=0.2mm,framesep=2.0mm,xleftmargin=5mm}

\DefineVerbatimEnvironment{codeplain}{Verbatim}
{fontsize=\small,xleftmargin=5mm}

\DefineVerbatimEnvironment{cline}{Verbatim}
{fontsize=\small,frame=leftline,framerule=0.3mm,framesep=2.0mm}


\newcommand{\complabel}[2]{#1 \\ \footnotesize #2}
\newcommand{\usedlabel}[2]{\complabel{#1}{\emph{#2}}}

% usage: \standardstack{scale}{objective}{Jacobian}{TS}{DMDA}{DMPlex}
%   where last 5 args are either "dashed" or empty
\def \standardstack#1#2#3#4#5#6{
\begin{tikzpicture}[scale=#1,
                    >={Latex[length=2mm]},
  component/.style={
     rectangle,draw,fill=white,align=center,line width=1pt},
  userfcn/.style={
     rounded corners,draw,fill=white,draw,align=center,line width=1pt,font={\itshape,\normalsize}}]

\draw[line width=1pt] (3,8) node[userfcn,minimum width=80mm] (usercode) {user code \\ \vspace{10mm}};
\draw[line width=1pt] (-0.5,8.2) node[userfcn,#2] (objcode) {objective};
\draw[line width=1pt] (3,7.5) node[userfcn] (rescode) {residual};
\draw[line width=1pt] (6.7,8.2) node[userfcn,#3] (jaccode) {Jacobian};

\draw[line width=1pt] (-4,4.7) node[component,#4] (ts) {\complabel{\pTS}{time-stepping}};

\draw[line width=1pt] (-0.5,4) node[component] (snes) {\complabel{\pSNES}{nonlinear solver}};
\draw[line width=1pt] (-0.5,2) node[component] (ksp)  {\complabel{\pKSP}{linear solver}};
\draw[line width=1pt] (-0.5,0) node[component] (pc)   {\complabel{\pPC}{preconditioner}};

\draw[line width=1pt] (4,4) node[component,#5] (dmda) {\complabel{\pDMDA}{structured grid}};
\draw[line width=1pt,dashed] (8.5,4.5) node[component,#6] (dmplex) {\complabel{\pDMPlex}{mesh}};

\draw[line width=1pt] (3,0) node[component] (matj)   {\usedlabel{\pMat}{Jacobian}};
\draw[line width=1pt] (7,0) node[component] (vecs)   {\pVecs \\ \footnotesize  \emph{solution, other fields}};

\path
   ([xshift=2em]objcode.south) edge[->,bend right,distance=6.5em,#2] node {} (vecs)
   ([xshift=2em]rescode.south) edge[->,bend left] node {} ([xshift=2.5em]vecs.north)
   ([xshift=1em]jaccode.south) edge[->,#3] node {} ([xshift=1em]vecs.north)
   (jaccode.south) edge[->,bend left,#3] node {} (matj)
   ([xshift=-13em]usercode.south) edge[->,#4] node {} (ts)
   ([xshift=-10em]usercode.south) edge[->] node {} (snes)
   ([xshift=2em]usercode.south) edge[->,#5] node {} (dmda)
   ([xshift=14em]usercode.south) edge[->,#6] node {} (dmplex)
   ([xshift=9em]usercode.south) edge[->] node {} (vecs)
   (ts) edge[#4] node {} (snes)
   (snes) edge node {} (ksp)
   (ksp) edge node {} (pc);
\end{tikzpicture}
}

\newcommand{\bA}{\mathbf{A}}
\newcommand{\bB}{\mathbf{B}}
\newcommand{\bE}{\mathbf{E}}
\newcommand{\bF}{\mathbf{F}}
\newcommand{\bG}{\mathbf{G}}
\newcommand{\bJ}{\mathbf{J}}
\newcommand{\bR}{\mathbf{R}}
\newcommand{\bU}{\mathbf{U}}

\newcommand{\bb}{\mathbf{b}}
\newcommand{\bc}{\mathbf{c}}
\newcommand{\be}{\mathbf{e}}
\newcommand{\bbf}{\mathbf{f}} % \bf already defined
\newcommand{\bg}{\mathbf{g}}
\newcommand{\bn}{\mathbf{n}}
\newcommand{\bp}{\mathbf{p}}
\newcommand{\bq}{\mathbf{q}}
\newcommand{\br}{\mathbf{r}}
\newcommand{\bs}{\mathbf{s}}
\newcommand{\bu}{\mathbf{u}}
\newcommand{\bv}{\mathbf{v}}
\newcommand{\bw}{\mathbf{w}}
\newcommand{\bx}{\mathbf{x}}
\newcommand{\by}{\mathbf{y}}
\newcommand{\bz}{\mathbf{z}}

\newcommand{\CC}{\mathbb{C}}
\newcommand{\RR}{\mathbb{R}}
\newcommand{\ZZ}{\mathbb{Z}}

\newcommand{\X}{\times}  % for nonzero entries in matrices

\newcommand{\XX}{$\bm{\times}$}  % for ticks in tables
\newcommand{\gX}{{\color{Gray} $\times$}}

\newcommand{\eps}{\epsilon}
\newcommand{\lam}{\lambda}
\newcommand{\lap}{\triangle}

\newcommand{\Div}{\ensuremath{\nabla\cdot}}
\newcommand{\Curl}{\ensuremath{\nabla\times}}
\newcommand{\grad}{\nabla}

\newcommand{\ip}[2]{\ensuremath{\left<#1,#2\right>}}

% iteration vec display in R^2
\newcommand{\twovect}[4]{\ensuremath{{#1}_{#2} =
                            \begin{bmatrix} #3 \\ #4 \end{bmatrix}}}
\newcommand{\rvect}[3]{\twovect{\bu}{#1}{#2}{#3}}

\newcommand{\cond}{\operatorname{cond}}
\newcommand{\onull}{\operatorname{null}}
\newcommand{\rank}{\operatorname{rank}}
\newcommand{\range}{\operatorname{range}}
\newcommand{\Span}{\operatorname{span}}

\renewcommand{\Re}{\operatorname{Re}}
\renewcommand{\Im}{\operatorname{Im}}

\newcommand{\Th}{\mathcal{T}_h}
\newcommand{\Pone}{\mathbf{P}_1}

\newcommand{\Matlab}{\textsc{Matlab}\xspace}
\newcommand{\Triangle}{\textsc{triangle}\xspace}
\newcommand{\MPI}{\textsc{MPI}\xspace}

\newcommand{\PETSc}{\textsc{PETSc}\xspace}

\newcommand{\pDM}{\texttt{DM}\xspace}
\newcommand{\pDMs}{\texttt{DM}s\xspace}

\newcommand{\pDMDA}{\texttt{DMDA}\xspace}
\newcommand{\pDMDAs}{\texttt{DMDA}s\xspace}

\newcommand{\pDMPlex}{\texttt{DMPlex}\xspace}
\newcommand{\pDMPlexs}{\texttt{DMPlex}s\xspace}

\newcommand{\pIS}{\texttt{IS}\xspace}
\newcommand{\pISs}{\texttt{IS}s\xspace}

\newcommand{\pKSP}{\texttt{KSP}\xspace}
\newcommand{\pKSPs}{\texttt{KSP}s\xspace}

\newcommand{\pPC}{\texttt{PC}\xspace}
\newcommand{\pPCs}{\texttt{PC}s\xspace}

\newcommand{\pSNES}{\texttt{SNES}\xspace}
\newcommand{\pSNESs}{\texttt{SNES}s\xspace}

\newcommand{\pTS}{\texttt{TS}\xspace}
\newcommand{\pTSs}{\texttt{TS}s\xspace}

\newcommand{\pMat}{\texttt{Mat}\xspace}
\newcommand{\pMats}{\texttt{Mat}s\xspace}

\newcommand{\pVec}{\texttt{Vec}\xspace}
\newcommand{\pVecs}{\texttt{Vec}s\xspace}


\setcounter{secnumdepth}{0}

% eventually this is a good idea:
%\usepackage{makeidx}
%\makeindex

\begin{document}

\begin{comment}
% epigraph page first
\newpage\thispagestyle{empty}
\openepigraph{%
\dots when there are disputes among persons, we can simply say: Let us calculate, without further ado, to see who is right.
}{Gottfried Wilhelm Leibniz}
\vfill
\openepigraph{%
Developing parallel, nontrivial PDE solvers that deliver high performance is still difficult and requires months (or even years) of concentrated effort.  PETSc is a toolkit that can ease these difficulties and reduce the development time, but it is not a black-box PDE solver, nor a silver bullet
}{Barry Smith}
\vfill
\openepigraph{%
Tufte's style is known for its extensive use of sidenotes, tight integration of graphics with text, and well-set typography.
}{The Tufte-LaTeX\ Developers}
\vfill

\frontmatter
\end{comment}

\maketitle


\newpage
\begin{fullwidth}
~\vfill
\thispagestyle{empty}
\setlength{\parindent}{0pt}
\setlength{\parskip}{\baselineskip}
Copyright \copyright\ \the\year\ \thanklessauthor

\par\smallcaps{Published by \thanklesspublisher}

%\par  [license here] I have no idea if I should license this at this stage

\par\textit{First printing, \monthyear}
\end{fullwidth}

\tableofcontents


% Thanks to \mbox{Jed Brown} and \mbox{Constantine Khroulev}, fellow students, and gurus.


\chapter*{Preface}

\section{Why read this book?}

This book is about numerically solving linear and nonlinear partial differential equations (PDEs) by writing C\cite{KernighanRitchie1988} code that directly calls \PETSc.  It tries to both explain the ideas and illustrate them through example codes.  The example codes come with enough background information and context so that readers can easily use them as a basis for further developments.  Demonstrated scalability is a goal, so runtime options are explained and compared, and explored in the exercises.

This book is written from the conviction that \emph{better access to common knowledge among experts} advances scientific computing as a discipline.  An expert in \PETSc may say about this book that ``I knew all that'' \emph{and} that ``this book is a fast on-ramp to what I already know.''  That is precisely my hope.

So, let's suppose you have taken a mathematics course or two in partial differential equations (PDEs).  You have written a few codes in C, and probably some in \Matlab or python or similar prototyping languages.  You are interested in solving PDE models numerically in parallel on big problems.  Then this book is for you.

\section{Get the example codes}

Use \texttt{git} to get the example C codes, and \LaTeX\xspace sources for the book too:
\begin{cline}
$ git clone https://github.com/bueler/p4pdes.git
\end{cline}
%$
In the text we use the C codes in subdirectories \texttt{p4pdes/c/ch}$N$\texttt{/} for $N$ equal to the chapter number.

\section{What is \PETSc?}

The Portable, Extensible Toolkit for Scientific computing (\PETSc)\sidenote{Say it ``pets sea.''  The homepage for PETSc, including download and installation instructions, is \href{http://www.mcs.anl.gov/petsc/index.html}{www.mcs.anl.gov/petsc}.} is an open-source, mathematical software library built on top of the standard software layer for large-scale parallel computation, namely the Message Passing Interface (MPI) \citep{Groppetal1999}.  Thus \PETSc is a framework capable of solving problems like PDEs at ``large scale,'' that is, at high resolution and on supercomputers with hundreds to millions of cores.

\PETSc is not particularly new.  Version 2.0, the first version to make an impact in the scientific computing world, was built in 1994.  A well-known monograph \citet{Smithetal1996}\sidenote{B.~Smith, P.~Bjorstad, and W.~Gropp. \emph{Domain decomposition: parallel multilevel methods for elliptic partial differential equations}. Cambridge University Press, 1996} uses \PETSc 2.0 for scalable solutions of linear PDEs.  That book focusses on pre-conditioned iterative linear solvers and domain decomposition.  For example, methods like additive Schwarz are shown to scalably-solve the Poisson equation on irregular domains.

But \PETSc is now at version \PETSCVERSION.\sidenote{Version \PETSCMINORVERSION is current in \PETSCMONTH.}  It has evolved into a more powerful toolbox with a much richer API (application program interface).  Typical examples and applications are for nonlinear PDEs.  Nonlinear, multigrid, and multiphysics\sidenote{This buzzword refers to a diverse system of coupled PDEs with nontrivial scalings among the variables.} parts of the API are now highly-visible to users.  The \PETSc strategy is to compose Newton's method and mesh topology tools with a run-time choice of preconditioners and iterative linear solvers, so navigating this ``stack'' requires more user knowledge than a generation ago.

In summary, \PETSc may not be a silver bullet, but it presents users with many powerful tools for solving hard problems, well beyond iterative linear algebra.  As twenty years have passed since version 2.0 and \citet{Smithetal1996}, a new book about \PETSc is appropriate.


\section{What I need from you, the reader}

To make sense of this book, some of the mathematical theory\sidenote{\citep{Evans1998} is recommended, but not really a prerequisite.} of PDEs must be familiar.  I will also assume that the reader has a bit of \emph{practical intuition} about such problems---perhaps the better term in ``maturity'' with respect to PDEs---including exposure to nonlinear problems.\sidenote{\citep{Ockendonetal2003} is recommended.}  Of course, all applied mathematicians, distinctly including this author, are wanting when it comes to injesting the mathematical theory of, and building intuition for, nonlinear PDEs.

Multiple numerical discretization paradigms will arise here, but at least one numerical approach to PDEs should probably already be in the reader's toolbox.  That might be the finite element method (FEM) \citep{Braess2007,Elmanetal2005}, finite differences \citep{MortonMayers2005}, finite volumes \citep{LeVeque2002}, or spectral methods \citep{KarniadakisSherwin2013,Trefethen2000}.  Exposure to multigrid ideas \citep{Briggsetal2000} would be helpful, but the concepts will be reviewed.

Certainly ideas from numerical linear algebra \citep{Greenbaum1997,TrefethenBau1997} will appear, often with only a brief introduction.  The definitions of vector norms and (induced) matrix norms, along with the LU and Cholesky decompositions, will be assumed.  The textbook by \citet{TrefethenBau1997} is the closest of the above-mentioned texts to a strict prerequisite for the material in this book.

Starting in Chapter \ref{chap:un} I will assume that you are interested in unstructured grids, though not at the exclusion of structured approaches.  Some basics of the FEM method will be reviewed in Chapters \ref{chap:of}, \ref{chap:un}, and \ref{chap:dp}, but the reader with some background understanding will benefit most.\sidenote{Priority topics for review include the weak form of a PDE and the idea of assembling the equations element-by-element.}


\section{There is much that this book does NOT do}

I'll assume you want to solve PDEs, but there are many other uses of \PETSc.  Furthermore, this book\begin{itemize}
\item  does not replace either the \PETSc \emph{User's Manual} or online searches of the \PETSc manual pages,
\item  does not really help you install \PETSc,
\item  does not use Fortran or C++; all examples are in C though we use ANSI C99 features,
\item  does not help with most of the many packages \PETSc links to,
\item  does not do a complete job of teaching the FEM or any other discretization paradigm for PDEs,
\item  does not particularly care whether its numerical solutions are good models of physical problems---that is your job, not mine,
\item  does not consider spatial dimensions except 1, 2, and 3,
\item  does not prove anything,\sidenote{We give evidence for convergence and scalability when possible.} and
\item  does not adequately cover what is known about nonlinear PDEs, much less what is not known.
\end{itemize}


\section{At the command line}

Before we really get started, what ``computer skills'' do I assume?  In summary, something more than what you need to get started with \Matlab, but certainly less than professional programmer abilities.  The numerical programming here is, fortunately, stereotyped coding using a modest language subset.

You need to have written and compiled C programs before.   Running and modifying the examples will inevitably expose some subtleties of the C language, but no more than would appear in a first college course in computer programming using C or a similar language.  Thus serious experience and/or capability as a C programmer is not needed.\sidenote{The author does not qualify as such.}  However, the concepts of compiling and linking, of including header files, of passing arguments by value and reference, and (most importantly) of pointer variables and arrays-as-pointers, should all be familiar.

\PETSc's configuration finds the compiler, and enables ``\texttt{make}'' as a build command, or it fails.  Therefore the examples included with this book can be run with no more work than to type ``\texttt{make}'' at the command line; with luck this does not lead to compiler errors.\sidenote{All examples in this book were built and run with the GNU C compiler ``\texttt{gcc}'' (\href{https://gcc.gnu.org/}{\texttt{gcc.gnu.org}}), but this fact would not be visible without mentioning it here.}  Of course, doing the exercises, or modifying the examples, will always require attention to compile-time errors.  Searching in the \PETSc HTML manual pages for the various commands and data types, online at
\begin{quote}
\href{http://www.mcs.anl.gov/petsc/documentation/index.html}{\texttt{www.mcs.anl.gov/petsc/documentation/index.html}}
\end{quote}
or downloaded with the \PETSc source, should be the reader's first action for resolving compile-time errors.  The reader should review the \emph{\PETSc User's Manual} \citep{petsc-user-ref}; it explains best why the API is designed in the way it is.

Finally, this book assumes a \emph{bash} shell (\href{https://www.gnu.org/software/bash/bash.html}{\texttt{www.gnu.org/software/bash/bash.html}}) or a shell that interprets bash syntax.  Uses of bash syntax are mostly trivial, but examples of ``for loop'' syntax appear every once in a while:
\begin{cline}
$ for N in 1 2 3; do echo "count is $N"; done
count is 1
count is 2
count is 3
\end{cline}


\mainmatter

%\setcounter{chapter}{0}

\chapter{Getting started with PETSc}
\label{chap:gs}
\renewcommand{\CODELOC}{ch1/}

\chapter{Getting started with PETSc}
\label{chap:getstarted}

\section{A code that does almost nothing, but in parallel}

The purpose of the \PETSc library is to help you solve scientific and engineering problems, such as PDEs of course, on distributed computers.  But \PETSc is built ``on top of'' the Message Passing Interface (MPI; \citep{Groppetal1999}) library, and some of the flavor of that library comes through.  We start with an example C code that is essentially an introductory MPI example, though it calls \PETSc for some basic tasks.

This code \texttt{c1e.c}, shown in its entirety in Figure \ref{code:e}, computes Euler's constant
\begin{equation}
e = \sum_{n = 0}^\infty \frac{1}{n!} \approx 2.718281828 \label{introeseries}
\end{equation}
It does the computation in a distributed manner by computing one term of the infinite series on each process.  Thus it computes a better estimate of $e$ when run on more MPI processes. While this is a silly use of \PETSc, it is an easy to understand parallel computation.

As with any C source for an executable, \texttt{c1e.c} has a function called \texttt{main()} which takes inputs from the command line, namely \texttt{argc} and \texttt{argv},\sidenote{Here \texttt{argc} is an \texttt{int} holding the argument count.  The second argument \texttt{argv} is an array of strings (i.e.~C type \texttt{char**}) holding the command line (i.e.~command and arguments).  However, in all codes in this book we simply pass these arguments on to \PETSc through the \texttt{PetscInitialize()} method.} and which outputs an \texttt{int} which is $0$ if the program succeeds.  Also, like all C codes, we include the needed headers; only \texttt{petscsys.h} is needed here, but later codes will include other \PETSc headers.

The substance of \texttt{main()} is to declare some variables, do a computation on each process, and communicate the results between processes to get an estimate of $e$.  Finally we report success (i.e.~return $0$) at the end.

More details of \texttt{c1e.c} can be described once we compile and run the code.  Please do the following to start:
\begin{cline}
$ cd p4pdes/c/  # download this book, and its codes in p4pdes/c/, by
                #     "git clone https://github.com/bueler/p4pdes.git"
$ make c1e
\end{cline}
For the command ``\texttt{make}'' to work there must be a makefile, of course, and it is \texttt{/p4pdes/c/makefile}, shown in Figure \ref{code:c1emakefile}.  For all the codes in this book, the makefile will have this form, exactly as recommended in the \PETSc User's Manual \citep{petsc-user-ref}.

\inputwhole{../c/c1e.c}{\texttt{p4pdes/c/c1e.c}}{Compute $e$ in parallel.}{code:e}

Run the code on one MPI process like this:
\begin{cline}
$ ./c1e
e is about 1.000000000000000
rank 0 did 1 flops
\end{cline}
%$
The value $1.0$ is a very poor estimate of $e$, but this code does better with more processes:
\begin{cline}
$ mpiexec -n 5 ./c1e
rank 3 did 7 flops
rank 4 did 9 flops
e is about 2.708333333333333
rank 0 did 1 flops
rank 1 did 3 flops
rank 2 did 5 flops
\end{cline}
%$
That's a better estimate of $e$, but hardly impressive.  On the other hand, with $N=20$ processes, and thus $N=20$ terms in series \eqref{introeseries}, we get a good estimate:
\begin{cline}
$ mpiexec -n 20 ./c1e
rank 9 did 19 flops
...
e is about 2.718281828459045
rank 0 did 1 flops
...
rank 18 did 37 flops
\end{cline}
%$

\cinputraw{c1emakefile.frag}{extract from \texttt{p4pdes/c/makefile}}{All \texttt{makefile}s for the \PETSc codes in this book look like this.}{}{//START}{//STOP}{code:c1emakefile}

Now, perhaps the reader is worried that this book was written using a large supercomputer whereas the reader has a little laptop with only a couple of cores.  Not so; these $N=5$ and $N=20$ process MPI calls work just fine on the author's four-core laptop because MPI processes are created as needed.\sidenote{Multitasking operating systems have been around for quite some time!}

The main job in \texttt{c1e.c} is to collect the sum of the terms of (truncated) infinite series \eqref{introeseries} onto each process.  After each process computes term $1/n!$, where $n$ is its rank in the MPI communicator, a call to \texttt{MPI\_Allreduce()} does the sum and then sends the sum back to each process.  We also get the rank of the current process with a call to \texttt{MPI\_Comm\_rank()}.  These direct uses of the MPI library illustrate that \PETSc, which tries to avoid duplication of MPI functionality, need not be called for some low-level parallel tasks.

In \texttt{c1e.c} we print the computed estimate of $e$, and we also have each process print its rank and the work it did.  Observe that \texttt{PetscPrintf()}, a formatted print command like \texttt{fprintf()} from the C standard library, is called twice, once with MPI communicator \texttt{PETSC\_COMM\_WORLD} and once with \texttt{PETSC\_COMM\_SELF}.\sidenote{Thereby we illustrate collective and non-collective operations in the same program.}  The first of these \texttt{STDOUT} printing jobs is therefore \emph{collective} over all processes, and thus only done once, and other printing job is individual to each rank.\sidenote{A process is often just called a \emph{rank} in MPI language.}  In the output the \texttt{PETSC\_COMM\_SELF} printed lines appear in almost random order because the print occurs as soon as that process reaches that line.

Every \PETSc program should start and end with the commands \texttt{PetscInitialize()} and \texttt{PetscFinalize()}:
\begin{code}
PetscInitialize(&argc,&args,(char*)0,help);
... everything else goes here ...
PetscFinalize();
\end{code}
Also, so that this \PETSc code can provide useful usage help, we add a \texttt{help} string at the start; it is a good place to say what the purpose of the code is.

There is one more observation about \texttt{c1e.c} and all \PETSc programs: there is error-checking clutter from capturing and checking the return code of each method called.  While languages other than C could help with decluttering this stuff, we are stuck with lines that look like
\begin{code}
ierr = PetscCommand(...); CHKERRQ(ierr);
\end{code}
The explanation of this clutter is that almost all \PETSc methods, and most user-written methods in \PETSc programs, return an \texttt{int} for error checking, with value $0$ if successful.  In the line above, \texttt{ierr} is a \texttt{PetscInt} type and \texttt{CHKERRQ()} is a macro which does nothing if \texttt{ierr == 0} but which stops the program otherwise.  In the nonzero case the program stops with a traceback, namely a list of the nested methods, in reverse order, showing the line numbers and method names of the location where the error occurred.  This traceback tends to be the first line of defense when debugging run-time errors.  Examples in this book always capture-and-check the returned error code in this way, despite the clutter.  On the other hand, after this chapter we will strip the ``\texttt{ierr =}'' and ``\texttt{CHKERRQ(ierr);}'' clutter from the code displayed in the text, even though it is still present in the source file itself.


\section{Linear systems}

Of course, our goal in the remainder of the book is to compute more interesting quantities than Euler's constant $e$.  At the core of most \PETSc computations is a finite-dimensional linear system.  Before solving such systems in \PETSc, it is useful to recall the most basic ideas of numerical linear algebra.

Suppose $\bb\in \RR^{N}$ is a column vector and $A\in\RR^{N\times N}$ is a square matrix.  The linear system
\begin{equation}
A \bu = \bb \label{introsystem}
\end{equation}
has a unique solution if $A$ is invertible, namely
\begin{equation}
\bu = A^{-1} \bb. \label{introsolution}
\end{equation}
This is simple in theory.

It is not so simple in practice, however, when solving large systems on a computer.  There are two key facts to keep in mind while working numerically  \citep{TrefethenBau}.
\renewcommand{\labelenumi}{\roman{enumi})}
\begin{enumerate}
\item \emph{limit to accuracy}:  If real numbers are represented on the computer with machine precision $\eps$ then the solution of \eqref{introsystem} can only be computed within an error $\kappa(A) \eps$ where $\kappa(A) = \|A\|_2 \|A^{-1}\|_2$ is the (2-norm) \emph{condition number} of $A$.\sidenote{Fact i) is about \emph{conditioning} not \emph{methods}.  Informally speaking, there are linear systems $A,\bb$ that are the same to within $\eps$ but for which the infinite-precision solutions $\bu$ are different by the amount $\kappa(A) \eps$.}
\item \emph{cost of direct solutions}:  Computation of solution \eqref{introsolution} by a direct method like Gauss elimination, whether actually forming $A^{-1}$ or not, is an $O(N^3)$ operation.
\end{enumerate}

For a sense of the consequences of these facts, let's put in some numbers for $\eps$, $\kappa(A)$, and $N$.  On most computers the precision for the C \texttt{double} type, the modern default 64-bit representation of real numbers, is $\eps = 2.2 \times 10^{-16}$.  By i), a linear system having $\kappa(A) \approx 10^{10}$, for example, can only be solved to about six digits of precision.

Regarding computational cost, because of ii) a linear system with $N=10^6$ equations requires $\sim 10^{18}$ operations to solve by Gauss elimination.  While even modern supercomputers take a while to do a quintillion operations, so that Gauss elimination is impractical for systems of $N=10^6$ equations, we will successfully solve problems of this size on a single processor in a few seconds, and in $O(N)$ operations, in Chapter \ref{chap:multigrid}.
%time ./c4poisson -da_refine 7 -ksp_type cg -pc_type mg
%on 1153 x 1153 grid:  iterations 2, residual norm = 1.88082e-05
%real 11.18

To build our first \PETSc code for a linear system, we describe the \PETSc objects which store vectors and matrices.


\section{\PETSc \pVec and \pMat objects}

The \pVec object is a container and interface for a distributed vector, and a \pMat object holds a distributed matrix.  Although \PETSc is written in C, not C++, it is a relentlessly object-oriented software library.  Consider the operations which might touch a matrix object \texttt{A} in a linear system like \eqref{introsystem}:
\begin{code}
Mat A;
MatCreate(COMM,&A);
MatSetSizes(A,PETSC_DECIDE,PETSC_DECIDE,N,N);
PetscObjectSetName((PetscObject)A,"A");
MatSetOptionsPrefix(A,"a_");
MatSetFromOptions(A);
... fill entries of (i.e. assemble) A ...
... solve system with A ...
MatDestroy(&A);
\end{code}
In fact, for ``\pVec'' objects storing vectors, ``\pMat'' objects storing matrices, and indeed for all \PETSc object types, this basic sequence of operations applies:
\begin{code}
Object X;
ObjectCreate(COMM,&X);
... set properties of X from code ...
ObjectSetFromOptions(X);  // allows run-time setting of properties
... use X ...
ObjectDestroy(&X);
\end{code}
Of course ``\texttt{Object}'' here is merely a meta-name for a \PETSc type like \pVec or \pMat.

\PETSc objects are generically distributed across, and accessible from, multiple MPI processes.  Therefore the first argument of an \texttt{ObjectCreate()} method is an MPI communicator (``\texttt{COMM}'').  All processes in \texttt{COMM} must call the \texttt{ObjectCreate()} method.

Evidently \texttt{Mat A} above has an internal representation with nontrivial structure, but that is hidden.  Indeed the data structure inside \texttt{A} depends on runtime choices, the most basic being that the number of bytes used to store \texttt{A} on a given MPI process will depend on the number of processes.  At a deeper level, a \PETSc \pMat object need not even \emph{have} entries, but it may instead represent code that applies a linear operator to vectors.

Because of the call to \texttt{MatSetOptionsPrefix()}, run-time options can specifically address the particular \pMat object.  For example, the run-time option \texttt{-a\_mat\_view} will print out the entries of \texttt{A}.  An option prefix like ``\texttt{a\_}'' is especially helpful in distinguishing \pMat objects at the command line in a context with multiple \pMats.

Once \texttt{A} is created and set up by the first five commands \texttt{MatCreate()}--\texttt{MatSetFromOptions()}, then various methods become valid for \texttt{A}, for example including the \texttt{MatSetValues()} method to set entries in \texttt{A}.


\section{Assembly and parallel layout of \pVecs and \pMats}

Fundamentally, a \pVec or \pMat can store its entries in parallel across all the processes in the MPI communicator used when creating it.  For the \pVec type, the create-assemble sequence of a vector with four entries might look like
\begin{code}
Vec x;
PetscInt   i[4] = {0, 1, 2, 3};
PetscReal  vals[4] = {11.0, 7.0, 5.0, 3.0};

VecCreate(COMM,&x);
VecSetSizes(x,PETSC_DECIDE,4);
VecSetFromOptions(x);
VecSetValues(x,4,i,vals,INSERT_VALUES);
VecAssemblyBegin(x);
VecAssemblyEnd(x);
\end{code}
The four entries of \texttt{Vec x} are (obviously) set by \texttt{VecSetValues()}, putting values from array \texttt{vals[]} at the indices given by \texttt{i[]}.  Potentially such operations require communication between processes, because entries of \texttt{x} which are stored on process $m$ could be set by processor $n$.  Such communication is started and ended by the \texttt{VecAssemblyBegin(), VecAssemblyEnd()} pair of commands.

\begin{marginfigure}
\begin{tikzpicture}[scale=3.5]
  \pgfmathsetmacro\fourth{1.0/4.0}
  \pgfmathsetmacro\xoff{0.12}
  \pgfmathsetmacro\yoff{0.12}
  \draw[xstep=\fourth,ystep=\fourth,black,thick] (0.0,0.0) grid (\fourth,1.0);
  \node at (-0.2,1.0-\yoff) {$i=0$};
  \node at (\xoff,1.0-\yoff) {$11.0$};
  \node at (-0.2,0.75-\yoff) {$i=1$};
  \node at (\xoff,0.75-\yoff) {$7.0$};
  \node at (-0.2,0.5-\yoff) {$i=2$};
  \node at (\xoff,0.5-\yoff) {$5.0$};
  \node at (-0.2,0.25-\yoff) {$i=3$};
  \node at (\xoff,0.25-\yoff) {$3.0$};
  \draw[decoration={brace,mirror,raise=5pt},decorate,line width=1pt] (0.3,0.0) -- (0.3,1.0);
  \node at (0.65,0.51) {rank $=0$};
\end{tikzpicture}
\bigskip
\caption{A sequential \pVec layout, all on rank $=0$ process.}
\label{fig:seqveclayout}
\end{marginfigure}

Suppose this sequence appears in a program called \texttt{myprogram.c}.  If the program is run sequentially on one process, i.e.~as
\begin{cline}
$ ./myprogram.c
\end{cline}
%$
then, at the end of the above create-assemble sequence, the storage of \texttt{x} looks like Figure \ref{fig:seqveclayout}.

However, if run as
\begin{cline}
$ mpiexec -n 2 ./myprogram.c
\end{cline}
%$
then the layout looks like Figure \ref{fig:mpitwoveclayout}.  In this case the argument \texttt{PETSC\_DECIDE} in \texttt{VecSetSizes()} is active because \PETSc \emph{decides} to put the first two entries of \texttt{x} on the rank $0$ process and the other two on the rank $1$ process. 

\begin{marginfigure}
\begin{tikzpicture}[scale=3.5]
  \pgfmathsetmacro\fourth{1.0/4.0}
  \pgfmathsetmacro\xoff{0.12}
  \pgfmathsetmacro\yoff{0.12}
  \draw[xstep=\fourth,ystep=\fourth,black,thick] (0.0,0.0) grid (\fourth,1.0);
  \node at (-0.2,1.0-\yoff) {$i=0$};
  \node at (\xoff,1.0-\yoff) {$11.0$};
  \node at (-0.2,0.75-\yoff) {$i=1$};
  \node at (\xoff,0.75-\yoff) {$7.0$};
  \node at (-0.2,0.5-\yoff) {$i=2$};
  \node at (\xoff,0.5-\yoff) {$5.0$};
  \node at (-0.2,0.25-\yoff) {$i=3$};
  \node at (\xoff,0.25-\yoff) {$3.0$};
  \draw[decoration={brace,mirror,raise=5pt},decorate,line width=1pt] (0.3,0.52) -- (0.3,1.0);
  \node at (0.65,0.752) {rank $=0$};
  \draw[decoration={brace,mirror,raise=5pt},decorate,line width=1pt] (0.3,0.0) -- (0.3,0.48);
  \node at (0.65,0.251) {rank $=1$};
\end{tikzpicture}
\bigskip
\caption{A parallel \pVec layout on two processes.  Because we call ``\texttt{VecSetSizes(x,PETSC\_DECIDE,4)}'', \PETSc decides to split the storage in the middle.}
\label{fig:mpitwoveclayout}
\end{marginfigure}

Though in the current context it is completely artificial, one could override the ``\texttt{PETSC\_DECIDE}'' parallel layout by replacing the \texttt{VecSetSizes()} line with this block of code:
\begin{code}
PetscMPIInt rank;
MPI_Comm_rank(COMM,&rank);
if (rank == 0) {
  VecSetSizes(x,3,4);
} else if (rank == 1) {
  VecSetSizes(x,1,4);
} else {
  SETERRQ(COMM,1,"this code only works with size==2 communicators");
}
\end{code}
That is, we could specify the ``local size'' argument to \texttt{VecSetSizes()}, instead of using \texttt{PETSC\_DECIDE}.\sidenote{In fact such inflexible coding is rarely necessary.}  The resulting layout is shown in Figure \ref{fig:artificialmpitwoveclayout}.

In summary, the reader is allowed to think of a \PETSc \pVec as a one-dimensional C array with its range of indices and contents ``broken-up'' across the processes in the MPI communicator used in the \texttt{VecCreate()} command.

Parallel matrix objects \pMat, which are at some level conceived as 2D arrays, obviously require an additional choice regarding distribution.  But \PETSc makes this choice inside the implementation of \pMat.  Namely, \PETSc always stores ranges of complete rows on each process.  Said another way, conceptually at least, the column vectors which form the \pMat are stored like \pVecs.

While other storage formats are possible, the one usually used in this book is \emph{parallel compressed sparse row storage}, what \PETSc calls the \texttt{MATMPIAIJ} type.  That is, a range of rows is owned by each process (parallel row storage), and within each owned range of rows only the nonzero entries are stored (sparse), and furthermore the data structures store these nonzero entries contiguously in an array, with an additional contiguous index array (compressed).

\begin{marginfigure}
\begin{tikzpicture}[scale=3.5]
  \pgfmathsetmacro\fourth{1.0/4.0}
  \pgfmathsetmacro\xoff{0.12}
  \pgfmathsetmacro\yoff{0.12}
  \draw[xstep=\fourth,ystep=\fourth,black,thick] (0.0,0.0) grid (\fourth,1.0);
  \node at (-0.2,1.0-\yoff) {$i=0$};
  \node at (\xoff,1.0-\yoff) {$11.0$};
  \node at (-0.2,0.75-\yoff) {$i=1$};
  \node at (\xoff,0.75-\yoff) {$7.0$};
  \node at (-0.2,0.5-\yoff) {$i=2$};
  \node at (\xoff,0.5-\yoff) {$5.0$};
  \node at (-0.2,0.25-\yoff) {$i=3$};
  \node at (\xoff,0.25-\yoff) {$3.0$};
  \draw[decoration={brace,mirror,raise=5pt},decorate,line width=1pt] (0.3,0.27) -- (0.3,1.0);
  \node at (0.65,0.65) {rank $=0$};
  \draw[decoration={brace,mirror,raise=5pt},decorate,line width=1pt] (0.3,0.0) -- (0.3,0.23);
  \node at (0.65,0.125) {rank $=1$};
\end{tikzpicture}
\bigskip
\caption{A non-default parallel \pVec layout on two processes, by forcing local sizes to be not equal.}
\label{fig:artificialmpitwoveclayout}
\end{marginfigure}

For example, the following code creates and assembles a \pMat \texttt{A} with four rows and four columns:
% see testmatcreate.c
\begin{code}
Mat A;
PetscInt  i, j[3];
PetscReal v[3];

MatCreate(PETSC_COMM_WORLD,&A);
MatSetSizes(A,PETSC_DECIDE,PETSC_DECIDE,4,4);
MatSetFromOptions(A);
MatSetUp(A);

i = 0;
j[0] = 0;    j[1] = 1;
v[0] = 4.0;  v[1] = -1.0;
MatSetValues(A,1,&i,2,j,v,INSERT_VALUES);
i = 1;
j[0] = 0;    j[1] = 1;    j[2] = 2;
v[0] = -1.0; v[1] = 4.0;  v[2] = -1.0;
MatSetValues(A,1,&i,3,j,v,INSERT_VALUES);
i = 2;
j[0] = 1;    j[1] = 2;    j[2] = 3;
MatSetValues(A,1,&i,3,j,v,INSERT_VALUES);
i = 3;
j[0] = 2;    j[1] = 3;
MatSetValues(A,1,&i,2,j,v,INSERT_VALUES);

MatAssemblyBegin(A,MAT_FINAL_ASSEMBLY);
MatAssemblyEnd(A,MAT_FINAL_ASSEMBLY);
\end{code}
The method \texttt{MatSetValues()} sets a \emph{block} of values, and in this case we use it to set each row of the matrix.  Thus ``\texttt{1,\&i}'' arguments to \texttt{MatSetValues()} say ``we are setting one row, and look at the location of integer \texttt{i} for the (global) index.''  The ``\texttt{3,\&j}'' arguments, used in the \texttt{i} $=1,2$ rows, say ``we are setting three values in the row, and look at integer array \texttt{j} for the (global) indices.''

If the above lines appeared in \texttt{myprogram.c}, and if it was run
\begin{cline}
$ mpiexec -n 2 ./myprogram
\end{cline}
%$
then this \pMat would have the entries and layout shown in Figure \ref{fig:mpitwomatlayout}.

\begin{marginfigure}
\begin{tikzpicture}[scale=3.5]
  \pgfmathsetmacro\fourth{1.0/4.0}
  \pgfmathsetmacro\xoff{0.12}
  \pgfmathsetmacro\yoff{0.12}
  \draw[xstep=\fourth,ystep=\fourth,black,thick] (0.0,0.0) grid (1.0,1.0);

  \node at (-0.2, 1.0-\yoff) {$i=0$};
  \node at (-0.2,0.75-\yoff) {$i=1$};
  \node at (-0.2, 0.5-\yoff) {$i=2$};
  \node at (-0.2,0.25-\yoff) {$i=3$};

  \node at ( 1.0-\xoff, 1.1) {$j=3$};
  \node at (0.75-\xoff, 1.1) {$j=2$};
  \node at ( 0.5-\xoff, 1.1) {$j=1$};
  \node at (0.25-\xoff, 1.1) {$j=0$};

  \node at (\xoff,1.0-\yoff) {$4.0$};
  \node at (0.25+\xoff,1.0-\yoff) {$-1.0$};

  \node at (\xoff,0.75-\yoff) {$-1.0$};
  \node at (0.25+\xoff,0.75-\yoff) {$4.0$};
  \node at (0.5+\xoff,0.75-\yoff) {$-1.0$};

  \node at (0.25+\xoff,0.5-\yoff) {$-1.0$};
  \node at (0.5+\xoff,0.5-\yoff) {$4.0$};
  \node at (0.75+\xoff,0.5-\yoff) {$-1.0$};

  \node at (0.5+\xoff,0.25-\yoff) {$-1.0$};
  \node at (0.75+\xoff,0.25-\yoff) {$4.0$};

  \draw[decoration={brace,mirror,raise=5pt},decorate,line width=1pt] (1.01,0.52) -- (1.01,1.0);
  \node at (1.3,0.752) {rank $=0$};
  \draw[decoration={brace,mirror,raise=5pt},decorate,line width=1pt] (1.01,0.0) -- (1.01,0.48);
  \node at (1.3,0.251) {rank $=1$};
\end{tikzpicture}
\bigskip
\caption{A parallel \pMat layout on two processes.  Blank entries are zeros, which are not stored because of ``compressed sparse'' storage.}
\label{fig:mpitwomatlayout}
\end{marginfigure}

We can have \PETSc show us the \pMat in different formats at the command line:
\begin{cline}
$ ./myprogram -mat_view
Mat Object: 1 MPI processes
  type: seqaij
row 0: (0, 4)  (1, -1)
row 1: (0, -1)  (1, 4)  (2, -1)
row 2: (1, -1)  (2, 4)  (3, -1)
row 3: (2, -1)  (3, 4)
$ ./myprogram -mat_view ::ascii_dense
Mat Object: 1 MPI processes
  type: seqaij
 4.00000e+00  -1.00000e+00  0.00000e+00  0.00000e+00
 -1.00000e+00  4.00000e+00  -1.00000e+00  0.00000e+00
 0.00000e+00  -1.00000e+00  4.00000e+00  -1.00000e+00
 0.00000e+00  0.00000e+00  -1.00000e+00  4.00000e+00
\end{cline}
The former view shows the compressed sparse storage, wherein only the nonzero values are shown, as pairs with column index and value, while the latter is a traditional (``dense'') display where zero values are shown.  In these cases the matrix was stored in \emph{serial} compressed sparse row format, the \texttt{MATSEQAIJ} type.

The result vector of a \pMat-\pVec product is, of course, a linear combination of the columns of the \pMat.  Thus, in practice, ``parallel row storage'' layout normally means two things:\begin{itemize}
\item \PETSc internally distributes the rows of the \pMat $A$ the same way as the entries of the intended output \pVec $b$, i.e.~if $Ax=b$ for some $x$, at least when \texttt{PETSC\_DECIDE} is used in setting the \pMat sizes, and
\item before \PETSc multiplies a matrix times a vector, \PETSc automatically communicates (``scatters'') the whole vector to each process.
\end{itemize}
After the scatter the \pMat-\pVec product is a local operation, requiring no further communication.

FIXME: assembly of \pMat a bit idiosyncratic; \texttt{MatSetValues()} sets a \emph{block} of values, not arbitrary inserts into sparse storage; need either MatXXXSetPreallocation() or MatSetUp() before MatGetOwnershipRange()


\section{Solve a linear system in \PETSc}

\cinputpartnostrip{c1matvec.c}{Initialize \PETSc and set up \pVecs and \pMat.}{I}{}{//ENDSETUP}{code:matvecpartone}

FIXME: emphasize that \texttt{MatGetOwnershipRange()} means that different processes are assembling different rows, unlike earlier example where all processes inserted all values

\cinputpartnostrip{c1matvec.c}{Assemble \pMat $A$.  Assemble right-hand side $b$ via exact solution to system.}{II}{//ENDSETUP}{//ENDASSEMBLY}{code:matvecparttwo}

\cinputpartnostrip{c1matvec.c}{Set up \pKSP.  Solve.  Finalize.}{III}{//ENDASSEMBLY}{//END}{code:matvecpartthree}

FIXME: show sparse and Matlab-format output for \texttt{A}, noting we get this at the \texttt{MatAssemblyEnd()} stage %$ ./c1matvec -a_mat_view ::ascii_matlab

FIXME: solve without choosing method

FIXME: show how to solve with Gaussian elimination


\section{A bit more numerical linear algebra}

Many methods other than Gauss elimination are possible for solving linear systems.  Usually these methods are iterative, and they often use the \emph{residual}.  By definition, if $\bu_0\in \RR^N$ is a vector then by the residual for $\bu_0$, as an estimate of the solution to equation \eqref{introsystem}, is the vector
\begin{equation}
\br_0 = \bb - A \bu_0. \label{residualdefn}
\end{equation}

Evaluating the residual for a known vector $\bu_0$ requires only applying $A$ to it, an $O(N^2)$ operation at most.  However, most discretization schemes for PDEs generate matrices $A$ that are \emph{sparse}, with many more zero entries than nonzeros, and often the number of nonzeros per row is independent of $N$.  In such cases the operation $A\bu_0$ can be implemented in $O(N)$ operations.

The \emph{Richardson iteration} is an example of an iterative method based on the residual.  If $\bu_0$ is an initial estimate of the solution then it simply adds a multiple $\omega$ of the residual at each step:
\begin{equation}
\bu_{k+1} = \bu_k + \omega (\bb - A \bu_k).  \label{introrichardson}
\end{equation}
If significantly fewer than $O(N^2)$ steps are needed to make $\bu_k$ an adequate approximation of the exact solution $\bu$, then the Richardson iteration can improve on Gauss elimination.  On the other hand, the Richardson iteration may not converge.

\newcommand{\rvect}[3]{\ensuremath{\bu_{#1} = \begin{bmatrix} #2 \\ #3 \end{bmatrix}}}

\medskip\noindent\hrulefill
\begin{example} Consider the linear system
\begin{equation}
A \bu
= \begin{bmatrix}
10 & -1 \\ -1 & 1
\end{bmatrix}
\begin{bmatrix} u_1 \\ u_2 \end{bmatrix}
= \begin{bmatrix} 8 \\ 1 \end{bmatrix}
= \bb
 \label{introexample}
\end{equation}
which has solution $\bu = [1\,\, 2]^\top$.  If we start with estimate $\bu_0 = [0\,\, 0]^\top$ then the unweighted ($\omega=1$) Richardson iteration \eqref{introrichardson} gives a sequence of vectors % see ../matlab/richardsonex.m
\begin{equation}
\rvect{0}{0}{0}, \rvect{1}{8}{1}, \rvect{2}{-63}{9}, \rvect{3}{584}{-62}, \dots
\end{equation}
This sequence is not heading toward the solution $\bu = [1\,\, 2]^\top$.
\end{example}
\noindent\hrulefill

If we rewrite \eqref{introrichardson} as
\begin{equation}
\bu_{k+1} = (I - \omega A) \bu_k + \omega \bb  \label{introrewriterichardson}
\end{equation}
then it is easy to believe that the ``size'' of the matrix $I-\omega A$ will determine whether $\lim_{k\to\infty} \bu_k$ exists, for generic starting vectors $\bu_0$.

To examine such questions, we recall the definitions of \emph{eigenvalue} and \emph{singular value}.  A complex number $\lambda \in \CC$ is an eigenvalue of a square matrix $B\in\RR^{N\times N}$ if there is a nonzero vector $\bv\in\CC^N$ so that $B \bv = \lambda \bv$.    The singular values are the square roots of the eigenvalues of the matrix $B^*B$.\sidenote{The matrix $B^*B$ is symmetric and positive-definite so its eigenvalues are nonnegative.}  Singular values are also geometrically-defined as the lengths of semi-axes of the ellipsoid in $\RR^N$ that results from applying $B$ to all vectors in the unit sphere of $\RR^N$ \citep{TrefethenBau}.

The set of all eigenvalues of $B$ is the \emph{spectrum} $\sigma(B)$ of $B$, and properties of matrices that can be described in terms of eigenvalues or singular values are generically called ``spectral properties.''  For example, recall that $\|B\|_2$ is equal to the largest singular value of $B$, while $\|B^{-1}\|_2$ is equal to the inverse of the smallest singular value of $B$.  The 2-norm condition number $\kappa(B)$ is the ratio of largest to smallest singular values,\sidenote{The condition number of $B$ well-visualized as the eccentricity of the ellipsoid we used in defining the singular values geometrically.}, and thus is a spectral property of $B$.

It is an easy exercise to show that the Richardson iteration \eqref{introrichardson} will converge if and only if all the eigenvalues of $B=I-\omega A$ are inside the unit circle.  Defining the \emph{spectral radius} $\rho(B)$ of a matrix $B$ as the maximum norm of the eigenvalues of $A$, we can describe the convergence of the Richardson iteration this way:
\begin{equation}
\text{\eqref{introrichardson} converges if and only if } \rho(I-\omega A) < 1. \label{introconvergethm}
\end{equation}
We can also show that $\rho(B) \le \|B\|_2$ so \eqref{introrewriterichardson} converges if $\|I-\omega A\|_2 < 1$, but this norm condition is merely sufficient, while \eqref{introconvergethm} is necessary and sufficient.

Other examples of iterative methods include the classical Jacobi and Gauss-Siedel iterations.  The most powerful methods generate optimal, in various senses \citep{TrefethenBau}, estimates $\bu_k$ which are each linear combinations of vectors $\bb,A\bb,A^2\bb,\dots,A^{k-1}\bb$.  These methods are collectively called \emph{Krylov space methods} because the span of these vectors is a Krylov space.  Typically the effectiveness of the iteration on a given matrix $A$ depends on the eigenvalues or singular values of $A$.  We use Krylov space methods in Chapter \ref{chap:structured} and all later Chapters.  Examples are conjugate gradients (CG) and minimum residual methods (e.g.~MINRES or GMRES) \citep{Greenbaum1997}.

There are many other systems which are equivalent to \eqref{introsystem}.  In fact, if $P\in\RR^{N\times N}$ is an invertible square matrix then the systems
\begin{equation}
(P^{-1} A) \bu = P^{-1} \bb \label{introleftpre}
\end{equation}
and
\begin{equation}
(A P^{-1}) (P\bu) = \bb \label{introrightpre}
\end{equation}
obviously have the same solution $\bu$ as \eqref{introsystem}.  However, matrices $P^{-1} A$ or $A P^{-1}$ may have different eigenvalues, condition numbers, and so on, namely different spectral properties, from $A$.  While the accuracy of the approximate solution $\bu$ cannot be improved beyond the $\kappa(A) \eps$ level---fact i) cannot be overcome in that sense---methods can indeed take advantage of better conditioning or other spectral properties to generate $\bu$ more quickly.  Especially if $P^{-1}$ is easy to apply\sidenote{The inverse matrix $P^{-1}$ is ``easy to apply'' exactly if the system $P\bv = \bc$ is easy to solve for $\bv$ in the sense of low computational cost.} then this can be an advantageous idea when trying to approximate $\bu$ quickly.  Equivalent systems \eqref{introleftpre} and \eqref{introrightpre} are referred to as \emph{preconditioned} systems, with \eqref{introleftpre} called \emph{left preconditioning} and \eqref{introrightpre} called \emph{right-preconditioning}.  Preconditioning can help in making our Richardson iteration example converge.  

\medskip\noindent\hrulefill
\begin{examplecont}  Suppose we use the diagonal matrix from  \eqref{introexample} as $P$:
\begin{equation}
P = \begin{bmatrix}
10 & 0 \\ 0 & 1
\end{bmatrix}.  \label{introP}
\end{equation}
Being diagonal, this $P$ is easy to invert and apply.  The preconditioned Richardson iteration using $P$, namely
\begin{equation}
\bu_{k+1} = \bu_k + \omega (P^{-1} \bb - P^{-1} A \bu_k),  \label{introprerichardson}
\end{equation}
is much better behaved.  With $\bu_0 = [0\,\, 0]^*$ again we get this sequence from \eqref{introprerichardson}:
\begin{equation}
\rvect{0}{0}{0}, \rvect{1}{0.8}{1.0}, \rvect{2}{0.9}{1.8}, \rvect{3}{0.98}{1.90}, \dots
\end{equation}
This sequence is apparently going to $\bu = [1\,\, 2]^*$.  Of course, the explanation is not hard to see; in this case
\begin{equation}
\rho(I-A) = -9.1, \qquad \rho(I-P^{-1} A) = 0.32.
\end{equation}
Convergence claim \eqref{introconvergethm} matches this bit of evidence.
\end{examplecont}
\noindent\hrulefill

Many iterative methods superior to the Richardson iteration are implemented in \PETSc.  We will use several of them, with choice of method only at run-time.


\caveat{But \Matlab is all you want if scale does not matter.}


\section{Exercises}

\renewcommand{\labelenumi}{\arabic{chapter}.\arabic{enumi}\quad}
\begin{enumerate}
\item Program \texttt{c1e.c} does a terrible job of load-balancing because the computation of the factorial $n!$ requires more flops when the process rank is larger.  Modify the code to balance the load almost perfectly, with exactly one multiply operation on each \texttt{rank>0} process, by using a blocking send and receive operations (\texttt{MPI\_Send(),MPI\_Recv()}) to pass the result of the last factorial to the next rank.  (\emph{Of course now we have a code that does a ridiculous amount of communication.})
% e1balanced.c
\item Modify \texttt{c1matvec.c} to read an integer option\sidenote{Use methods \texttt{PetscOptionsBegin()}, \texttt{PetscOptionsInt()}, and \texttt{PetscOptionsEnd()} so that \texttt{-help} output explains the option.} \texttt{-N} and form an $N\times N$ matrix, with $-2$ on the diagonal and $1$ on the super- and sub-diagonals as before.  (Choose any desired right-hand-side.)  Use \PETSc runtime options to estimate the condition number of the matrix for sample $N$ values; the online \PETSc FAQ page\sidenote{\texttt{www.mcs.anl.gov/petsc/documentation/faq.html}} will help with these options.
% see "How can I determine the condition number of a matrix?" on the PETSc FAQ page; "be sure to avoid restarts"
% -pc_type none -ksp_type gmres -ksp_monitor_singular_value -ksp_gmres_restart 1000
\end{enumerate}

\chapter{Finite-dimensional linear systems}
\label{chap:ls}
\renewcommand{\CODELOC}{ch2/}

\chapter{Finite linear systems}
\label{chap:linearsystem}

\section{The facts of (numerical) life}

Our goal is to compute more interesting quantities than Euler's constant $e$.  At the core of most \PETSc computations is a finite-dimensional linear system.  Before solving such systems in \PETSc, we recall the most basic ideas of numerical linear algebra.

Suppose $\bb\in \RR^{N}$ is a column vector and $A\in\RR^{N\times N}$ is a square matrix.  The linear system
\begin{equation}
A \bu = \bb \label{introsystem}
\end{equation}
has a unique solution if $A$ is invertible, namely
\begin{equation}
\bu = A^{-1} \bb. \label{introsolution}
\end{equation}
This is simple in theory.

It is not so simple in practice, however, to solve large systems on a computer.  There are two key facts to keep in mind while working numerically  \citep{TrefethenBau}:
\renewcommand{\labelenumi}{\roman{enumi})}
\begin{enumerate}
\item \label{limittoaccuracy} \emph{limit to accuracy}:  If real numbers are represented on the computer with machine precision $\eps$ then the solution of \eqref{introsystem} can only be computed within an error $\kappa(A) \eps$ where $\kappa(A) = \|A\|_2 \|A^{-1}\|_2$ is the (2-norm) \emph{condition number} of $A$.
\item \emph{cost of direct solutions}:  Computation of solution \eqref{introsolution} by a direct method like Gauss elimination,\sidenote{Or by QR, so that we have a \emph{backward stable} direct method.} whether actually forming $A^{-1}$ or not, is an $O(N^3)$ operation.
\end{enumerate}

Fact i) is about \emph{conditioning} not \emph{methods}.  Informally speaking, there are linear systems that are the same to within $\eps$ but for which the infinite-precision solutions $\bu$ differ by an amount $\kappa(A) \eps$.

On modern computers the precision for the C \texttt{double} type, the default 64-bit representation of real numbers, is $\eps = 2.2 \times 10^{-16}$.  Thus by i), a linear system having $\kappa(A) \approx 10^{10}$, for example, can only be solved to about six digits of precision.  While a matrix with such a large condition number is ``poorly-conditioned,'' it is easy to reach that level for $\kappa(A)$ when discretizing PDEs.  We have to be aware of conditioning when forming expectations about solution accuracy.

By ii) a generic linear system with $N=10^6$ equations requires $10^{18}$ or so operations to solve by Gauss elimination.  Even modern supercomputers take a while to do a quintillion operations.  However, though Gauss elimination is impractical for systems of $N=10^6$ equations, we will successfully solve PDE-generated linear systems of this size on a single processor in a few seconds, and in $O(N)$ operations, in Chapter \ref{chap:multigrid}.  The key fact that discretized PDEs generate linear systems with exploitable structure, especially \emph{sparsity} but also coefficient smoothness, will have to be exposed in serious numerical PDE solutions because naive application of direct methods is too slow.
%time ./c4poisson -da_refine 7 -ksp_type cg -pc_type mg
%on 1153 x 1153 grid:  iterations 2, residual norm = 1.88082e-05
%real 11.18


\section{\PETSc \pVec and \pMat objects}

To build our first \PETSc code to solve a linear system, we need the \emph{objects} which hold vectors (\pVec) and matrices (\pMat).  Note that, although \PETSc is written in C, and not C++ for example, it is a relentlessly object-oriented software library.  Consider the operations which create and configure a matrix object \texttt{A} for linear system \eqref{introsystem}:
\begin{code}
Mat A;
MatCreate(COMM,&A);
MatSetSizes(A,PETSC_DECIDE,PETSC_DECIDE,N,N);
MatSetOptionsPrefix(A,"a_");
MatSetFromOptions(A);
... fill entries of (i.e. assemble) A ...
... solve system with A ...
MatDestroy(&A);
\end{code}
We can think of these calls as using methods ``owned'' by the \pMat type, in the sense that they manipulate the internal representation of a \pMat, which is, happily, hidden from us.  If not true ``objects'', major \PETSc types are at least abstract data types with a hidden implementation.

The act of filling entries will treat a \PETSc \pMat as an abstract object.  In fact, a \pMat need not even have entries because it might, instead, contain code that applies a linear operator to vectors.  Furthermore the data structure inside a \pMat depends on runtime choices, the most basic being that the number of bytes used to store entries on a given MPI process will depend on the number of processes.

For all \PETSc object types this basic sequence of operations applies:\sidenote{Here ``\texttt{Object}'' is a meta-name for a \PETSc type like \pVec or \pMat.}
\begin{code}
Object X;
ObjectCreate(COMM,&X);
... code sets properties of X ...
ObjectSetFromOptions(X);  // allows run-time setting
                          // or overriding of properties
... code uses X ...
ObjectDestroy(&X);
\end{code}

Because \PETSc objects are generally distributed across, and accessible from, multiple MPI processes, the first argument of an \texttt{ObjectCreate()} method is an MPI communicator (``\texttt{COMM}'').  We will usually use \texttt{PETSC\_COMM\_WORLD} as it is the default communicator formed from all \texttt{N} processes when we start a run with ``\texttt{mpiexec -n N}.''  All processes in \texttt{COMM} must call the \texttt{ObjectCreate()} and  \texttt{ObjectDestroy()} methods.  That is, these are ``collective'' operations.

In the \pMat code above, note the call to \texttt{MatSetOptionsPrefix()}.  Through this ``prefix'', run-time options can address the particular \pMat object.  For example, the run-time option \texttt{-a\_mat\_view} will print out the entries of \texttt{A}.  An option prefix like ``\texttt{a\_}'' is especially helpful in distinguishing multiple \pMat objects at the command line.

Once \pMat \texttt{A} is created and set up by the first several commands \texttt{MatCreate()}--\texttt{MatSetFromOptions()}, then various methods become valid for \texttt{A}, for example including the \texttt{MatSetValues()} method to set entries in \texttt{A}.  We will use that method after looking at the parallel layout of vectors and matrices.


\section{Assembly and parallel layout of \pVecs and \pMats}

A \pVec or \pMat stores its entries in parallel across all the processes in the MPI communicator used when creating it.  For example, the create-assemble sequence of a \pVec with four entries might look like
\begin{code}
Vec x;
PetscInt   i[4] = {0, 1, 2, 3};
PetscReal  v[4] = {11.0, 7.0, 5.0, 3.0};

VecCreate(COMM,&x);
VecSetSizes(x,PETSC_DECIDE,4);
VecSetFromOptions(x);
VecSetValues(x,4,i,v,INSERT_VALUES);
VecAssemblyBegin(x);
VecAssemblyEnd(x);
\end{code}
The four entries of \texttt{Vec x} are set by \texttt{VecSetValues()}, putting values from array \texttt{v} at the indices given by \texttt{i}.

The operation of setting values in \texttt{x} may require communication between processes because entries which are to be stored on one process could be set by another process.  Such communication occurs between the \texttt{VecAssemblyBegin()} and \texttt{VecAssemblyEnd()} commands.

\begin{marginfigure}
\begin{tikzpicture}[scale=3.5]
  \pgfmathsetmacro\fourth{1.0/4.0}
  \pgfmathsetmacro\xoff{0.12}
  \pgfmathsetmacro\yoff{0.12}
  \draw[xstep=\fourth,ystep=\fourth,black,thick] (0.0,0.0) grid (\fourth,1.0);
  \node at (-0.2,1.0-\yoff) {$i=0$};
  \node at (\xoff,1.0-\yoff) {$11.0$};
  \node at (-0.2,0.75-\yoff) {$i=1$};
  \node at (\xoff,0.75-\yoff) {$7.0$};
  \node at (-0.2,0.5-\yoff) {$i=2$};
  \node at (\xoff,0.5-\yoff) {$5.0$};
  \node at (-0.2,0.25-\yoff) {$i=3$};
  \node at (\xoff,0.25-\yoff) {$3.0$};
  \draw[decoration={brace,mirror,raise=5pt},decorate,line width=1pt] (0.3,0.0) -- (0.3,1.0);
  \node at (0.65,0.51) {rank $=0$};
\end{tikzpicture}
\bigskip
\caption{A sequential \pVec layout, all on rank $=0$ process.}
\label{fig:seqveclayout}
\end{marginfigure}

The reader is allowed to think of a \PETSc \pVec as a one-dimensional C array with its contents split across the processes in the MPI communicator used in the \texttt{VecCreate()} command.  For example, if the above code appears in \texttt{mycode.c}, and if it is run sequentially on one process, i.e.~as
\begin{cline}
$ ./mycode.c
\end{cline}
%$
then, at the end of the above create-set-assemble sequence, the storage of \texttt{x} looks like Figure \ref{fig:seqveclayout}.  However, if run as
\begin{cline}
$ mpiexec -n 2 ./mycode.c
\end{cline}
%$
then the layout looks like Figure \ref{fig:mpitwoveclayout}.  In this case the argument \texttt{PETSC\_DECIDE} in \texttt{VecSetSizes()} is active, and the decision is to put the first two entries of \texttt{x} on the rank $0$ process and the other two on the rank $1$ process. 

\begin{marginfigure}
\begin{tikzpicture}[scale=3.5]
  \pgfmathsetmacro\fourth{1.0/4.0}
  \pgfmathsetmacro\xoff{0.12}
  \pgfmathsetmacro\yoff{0.12}
  \draw[xstep=\fourth,ystep=\fourth,black,thick] (0.0,0.0) grid (\fourth,1.0);
  \node at (-0.2,1.0-\yoff) {$i=0$};
  \node at (\xoff,1.0-\yoff) {$11.0$};
  \node at (-0.2,0.75-\yoff) {$i=1$};
  \node at (\xoff,0.75-\yoff) {$7.0$};
  \node at (-0.2,0.5-\yoff) {$i=2$};
  \node at (\xoff,0.5-\yoff) {$5.0$};
  \node at (-0.2,0.25-\yoff) {$i=3$};
  \node at (\xoff,0.25-\yoff) {$3.0$};
  \draw[decoration={brace,mirror,raise=5pt},decorate,line width=1pt] (0.3,0.52) -- (0.3,1.0);
  \node at (0.65,0.752) {rank $=0$};
  \draw[decoration={brace,mirror,raise=5pt},decorate,line width=1pt] (0.3,0.0) -- (0.3,0.48);
  \node at (0.65,0.251) {rank $=1$};
\end{tikzpicture}
\bigskip
\caption{A parallel \pVec layout on two processes.  Because we call ``\texttt{VecSetSizes(x,PETSC\_DECIDE,4)}'', \PETSc decides to split the storage in the middle.}
\label{fig:mpitwoveclayout}
\end{marginfigure}

\pMat objects, however, are not actually 2D C arrays even in serial (i.e.~in one-process runs).  Compared to \pVecs they require additional choices regarding parallel distribution.  Though this is hidden inside the implementation of \pMat, the most common storage format is \emph{parallel compressed sparse row storage}, what \PETSc calls the \texttt{MATMPIAIJ} type.  In this type a range of rows is owned by each process (parallel row storage), and within each owned range of rows only the specifically-allocated\sidenote{These are generally the nonzero entries, and usually referred-to as such.} entries are stored (sparse), and furthermore nonzero entries are stored contiguously in memory along with an additional index array (compressed).

\pMat objects are linear operators so their major purpose is to multiply \pVecs.  The result vector of a \pMat-\pVec product is a linear combination of the columns of the \pMat.  Thus, in practice, ``parallel row storage'' of the \pMat means these things:
\begin{itemize}
\item \PETSc internally distributes the rows of the \pMat $A$ the same way as the entries of the intended \emph{output} (i.e.~column) \pVec.  Thus if $Ax=b$ for some $x$ then row $i$ of $A$ is on the rank $m$ processor if and if entry $i$ of $b$ is on the rank $m$ processor.  (At least this can be expected when \texttt{PETSC\_DECIDE} is used in setting both the \pVec and \pMat sizes and they are of the same dimension.)
\item Before \PETSc computes a \pMat-\pVec product, \PETSc communicates (``scatters'') the whole \pVec to each process.
\item After the scatter the \pMat-\pVec product is a local operation, requiring no further communication.
\end{itemize}

But one doesn't really need to know all this to assemble a \pMat!  For example, here is how to create and assemble a $4\times 4$ \pMat one row at a time:
\begin{code}
Mat A;
PetscInt  i, j[4] = {0, 1, 2, 3};
PetscReal v[4];

MatCreate(PETSC_COMM_WORLD,&A);
MatSetSizes(A,PETSC_DECIDE,PETSC_DECIDE,4,4);
MatSetFromOptions(A);
MatSetUp(A);

i = 0;  v[0] = 1.0;  v[1] = 2.0;  v[2] = 3.0;
MatSetValues(A,1,&i,3,j,v,INSERT_VALUES);
i = 1;  v[0] = 2.0;  v[1] = 1.0;  v[2] = -2.0;  v[3] = -3.0;
MatSetValues(A,1,&i,4,j,v,INSERT_VALUES);
i = 2;  v[0] = -1.0;  v[1] = 1.0;  v[2] = 1.0;  v[3] = 0.0;
MatSetValues(A,1,&i,4,j,v,INSERT_VALUES);
j[0] = 1;  j[1] = 2;  j[2] = 3;
i = 3;  v[0] = 1.0;  v[1] = 1.0;  v[2] = -1.0;
MatSetValues(A,1,&i,3,j,v,INSERT_VALUES);

MatAssemblyBegin(A,MAT_FINAL_ASSEMBLY);
MatAssemblyEnd(A,MAT_FINAL_ASSEMBLY);
\end{code}
The method \texttt{MatSetValues()} sets multiple values, in this case a row.  The ``\texttt{1,\&i}'' arguments to \texttt{MatSetValues()} say that we are setting one row with global index \texttt{i}.  The ``\texttt{3,j}'' or ``\texttt{4,j}'' arguments say that we set the row using integer array \texttt{j} for the (global) column indices.

\begin{marginfigure}
\begin{tikzpicture}[scale=3.5]
  \pgfmathsetmacro\fourth{1.0/4.0}
  \pgfmathsetmacro\xoff{0.12}
  \pgfmathsetmacro\yoff{0.12}
  \draw[xstep=\fourth,ystep=\fourth,black,thick] (0.0,0.0) grid (1.0,1.0);

  \node at (-0.2, 1.0-\yoff) {$i=0$};
  \node at (-0.2,0.75-\yoff) {$i=1$};
  \node at (-0.2, 0.5-\yoff) {$i=2$};
  \node at (-0.2,0.25-\yoff) {$i=3$};

  \node at ( 1.0-\xoff, 1.1) {$j=3$};
  \node at (0.75-\xoff, 1.1) {$j=2$};
  \node at ( 0.5-\xoff, 1.1) {$j=1$};
  \node at (0.25-\xoff, 1.1) {$j=0$};

  \node at (\xoff,1.0-\yoff) {$1.0$};
  \node at (0.25+\xoff,1.0-\yoff) {$2.0$};
  \node at (0.5+\xoff,1.0-\yoff) {$3.0$};
  \node at (0.75+\xoff,1.0-\yoff) {};

  \node at (\xoff,0.75-\yoff) {$2.0$};
  \node at (0.25+\xoff,0.75-\yoff) {$1.0$};
  \node at (0.5+\xoff,0.75-\yoff) {$-2.0$};
  \node at (0.75+\xoff,0.75-\yoff) {$-3.0$};

  \node at (\xoff,0.5-\yoff) {$-1.0$};
  \node at (0.25+\xoff,0.5-\yoff) {$1.0$};
  \node at (0.5+\xoff,0.5-\yoff) {$1.0$};
  \node at (0.75+\xoff,0.5-\yoff) {$0.0$};

  \node at (\xoff,0.25-\yoff) {};
  \node at (0.25+\xoff,0.25-\yoff) {$1.0$};
  \node at (0.5+\xoff,0.25-\yoff) {$1.0$};
  \node at (0.75+\xoff,0.25-\yoff) {$-1.0$};

  \draw[decoration={brace,mirror,raise=5pt},decorate,line width=1pt] (1.01,0.52) -- (1.01,1.0);
  \node at (1.3,0.752) {rank $=0$};
  \draw[decoration={brace,mirror,raise=5pt},decorate,line width=1pt] (1.01,0.0) -- (1.01,0.48);
  \node at (1.3,0.251) {rank $=1$};
\end{tikzpicture}
\bigskip
\caption{A parallel \pMat layout on two processes.  Blank entries are not allocated.}
\label{fig:mpitwomatlayout}
\end{marginfigure}

If the above lines appeared in \texttt{mycode.c}, and if it was run
\begin{cline}
$ mpiexec -n 2 ./mycode
\end{cline}
%$
then the layout would be as in Figure \ref{fig:mpitwomatlayout}.

We can have \PETSc show us the entries in the \pMat in different formats at the command line:
\begin{cline}
$ ./mycode -mat_view
Mat Object: 1 MPI processes
  type: seqaij
row 0: (0, 1)  (1, 2)  (2, 3)
row 1: (0, 2)  (1, 1)  (2, -2)  (3, -3)
row 2: (0, -1)  (1, 1)  (2, 1)  (3, 0)
row 3: (1, 1)  (2, 1)  (3, -1)
$ ./mycode -mat_view ::ascii_dense
Mat Object: 1 MPI processes
  type: seqaij
 1.00000e+00  2.00000e+00  3.00000e+00  0.00000e+00
 2.00000e+00  1.00000e+00  -2.00000e+00  -3.00000e+00
 -1.00000e+00  1.00000e+00  1.00000e+00  0.00000e+00
 0.00000e+00  1.00000e+00  1.00000e+00  -1.00000e+00
\end{cline}
The first view shows the compressed sparse storage, with values as pairs with column index and value.  The second view is a traditional (``dense'') display where all zero values are shown, whether allocated or not.\sidenote{Other possibilities, not shown, are \texttt{-mat\_view ::ascii\_matlab}, which dumps in Matlab's text format, and \texttt{-mat\_view binary -viewer\_binary\_filename a.dat} which saves to file \texttt{a.dat} in \PETSc's scalable binary format.}  Note that \texttt{-mat\_view} output is activated by the \texttt{MatAssemblyBegin/End()} calls.

In these cases the matrix was stored in \emph{serial} compressed sparse row format, the \texttt{MATSEQAIJ} type, because of the one-process execution.  If the code is run in parallel, i.e.~by \texttt{mpiexec -n N ./mycode}, then \texttt{-mat\_view}  reports \texttt{type:~mpiaij} corresponding to \pMat type \texttt{MATMPIAIJ}.  This is the ``parallel compressed sparse row storage'' described above.


\section{Numerical linear algebra: residual, iteration, preconditioning}

Direct methods like Gauss elimination \citep{TrefethenBau} are one way to solve a linear system.  The most powerful methods for our later PDE-solving uses, however, are iterative.  They use the \emph{residual} of some approximation of the solution to the linear system.

By definition, the residual of $\bu_0$ in linear system \eqref{introsystem} is the vector
\begin{equation}
\br_0 = \bb - A \bu_0. \label{residualdefn}
\end{equation}
Evaluating the residual for a known vector $\bu_0$ requires only applying $A$ to it, an $O(N^2)$ operation at worst.  Because most discretization schemes for PDEs generate matrices $A$ that are \emph{sparse}, with many more zero entries than nonzeros, and because often the number of nonzeros per row is small and independent of $N$, the operation $A\bu_0$ and the evaluation of the residual can usually be done in $O(N)$ operations.

The \emph{Richardson iteration}, which simply adds a multiple $\omega$ of the last residual at each step,
\begin{equation}
\bu_{k+1} = \bu_k + \omega (\bb - A \bu_k),  \label{introrichardson}
\end{equation}
is an example of an iterative method based on the residual.  If significantly fewer than $O(N^2)$ steps are needed to make $\bu_k$ an adequate approximation of the exact solution $\bu$, then the Richardson iteration can improve on Gauss elimination.  On the other hand, the Richardson iteration may not converge, as in the next Example.

\newcommand{\rvect}[3]{\ensuremath{\bu_{#1} = \begin{bmatrix} #2 \\ #3 \end{bmatrix}}}

\medskip\noindent\hrulefill
\begin{example} Consider the linear system
\begin{equation}
A \bu
= \begin{bmatrix}
10 & -1 \\ -1 & 1
\end{bmatrix}
\begin{bmatrix} u_1 \\ u_2 \end{bmatrix}
= \begin{bmatrix} 8 \\ 1 \end{bmatrix}
= \bb
 \label{introexample}
\end{equation}
which has solution $\bu = [1\,\, 2]^\top$.  If we start with estimate $\bu_0 = [0\,\, 0]^\top$ then the $\omega=1$ Richardson iteration \eqref{introrichardson} gives a sequence of vectors % see ../matlab/richardsonex.m
\begin{equation}
\rvect{0}{0}{0}, \rvect{1}{8}{1}, \rvect{2}{-63}{9}, \rvect{3}{584}{-62}, \dots
\end{equation}
This sequence is not heading toward the solution.
\end{example}
\noindent\hrulefill

\medskip
If we rewrite \eqref{introrichardson} as
\begin{equation}
\bu_{k+1} = (I - \omega A) \bu_k + \omega \bb  \label{introrewriterichardson}
\end{equation}
then it is easy to believe that the ``size'' of the matrix $I-\omega A$ will determine whether $\lim_{k\to\infty} \bu_k$ exists.

To make this precise we recall important definitions.  A complex number $\lambda \in \CC$ is an \emph{eigenvalue} of a square matrix $B\in\RR^{N\times N}$ if there is a nonzero vector $\bv\in\CC^N$ so that $B \bv = \lambda \bv$.  The set of all eigenvalues of $B$ is the \emph{spectrum} $\sigma(B)$ of $B$.  The \emph{singular values} are the square roots of the eigenvalues of the matrix $B^*B$, a symmetric and positive-definite matrix with nonnegative eigenvalues.  Singular values are geometrically-defined as the lengths of semi-axes of the ellipsoid in $\RR^N$ that results from applying $B$ to all vectors in the unit ball of $\RR^N$ \citep{TrefethenBau}.

Properties of matrices described in terms of eigenvalues or singular values are generically called ``spectral properties.''  For example, because $\|B\|_2$ is equal to the largest singular value of $B$, while $\|B^{-1}\|_2$ is equal to the inverse of the smallest singular value of $B$, these are spectral properties.  The 2-norm condition number $\kappa(B)=\|B\|_2/\|B^{-1}\|_2$ is thus a spectral property as well; it is visualized as the eccentricity of the ellipsoid above.

It is an easy exercise to show that the Richardson iteration \eqref{introrichardson} will converge if and only if all the eigenvalues of $B=I-\omega A$ are inside the unit circle.  Defining the \emph{spectral radius} $\rho(B)$ as the maximum magnitude of the eigenvalues of $B$, we can describe the convergence of the Richardson iteration:
\begin{equation}
\text{\eqref{introrichardson} converges if and only if } \rho(I-\omega A) < 1. \label{introconvergethm}
\end{equation}
One can also show that $\rho(B) \le \|B\|_2$, so \eqref{introrewriterichardson} converges if $\|I-\omega A\|_2 < 1$.

Considering spectral properties brings us to the important general point that there are many other systems which are equivalent to \eqref{introsystem}.  In fact, if $P\in\RR^{N\times N}$ is an invertible square matrix then the systems
\begin{equation}
(P^{-1} A) \bu = P^{-1} \bb \label{introleftpre}
\end{equation}
and
\begin{equation}
(A P^{-1}) (P\bu) = \bb \label{introrightpre}
\end{equation}
have the same solution(s) $\bu$ as \eqref{introsystem}.  However, matrices $P^{-1} A$ or $A P^{-1}$ generally have different eigenvalues, condition numbers, and so on---different spectral properties---from $A$.  While the accuracy of the approximate solution $\bu$ cannot be improved beyond the $\kappa(A) \eps$ level, as fact i) on page \pageref{limittoaccuracy} cannot be overcome, methods can take advantage of better spectral properties to generate $\bu$ more quickly.  This idea is effective if $P^{-1}$ is easy to apply, in the sense of low computational cost.

Equivalent systems \eqref{introleftpre} and \eqref{introrightpre} are referred to as \emph{preconditioned} systems, with \eqref{introleftpre} called \emph{left preconditioning} and \eqref{introrightpre} called \emph{right-preconditioning}.  The next example shows how preconditioning can make the Richardson iteration converge.

\medskip\noindent\hrulefill
\begin{examplecont}  Suppose we use the diagonal matrix from  \eqref{introexample} as $P$:
\begin{equation}
P = \begin{bmatrix}
10 & 0 \\ 0 & 1
\end{bmatrix}.  \label{introP}
\end{equation}
Being diagonal, this $P$ is easy to invert and apply.  The preconditioned Richardson iteration using $P$, namely
\begin{equation}
\bu_{k+1} = \bu_k + \omega (P^{-1} \bb - P^{-1} A \bu_k),  \label{introprerichardson}
\end{equation}
is better behaved.  With $\bu_0 = [0\,\, 0]^*$ again we get this sequence from \eqref{introprerichardson}:
\begin{equation}
\rvect{0}{0}{0}, \rvect{1}{0.8}{1.0}, \rvect{2}{0.9}{1.8}, \rvect{3}{0.98}{1.90}, \dots
\end{equation}
This sequence is apparently going to $\bu = [1\,\, 2]^*$.  The explanation is not hard to see; compare
\begin{equation}
\rho(I-A) = -9.1, \qquad \rho(I-P^{-1} A) = 0.32.
\end{equation}
This example illustrates convergence claim \eqref{introconvergethm}.
\end{examplecont}
\noindent\hrulefill

\medskip
Preconditioning using the diagonal as in the above example is called \emph{Jacobi} preconditioning.  It is related to, but not the same idea as, the classical Jacobi iteration \citep{Greenbaum1997} for solving a linear system.


\section{Numerical linear algebra: Krylov space methods}

The most powerful iterative methods for solving system \eqref{introsystem} generate optimal, in various senses \citep{TrefethenBau}, estimates $\bu_k$ which are linear combinations of vectors $\bz,A\bz,A^2\bz,\dots,A^{k-1}\bz$ where $\bz$ is a fixed vector (which is often $\bz=\bb$).  These methods are collectively called \emph{Krylov space methods} because the span of such vectors is a \emph{Krylov space}.  Examples of such methods include the Richardson iteration, conjugate gradients (CG), and minimum residual methods (e.g.~MINRES or GMRES; \citet{Greenbaum1997,Saad2003}).  The classical Jacobi and Gauss-Siedel \citep{Greenbaum1997} iterative methods are not Krylov space methods because they involve extracting parts of (entries of) $A$ as a matrix.  The effectiveness of a given Krylov method on a system depends on the eigenvalues or singular values of the matrix $A$, i.e.~on its spectral properties.  As many Krylov space methods are built into and fully-supported by \PETSc, we will use them heavily in all later Chapters.

To be concrete, given $\bb$ the $n$th Krylov space in defined to be
    $$\mathcal{K}_n = \operatorname{span}\{\bb,A\bb,A^2\bb,\dots,A^{n-1}\bb\}.$$
Suppose $\tilde\bu \in \mathcal{K}_n$ so
    $$\tilde\bu = c_0 \bb + c_1 A \bb + c_2 A^2 \bb + \dots + c_{n-1} A^{n-1} \bb,$$
or equivalently
    $$\tilde\bu = p_{n-1}(A) \bb$$
for the $n-1$ degree polynomial $p_{n-1}(x) = c_0 + c_1 x + \dots + c_{n-1} x^{n-1}$ applied to $A$.  Thus if we want $\tilde\bu$ to approximate $\bu$, the solution to \eqref{introsystem}, then we want to find a polynomial $p_{n-1}$ so that
    $$p_{n-1}(A) \approx A^{-1}.$$
Krylov space methods do this, subject to the important idea that if $p_{n-1}(z)$ is close to $1/z$ \emph{on the finite set of eigenvalues of} $A$, i.e.~on the spectrum $\sigma(A)\subset \CC$, then $p_{n-1}(A) \approx A^{-1}$.  Thus, whether $p_{n-1}$ is a ``good'' polynomial for approximately inverting $A$ is a spectral question about $A$.

For example, consider the Richardson iteration \eqref{introrichardson}, namely $\bu_{k+1} = \bu_k + \omega (\bb - A \bu_k)$

FIXME

\begin{figure}
\bigskip
\includegraphics[width=0.9\textwidth]{richpolyplot}
\caption{(FIXME: improve caption and move to Chapter 1)  The Richardson iteration \eqref{introrichardson} is equivalent to approximating $\bu=A^{-1}\bb$ by using these polynomials $p_k(x)$ to approximate $p_k(A)\approx A^{-1}$.}
\label{fig:richpolyplot}
\end{figure}

FIXME: flesh out CG and GMRES

Most Krylov methods are implemented in \PETSc, and we get access to them via a \pKSP object.  Which method is used can be controlled at runtime, and runtime experimentation is always appropriate.  For the Poisson problem in Chapter \ref{chap:structured}, for example, we will see that preconditioned CG is very effective, but eventually we will put the most emphasis on the preconditioning stage and see that multigrid is extraordinarily effective for that purpose (see Chapter \ref{chap:multigrid}).  We will return to Krylov space methods repeatedly in later Chapters.

Our brief introduction of iterative linear algebra is hardly adequate.  But we need now to return to \PETSc codes and solve linear systems.  In the upcoming material the reader can first treat \PETSc's linear solver object as a black box, but then explore how it works through runtime options.

\vfill
\clearpage
\newpage
\section{Solve a small linear system in \PETSc}

We already know how to create, fill, and destroy \pVec and \pMat objects.  The next code, \texttt{c1vecmatksp.c} shown in Figure \ref{code:vecmatksp}, does these steps.  To actually solve a linear system it also uses a Krylov space method object ``\pKSP''.  This linear solver uses a method chosen at runtime.

Besides the expected \texttt{Create/SetFromOptions/Destroy} sequence for \pKSP, there are two important commands needed even for the simplest examples.  First we need to tell \pKSP about the matrix in the linear system by the command
\begin{code}
KSPSetOperators(ksp,A,A);
\end{code}
Then the system is actually solved by
\begin{code}
KSPSolve(ksp,b,x);
\end{code}
which also supplies the right-hand side of the system (i.e.~an allocated and assembled \pVec \texttt{b}) and space for the solution (i.e.~an allocated \pVec \texttt{x}).

Why do we list \texttt{A} twice in calling \texttt{KSPSetOperators()}?  Recall the linear system $A\bx=\bb$ is equivalent to $P^{-1} A \bx = P^{-1} \bb$ (left preconditioning).  At runtime we are going to choose a preconditioning method which builds $P^{-1}$ from $A$ or from an approximation of $A$.  For example, \emph{incomplete LU} factorization of $A$ can be used in generating $P^{-1}$, or $P$ could be the diagonal of $A$ (as in the example above).  The second matrix argument to \texttt{KSPSetOperators()} is the \pMat from which $P^{-1}$ is built.  Note we do not supply $P$ itself, as that would require the user to write extra code.  It would also block easy choice among preconditioners at runtime.  Instead we supply ``material'' from which the preconditioner is built, and the most common such material is $A$ itself.

After calling \texttt{KSPSolve()} we display the solution \pVec by calling \texttt{VecView()}.  This and everything else about the code in Figure \ref{code:vecmatksp} should now be self-explanatory.  So let's run it:
\begin{cline}
$ make c1vecmatksp
$ ./c1vecmatksp
Vec Object: 1 MPI processes
  type: seq
1
0
2
-1
\end{cline}
This solves the system and gives us $\bx$.  But what happened and how to control it?

\inputwhole{../c/ch2/vecmatksp.c}{\texttt{p4pdes/c/c1vecmatksp.c}}{Solve a small linear system.}{code:vecmatksp}


\section{Revealing solvers at runtime}

To start to see what happened when we ran \texttt{./c1vecmatksp} above, we could start by viewing the \pVec and \pMat objects.  The result of ``\texttt{./c1vecmatksp -vec\_view -mat\_view}'' is perfectly clear, but it tells us nothing about the solver, and it gives no hints on alternative ways of solving the equations.

Here is a key idea which cannot be over-emphasized:
\begin{quote}
\emph{learning \PETSc requires viewing \emph{solver} objects at runtime.}
\end{quote}
Here we try \texttt{-ksp\_view}; the output is slightly-clipped for clarity:
\begin{cline}
$ ./c1vecmatksp -ksp_view
KSP Object: 1 MPI processes
  type: gmres
    GMRES: restart=30, using Classical (unmodified) Gram-Schmidt ...
    GMRES: happy breakdown tolerance 1e-30
  maximum iterations=10000, initial guess is zero
  tolerances:  relative=1e-05, absolute=1e-50, divergence=10000
  left preconditioning
  using PRECONDITIONED norm type for convergence test
PC Object: 1 MPI processes
  type: ilu
    ILU: out-of-place factorization
    0 levels of fill
    tolerance for zero pivot 2.22045e-14
    matrix ordering: natural
    factor fill ratio given 1, needed 1
      ...
  linear system matrix = precond matrix:
  Mat Object:   1 MPI processes
    type: seqaij
    rows=4, cols=4
    total: nonzeros=16, allocated nonzeros=20
    total number of mallocs used during MatSetValues calls =0
      using I-node routines: found 1 nodes, limit used is 5
Vec Object: 1 MPI processes
  type: seq
1
0
2
-1
\end{cline}
%$
Here is some of what we learn:
\begin{itemize}
\item The default \pKSP solver is GMRES \citep{Saad2003}.\sidenote{This is \texttt{-ksp\_type gmres} as a command-line option.}
\item As GMRES builds an approximation of the solution from the Krylov space by saving more and more vectors, and thus uses up memory, one restarts after a number of iterations.  The default is \texttt{-ksp\_gmres\_restart 30}.
\item The default convergence tolerances for the \pKSP are \texttt{-ksp\_rtol 1.0e-5} and \texttt{-ksp\_atol 1.0e-50}.  In particular, \pKSP iterations stop when the residual norm has been reduced by $10^5$.
\item Inside the \pKSP is a \pPC preconditioner object.  We did not have to ask for one in the code, as every \pKSP has a \pPC.
\item The default \pPC is left preconditioning with ILU, i.e.~incomplete LU factorization, which is \texttt{-pc\_type ilu} as a runtime option.\sidenote{We will see in a moment that this is the \emph{serial} default \pPC, and there is more going on in parallel.}
\item Actually, the preconditioner is usually called ``ILU($0$)'', i.e.~with ``\texttt{0 levels of fill}'', so it does not use more memory than used by $A$ already.
\item The \pPC object has a copy of $A$ (``\texttt{Mat Object}''), because it was supplied as the second argument to \texttt{KSPSetOperators()} above.
\end{itemize}
Of course, we get the solution $\bx$ too.

While option \texttt{-ksp\_view} tells us what the solver \emph{is}, option \texttt{-ksp\_monitor} shows what it \emph{does}.  In this case, the \pKSP iteration is short:
\begin{cline}
$ ./c1vecmatksp -ksp_monitor
  0 KSP Residual norm 2.449489742783e+00
  1 KSP Residual norm 1.520235486122e-15
Vec Object: 1 MPI processes
...
\end{cline}
%$
The reason that the residual drops to nearly zero in one iteration is that GMRES sees a preconditioned system with an identity matrix in this case.  That is, $P^{-1} A = I$ because the ILU operation on this particular $A$ is actually a full LU factorization.  Our $A$ is banded and ``dense enough'' so that the LU factorization requires no fill-in.

For a small system like this a direct solve also works fine.  To see how to do it, first do
\begin{cline}
$ ./c1vecmatksp -help | grep ksp_type
  -ksp_type <gmres>: Krylov method (one of) cg ... preonly ... (KSPSetType)
$ ./c1vecmatksp -help | grep pc_type
  -pc_type <ilu>: Preconditioner (one of) none jacobi ... lu ... (PCSetType)
\end{cline}
to see the many solver and preconditioner options.  One of the preconditioners is ``\texttt{lu}'', suggesting a direct solve.  In this case we \emph{do not} want iterations, and one of the \pKSP options is ``\texttt{preonly}''.  Thus the direct solver combination we want is
\begin{cline}
$ ./c1vecmatksp -ksp_type preonly -pc_type lu
\end{cline}
%$
The reader should check that the output of \texttt{-ksp\_view} is as intended.\sidenote{The output of \texttt{-ksp\_monitor} is empty as there were, in fact, no iterations!}

When solving on multiple processors, using ILU($0$) as the default preconditioner makes no sense.  This is because, even when fill-in is avoided, the LU factorization algorithm would involve a great deal of interprocessor communication.  However, there are blocks along the diagonal of the system matrix $A$ which are entirely owned by a single processor.  The ILU($0$) method can be applied to these, and the result treated as an approximation $P^{-1}\approx A^{-1}$.  That is, the blocks along the diagonal can be approximately inverted by ILU($0$), and the result treated as a diagonal (i.e.~Jacobi) preconditioner.  So this is what is done by default \PETSc \pKSP settings in parallel:
\begin{itemize}
\item The default \pKSP is still GMRES with restart 30.
\item The default \pPC is \texttt{bjacobi}, i.e.~application of approximate diagonal-block inverses as $P^{-1}$.
\item Inside the \pPC is a \texttt{sub\_pc} object, also a \pPC, which is ILU($0$).\sidenote{There is even a \texttt{sub\_ksp} object, but it is \texttt{preonly}.  So let's just ignore that complication for now.}
\item These defaults correspond to options
\begin{quote}
\texttt{-ksp\_type gmres -pc\_type bjacobi -sub\_pc\_type ilu}
\end{quote}
\end{itemize}
The reader can confirm this understanding of the situation by running
\begin{cline}
mpiexec -n 2 ./c1vecmatksp -ksp_view
\end{cline}
But surely that's enough runtime options for now.  There will be more, especially in later chapters when we solve PDEs.

\section{More \PETSc functionality from the code side}

We have seen \PETSc code to set up and solve linear systems, but there is more to say.  The next example \texttt{c1tri.c}, split into Figures \ref{code:tripartone} and \ref{code:triparttwo}, introduces these additional concepts and their associated function calls:
\begin{enumerate}
\item Creating an option, so that the size of the linear system can be controlled at runtime, by using \texttt{PetscOptionsXXX()} calls.
\item Using \texttt{VecDuplicate()} for allocation.
\item Assembling a system of arbitrary size across an arbitrary number of processes, using \texttt{MatGetOwnershipRange()} to only set locally-owned rows.
\item Using \texttt{VecAXPY()}, \texttt{VecNorm()}, and \texttt{PetscPrintf()} to compute and display the error in a case where the exact solution is known.
\end{enumerate}

\vfill
\newpage
\cinputpartnostrip{ch2/tri.c}{Set up \pVec and \pMat objects for a tridiagonal system.}{I}{//STARTSETUP}{//ENDSETUP}{code:tripartone}

First build (``\texttt{make c1tri}'') and run the code with options that show what it is does by default:
\begin{cline}
$ ./c1tri -ksp_monitor -a_mat_view ::ascii_dense
Mat Object:(a_) 1 MPI processes
  type: seqaij
  3.00000e+00  -1.00000e+00   0.00000e+00   0.00000e+00 
 -1.00000e+00   3.00000e+00  -1.00000e+00   0.00000e+00 
  0.00000e+00  -1.00000e+00   3.00000e+00  -1.00000e+00 
  0.00000e+00   0.00000e+00  -1.00000e+00   3.00000e+00 
  0 KSP Residual norm 3.302822756884e+00 
  1 KSP Residual norm 5.519370044893e-16 
error for m = 4 system is |x-xexact|_2 = 5.1e-16
\end{cline}
%$

To get started on the new concepts shown in Figure \ref{code:tripartone}, notice that methods \texttt{PetscOptionsBegin()} and \texttt{PetscOptionsEnd()} bracket the call to \texttt{PetscOptionsInt()}.  We set an option prefix ``\texttt{-tri\_...}'' so that the new option we create is distinguished from the many built-in \PETSc options that start with \texttt{-ksp\_...} or \texttt{-vec\_...} or whatever.  Our use of \texttt{PetscOptionsInt()} creates an option \texttt{-tri\_m} which allows the user to set the variable \texttt{m}, and to leave the default \texttt{m}$=4$ unaltered if the option is not set.

To see the option and its default value in \texttt{-help} output do
\begin{cline}
$ make c1tri
$ ./c1tri -help | grep tri_
Solve a tridiagonal system of arbitrary size.  Option prefix = tri_.
  -tri_m <4>: dimension of linear system (None)
\end{cline}
Here, for instance, we reset the system size to be small and we view the matrix:
\begin{cline}
$ ./c1tri -tri_m 2 -a_mat_view
Mat Object:(a_) 1 MPI processes
  type: seqaij
row 0: (0, 3)  (1, -1) 
row 1: (0, -1)  (1, 3) 
error for m = 2 system is |x-xexact|_2 = 8.0e-16
\end{cline}
%$

After setting up the new option, next in \texttt{c1tri.c} the numerical solution \pVec \texttt{x} is created just as we did in the last example \texttt{c1vecmatksp.c} (Figure \ref{code:vecmatksp}).  But now we also want to create \pVecs \texttt{b} and \texttt{xexact}; the former is the right-hand side of the linear system and the latter holds the exact solution to the linear system so that we can evaluate the error in the numerical solution.  Though at this stage we have not filled any entries of \texttt{x}, we can save repetitive function calls to \texttt{VecCreate/SetSizes/SetFromOptions()} when creating these additional \pVecs.  That is, because they have the same properties and parallel layout we can use \texttt{VecDuplicate()} to allocate \pVecs just like \texttt{x}.\sidenote{There is a different method for copying the contents of \pVecs, namely \texttt{VecCopy()}.  It requires that the two \pVecs are already allocated and have the same layout.}

Next we assemble the matrix $A$.  This is a boring tridiagonal matrix with $3$ on the diagonal and $-1$ in the super- and sub-diagonals.  Though boring, we want to assemble it efficiently in parallel, something that will be important when solving 2D and 3D PDEs in later chapters.  However, only when \texttt{c1tri.c} is run do we know how many processes are in use.  A key method for efficient matrix assembly in parallel is the method \texttt{MatGetOwnershipRange()} which tells our program, as it is running on a particular process (rank), what rows it owns locally.  In the case of a many structured matrices like this one, we can avoid all interprocess communication by assembling exactly the rows we own.  As seen at the top of Figure \ref{code:triparttwo}, we call
\begin{quote}
\texttt{MatGetOwnershipRange(A,\&Istart,\&Iend)}
\end{quote}
to get the starting and ending row indices for the local process.  These are used as limits in a \texttt{for} loop over the locally own rows, wherein we call \texttt{MatSetValues()} to actually set the entries of $A$.  We call \texttt{MatAssemblyBegin/End()} to complete the assembly of $A$.

\cinputpartnostrip{ch2/tri.c}{Assemble and solve.}{II}{//ENDSETUP}{//ENDSOLVE}{code:triparttwo}

We need to assemble the right-hand side $\bb$ of the linear system.  But this is related to the question: how does one get the exact solution $\bx_{\text{exact}}$ to a linear system $A\bx_{\text{exact}} = \bb$?  The easiest way for demonstration purposes, as here, is to \emph{choose} $\bx_{\text{exact}}$ and then compute $\bb$ by multiplying by $A$.  Thus, we set (unimportant) values for \texttt{xexact}, and call \texttt{VecAssemblyBegin/End()} on it.  Then we compute \texttt{b} by calling \texttt{MatMult(A,xexact,b)}.

As in the earlier code, we set up  the \pKSP and then call \texttt{KSPSolve()} to approximately solve $A\bx = \bb$.  As the \pKSP is working we can see it reduce the $k$th-iteration residual norm $\|\bb-A\bx_k\|_2$ just by using option \texttt{-ksp\_monitor}.  In this case we also want to see that the actual error
	$$\|\bx - \bx_{\text{exact}}\|_2$$
is small when the \pKSP completes its work.  So, after getting \texttt{x} from \texttt{KSPSolve()} we compute the error with two commands,

\medskip
\begin{tabular}{lcrcl}
\text{\texttt{VecAXPY(x,-1.0,xexact)}}       & : & $\bx$                   & $\leftarrow$ & $-1\, \bx_{\text{exact}} + \bx$ \\
\text{\texttt{VecNorm(x,NORM\_2,\&errnorm)}} & : & \text{\texttt{errnorm}} & $\leftarrow$ & $\|\bx\|_2$.
\end{tabular}

\medskip
\noindent and then print \texttt{errnorm} by calling \texttt{PetscPrintf()}.


\section{A first look at performance}

The linear system assembled by \texttt{c1tri.c} is about as easy to solve as they get.   It is tridiagonal, symmetric, diagonally-dominant, and positive definite.\sidenote{Regarding the last property, see exercise \ref{exer:computeeigs}.}  So \PETSc ought to be able to solve it quickly, and parallelization ought to be effective.

\newcommand{\WORKSTATION}{\textsc{workstation}\xspace}
We can time a one-process solution:\sidenote{On the author's 2012-era workstation with 16 GB memory and a single Intel Core i7-3820 8-core cpu running at 3.60GHz.  We call this machine \WORKSTATION. \label{defineworkstation}}
\begin{cline}
$ time ./c1tri -tri_m 10000
error for m = 10000 system is |x-xexact|_2 = 8.0e-13
real 0.04
user 0.04
sys 0.00
\end{cline}
%$
Only the ``\texttt{real}'' time is worth considering, so from now on we will use an alias to get just this time:
\begin{cline}
$ alias timer='time -f "real %e"'
\end{cline}
%$

The above timing result of $.04$ seconds suggests that this $m=10^4$ dimension system is not yet big enough to be worth testing, so we experiment on what $m$ gives a linear system with noticeable solution time.  Here is a roll-your-own bash loop\sidenote{I have found \href{http://mywiki.wooledge.org/FullBashGuide}{mywiki.wooledge.org/ FullBashGuide} to be a reasonable online overview of bash.} to time some bigger solves:
\begin{cline}
$ for M in 10000 100000 1000000 10000000; do timer ./c1tri -tri_m $M; done
error for m = 10000 system is |x-xexact|_2 = 8.0e-13
real 0.04
error for m = 100000 system is |x-xexact|_2 = 3.1e-12
real 0.18
error for m = 1000000 system is |x-xexact|_2 = 4.8e-11
real 1.17
error for m = 10000000 system is |x-xexact|_2 = 3.5e-10
real 11.96
\end{cline}
Thus the default methods\sidenote{Recall: GMRES with ILU($0$) preconditioning.} solve a system of size $m=10^7$ (ten million) in a time of about 12 seconds.

It is reasonable to ask if other methods and/or more processors speed up the solution of the linear system of size $m=10^7$.  In particular we can try a direct solve (\texttt{-ksp\_type preonly -pc\_type lu}), the default serial solver (\texttt{-ksp\_type gmres -pc\_type ilu}), and the default parallel solver (\texttt{-ksp\_type gmres -pc\_type bjacobi -sub\_pc\_type ilu}).  We can turn off preconditioning (\texttt{-pc\_type none}) or switch from ILU($0$)-based preconditioning to Jacobi preconditioning (\texttt{-pc\_type jacobi}). 

To generate Table \ref{tab:c1tritiming} below, we did:
\begin{cline}
$ timer mpiexec -n N ./c1tri -tri_m 10000000 -ksp_rtol 1.0e-10 -ksp_type KSP -pc_type PC
\end{cline}
%$
for $N=1$ and $N=4$, and several \pKSP/\pPC choices.  The tighter tolerance \texttt{-ksp\_rtol 1.0e-10} was added because a fair comparison of iterative and direct methods needs significant accuracy in the iterative ones.

\newcommand{\intime}[1]{\input{../c/ch2/timing/tri/#1}}
\begin{figure}[ht]
\texttt{
\begin{tabular}{llll}
\underline{KSP}\hspace{0.5in} & \underline{PC}\hspace{0.8in} & \underline{N=1 \textrm{time (s)}}\hspace{0.3in} & \underline{N=4 \textrm{time (s)}} \\
preonly    & lu          & \intime{preonly.lu.1}       &  \\
           & cholesky    & \intime{preonly.cholesky.1} &  \\
richardson & jacobi      & \intime{richardson.jacobi.1}& \intime{richardson.jacobi.4} \\
gmres      & none        & \intime{gmres.none.1}       & \intime{gmres.none.4} \\
           & jacobi      & \intime{gmres.jacobi.1}     & \intime{gmres.jacobi.4} \\
           & ilu         & \intime{gmres.ilu.1}        &  \\
           & bjacobi/ilu &                             & \intime{gmres.bjacobi.4} \\
cg         & none        & \intime{cg.none.1}          & \intime{cg.none.4} \\
           & jacobi      & \intime{cg.jacobi.1}        & \intime{cg.jacobi.4} \\
           & icc         & \intime{cg.icc.1}           &  \\
           & bjacobi/icc &                             & \intime{cg.bjacobi.4} \\
\end{tabular}
}
\caption{Times for \texttt{c1tri.c} to solve systems of dimension $m=10^7$.  In this case the matrix is \emph{tridiagonal}, \emph{symmetric}, \emph{diagonally-dominant}, and \emph{positive definite}.  All runs were on \WORKSTATION (see page \pageref{defineworkstation}).} \label{tab:c1tritiming}
\end{figure}

This table deserves discussion:
\begin{itemize}
\item Entries ``\texttt{bjacobi/ilu}'' and ``\texttt{bjacobi/icc}'' correspond to options \texttt{-pc\_type bjacobi -sub\_pc\_type ilu} and \texttt{-pc\_type bjacobi -sub\_pc\_type icc}, respectively.  As already noted, the first of these is the parallel default.\sidenote{This table, at least, suggests it is a reasonable choice for that purpose.}
\item Counting floating point operations shows the leading-order work for Cholesky is $m^3/3$ while for LU it is $2 m^3/3$ \citep{TrefethenBau}.  This is reflected in the direct-solve results (with \texttt{-ksp\_type preonly}).  A diagonally-dominant tridiagonal problem is very well-behaved for direct methods, however, because no fill-in or pivoting occurs when they are applied.\sidenote{Do not assume that most $m=10^7$ dimension systems are good candidates for direct solves!}
\item Cholesky and conjugate gradient methods apply to symmetric positive-definite matrices only, which is what \texttt{c1tri.c} assembles.  These are \pKSP method \texttt{cg} and \pPC methods \texttt{cholesky} and \texttt{icc} (incomplete-Cholesky, also denoted ``ICC($0$)'').  We see some benefit to using CG on one-process runs, compared to the general (``non-symmetric'') method GMRES, but the matrix generated by \texttt{c1tri.c} is so well-behaved that both of these iterations work well.
\item There is indeed speed-up from $N=1$ to $N=4$ processes, on this single node 8-core workstation, for many methods.  The 2 to 3 times speed-up is relatively uniform across methods, but (again) the fact that the matrix is very well-structured makes this un-surprising.
\end{itemize}

In future Chapters, from ``real'' linear and non-linear systems generated from discretizing PDEs, we will create timing tables like Table \ref{tab:c1tritiming}, and re-consider the results.

\begin{marginfigure}
\bigskip
\includegraphics[width=0.9\textwidth]{line-graph-c1tri}
\caption{\PETSc can use X windows to produce line graphs at run time.  (This is not to say they are pretty.)}
\label{fig:line-graph-c1tri}
\end{marginfigure}

Finally, \PETSc can generate graphics showing convergence of iterative methods, at least if X windows are installed.  The line graph in Figure \ref{fig:line-graph-c1tri}, from
\begin{cline}
$ ./c1tri -tri_m 1000000 -ksp_rtol 1.0e-10 -pc_type jacobi \
    -ksp_monitor_lg_residualnorm -draw_pause 1
\end{cline}
%$
shows the residual norm logarithm versus the iteration number.

\bigskip
\section{Exercises}

\renewcommand{\labelenumi}{\arabic{chapter}.\arabic{enumi}\quad}
\begin{enumerate}
\item Show \eqref{introconvergethm}.
\item As the reader will undoubtedly experience, segmentation faults and memory leaks are inevitable when one develops \PETSc codes.  A standard tool for detecting/diagnosing these is \texttt{valgrind}.\sidenote{\href{http://valgrind.org/}{valgrind.org}}  We recommend running it frequently.  As an exercise, run
\begin{cline}
valgrind ./c1vecmatksp
\end{cline}
to see what \texttt{valgrind} shows for a leak-free program.  Then comment-out a \texttt{VecDestroy()} call in \texttt{c1vecmatksp.c} and rerun to see a common type of memory leak.
\item Consider Richardson iteration in the example
\begin{cline}
./c1tri -tri_m 100 -ksp_monitor -ksp_type richardson
\end{cline}
Un-preconditioned Richardson iteration fails (i.e.~add \texttt{-pc\_type none}); explain.  The default preconditioner succeeds (i.e.~\texttt{-pc\_type ilu}), but ILU($0$) is cheating because it becomes a complete LU factorization on this tridiagonal and diagonally-dominant $A$; we are really seeing a direct solve.  The same can be said for ICC($0$).  Confirm that, as in the example on page \pageref{introprerichardson}, Richardson iteration succeeds with \texttt{-pc\_type jacobi}, even though the diagonal is constant; explain.
\item \label{exer:computeeigs} Un-preconditioned GMRES solves the linear system in \texttt{c1tri.c} reasonably efficiently.  We can explain this by asking \PETSc to compute the eigenvalues of $A$ by using option\sidenote{The relevant \PETSc manual page says this option is ``intended only for assistance in understanding the convergence of iterative methods, not for eigenanalysis.  For accurate computation of eigenvalues we recommend using the excellant package SLEPc.''}
\begin{quote}
\texttt{-ksp\_compute\_eigenvalues}
\end{quote}
Because otherwise it computes the eigenvalues of the preconditioned operator $P^{-1}A$, add \texttt{-pc\_type none}.  Try dimensions $N=10,100,1000$.  Why does the  run
\begin{cline}
./c1tri -tri_n 1000 -pc_type none -ksp_compute_eigenvalues
\end{cline}
only show 11 eigenvalues of this $1000\times 1000$ matrix?  How do these eigenvalues explain the good behavior of unpreconditioned GMRES?\sidenote{See \citep{TrefethenBau} for help with both of these questions.}
\item The accuracy of direct solves (e.g.~\texttt{-ksp\_type preonly -pc\_type cholesky}) in \texttt{c1tri.c}, as measured by the reported error norm $\|\bx - \bx_{\text{exact}}\|_2$, decreases with increasing dimension.  Confirm and explain this observation.
\item Table \ref{tab:c1tritiming} includes a number of blanks.  For each one, explain why it is blank, experimenting if needed.
\item Table \ref{tab:c1tritiming} gives execution times not iteration count.  Generate the corresponding table of \pKSP iteration count by adding option \verb|-ksp_converged_reason| to the run commands.  Note the large number of ``coincidences'' in iteration count, i.e.~cases where iteration counts are identical; explain.  Which preconditioners have a strong or weak effect on iteration count?  Is timing or iteration count more useful?
\end{enumerate}

\chapter{Poisson equation on a structured grid}
\label{chap:st}
\renewcommand{\CODELOC}{ch3/}

\chapter{2. Linear PDEs: structured grids}

We start with a cliched example, the Poisson problem, because it is the right place to start.  Though the Poisson problem is a cliche in applied mathematics, it gives us the opportunity to use \PETSc several different ways, starting here with building a structured grid using \PETSc \pDMDA, and solving in parallel using a \pKSP object.

\section{Poisson-Dirichlet problem}

\cinputpart{c2poisson.c}{Set up \pDMDA object.}{I}{//CREATEGRID}{//ENDCREATEGRID}{code:ctwopoissoncreategrid}

\cinputpart{c2poisson.c}{Fill matrix entries using \texttt{MatSetValuesStencil}.}{II}{//CREATEMATRIX}{//ENDCREATEMATRIX}{code:ctwopoissoncreatematrix}

\cinputpart{c2poisson.c}{Solve using \pKSP, and report on solution.}{III}{//SOLVE}{//ENDSOLVE}{code:ctwopoissonsolve}

\section{Runtime control of linear solver}

FIXME: basic Krylov theory

\section{Time-dependent heat equation}

FIXME: we WON'T do explicit, but it would look like ...

FIXME: use TS for backward-euler


\chapter{Nonlinear equations}
\label{chap:nl}
\renewcommand{\CODELOC}{ch4/}

The simplest thing to say about nonlinear equations is that they change the functional form of the residual when compared to the linear case.  For a linear system the residual $\br$ is a certain linear (affine) function of the unknowns $\bu$, namely $\br = \bb - A \bu$.  Now we consider cases in which the residual, as a function of $\bu$, might be a polynomial, a transcendental function, or something more general.

Suppose that $\bF : \RR^N \to \RR^N$ is differentiable.  The input $\bx$ and output $\bF(\bx)$ are column vectors,\sidenote{Our name change $\bu\to\bx$ comes from now thinking more geometrically about the location of $\bx$.} so in that sense $\bF$ acts like square-matrix multiplication $\bx\mapsto A\bx$.  Just as an iterative linear solver reduces the linear residual $\br_k=\bb - A \bu_k$ to zero by generating a sequence $\bu_k$, for nonlinear $\bF$ we want to solve
\begin{equation}
   \bF(\bx) = 0   \label{eq:nl:equation}
\end{equation}
by iteration, generating approximations $\bx_k$ so that $\bF(\bx_k)$ goes to zero.

Newton's method linearizes \eqref{eq:nl:equation} around the most recent iterate $\bx_k$ and then it ``moves'' the new iterate $\bx_{k+1}$ to the location which solves the linear problem.  The new location is, one hopes, closer to the solution of \eqref{eq:nl:equation}.  Each iteration requires solving a linear system, and we already have \PETSc technology for that.  However, the cost of performing the linearization must be taken into account, choosing a smart distance to move will require additional choices, and all existing choices regarding the linear solver---especially preconditioning---remain active.  Indeed, solving a nonlinear system of equations generally requires all the tools for linear systems considered in Chapter \ref{chap:ls}, and more.

Large systems of nonlinear equations often arise in applications from PDE-type physical models with nonlinearities.  However, the current Chapter is largely about finite-dimensional systems of nonlinear equations \eqref{eq:nl:equation} as problems of their own.  Late in this Chapter we will, however, give an example of a discretized one-dimensional nonlinear PDE.


\section{Newton's method}

Suppose $\bx_k\in\RR^N$ is an approximation, whether good or bad, to the solution of \eqref{eq:nl:equation}.  Let $\bs\in\RR^N$.  Because $\bF$ is differentiable, by definition
\begin{equation}
    \bF(\bx_{k}+\bs) = \bF(\bx_k) + J_\bF(\bx_k) \bs + o(\|\bs\|)  \label{eq:nl:expandF}
\end{equation}
for some square matrix
\begin{equation}
J_\bF(\bx_k) = \begin{bmatrix}
    \frac{\partial F_0}{\partial x_0} & \dots & \frac{\partial F_0}{\partial x_{N-1}} \\
    \vdots & \ddots & \vdots \\
    \frac{\partial F_{N-1}}{\partial x_0} & \dots & \frac{\partial F_{N-1}}{\partial x_{N-1}}  \end{bmatrix}  \label{eq:nl:jacdefn}
\end{equation}
and some quantity $o(\|\bs\|)$ that goes to zero as the \emph{step length} $\|\bs\|$ goes to zero.  The matrix $J_\bF(\bx_k)$ is called the \emph{Jacobian} of $\bF$ at $\bx_k$; we also call the function $\bx \mapsto J_\bF(\bx)$ the Jacobian.

If $\bs$ is regarded as a \emph{step}, away from the current iterate $\bx_k$, then $\bx_{k+1} = \bx_k + \bs$ is the next iterate.  Each iteration of Newton's method approximately solves \eqref{eq:nl:equation} by truncating \eqref{eq:nl:expandF}---removing ``$+o(\|\bs\|)$''---and seeking $\bs$ so that the updated value $\bF(\bx_{k+1})$ would be zero if the truncated equation were true.  That is, each Newton step computes $\bs$ by the linear equation
\begin{equation}
    0 = \bF(\bx_k) + J_\bF(\bx_k) \bs.
\end{equation}
Writing the linear system in form ``$A\bu=\bb$,'' we see that an iteration of Newton's method requires solving a linear system and then doing a vector addition:
\refstepcounter{equation}\label{eq:nl:newton}  % \eqref{eq:nl:newton} for both eqns
\begin{align}
    J_\bF(\bx_k) \bs &= - \bF(\bx_k)  \label{eq:nl:newton:eq} \tag{\theequation a} \\
    \bx_{k+1} &= \bx_k + \bs  \label{eq:nl:newton:update} \tag{\theequation b}
\end{align}
This is Newton's method.  It is fairly simple in theory.

Actual practice for Newton's method is also not that complicated with \PETSc in hand, because it has a mature implementation of this core technology.  In fact, despite the reputation of Newton iteration as fragile or scary, it works just fine on many nonlinear systems once one adds some protections.  Primarily, a line search will, as needed, move a shorter distance to the new iterate than stated in \eqref{eq:nl:newton:update}; see page \pageref{sec:linesearch}.

Finite-dimensional nonlinear systems \eqref{eq:nl:equation} can be visualized as the problem of finding the intersections of curves, surfaces, or hypersurfaces, depending on dimension.  A small example gives us a start on the details.

\clearpage
\noindent\hrulefill
\begin{example}  Given parameter $b > 1$, the nonlinear equations
    $$y = \frac{1}{b} e^{bx}, \qquad x^2+y^2 = 1,$$
form intersecting curves in the plane.  The curves intersect twice, as shown in Figure \ref{fig:expcirclebasic}.  These equations are put in standard form \eqref{eq:nl:equation} by writing

\begin{marginfigure}
\includegraphics[width=1.2\textwidth]{figs/expcirclebasic}

\medskip
\caption{Newton iterates $\bx_k$ approach a solution of $\bF(\bx)=0$ for $\bF$ in \eqref{eq:nl:expcircleF} and $b=2$.}
\label{fig:expcirclebasic}
\end{marginfigure}

\begin{equation}
\label{eq:nl:expcircleF}
\bF(\bx) = \begin{bmatrix}
           \frac{1}{b} e^{b x_0} - x_1 \\
           x_0^2 + x_1^2 - 1
           \end{bmatrix}
\end{equation}
for $\bx\in \RR^2$ a column vector: $\bx = [x_0 \,\,\, x_1]^\top$.  Thus
\begin{equation}
\label{eq:nl:ecjacobian}
J_\bF(\bx) = \begin{bmatrix}
    e^{b x_0} & & -1 \\
    2 x_0   & & 2 x_1 \end{bmatrix}
\end{equation}
As also shown in Figure \ref{fig:expcirclebasic}, for $b=2$, if we start the Newton iteration with $\bx_0 = [1 \,\, 1]^\top$ then the sequence of iterates from \eqref{eq:nl:newton} is
    $$\twovect{\bx}{0}{1}{1}, \quad \twovect{\bx}{1}{0.619203}{0.880797}, \quad \twovect{\bx}{2}{0.394157}{0.948623}, \quad \dots$$
\noindent\hrulefill
%  FROM $ for N in 0 1 2; do ./expcircle -snes_fd -snes_max_it $N; done
\end{example}


\section{Using \pSNES with call-backs} \label{sec:usingsnes}

We will compute the Newton iterates in the example above by using a nonlinear solver object of type \pSNES\sidenote{SNES stands for ``scalable nonlinear equation solver.''} from \PETSc.  Note \pSNES uses the usual \texttt{Create/SetFromOptions/Destroy} sequence.  Our code provides a function $\bF$ which is a ``call-back'' in the sense that we supply it to the \pSNES, which then calls it with argument $\bx$ when it needs $\bF(\bx)$ during the Newton iteration.

Later we will also provide the \pSNES with a function which computes the Jacobian function $J_\bF$.  However, the Jacobian can be approximated by repeated $\bF$ evaluations because a derivative can also be approximated by finite differences.  Thus our first code avoids such a Jacobian ``call-back.''

The whole of \texttt{expcircle.c}, to solve the above Example, is in Code \ref{code:expcircle}.  The \texttt{main()} function starts by allocating \pVec \texttt{x} of fixed dimension 2 which will hold the iterates.  (Once the Newton iteration is ended it holds the converged estimate of the solution.)  Because both components of $\bx_0$ are $1$, it is initialized using \texttt{VecSet()}.  Next a duplicate \pVec \texttt{r} is created because the \pSNES needs it as space to store the (nonlinear) residual.

Then the \pSNES is created and configured.  Formula \eqref{eq:nl:expcircleF} is in method \texttt{FormFunction()}, which is supplied to the \pSNES object using \texttt{SNESSetFunction()}.  We call \texttt{SNESSetFromOptions()} because it gives us run-time control on how the Jacobian is calculated\sidenote{Options \texttt{-snes\_fd} and \texttt{-snes\_mf} are allowed; see Table \ref{tab:snesjacobianoptions:needed} below.} and on how the length of the step $\bs$ is determined.\sidenote{Through \texttt{-snes\_linesearch\_type} and related options; see page \pageref{sec:linesearch}.}  Then \eqref{eq:nl:equation} is solved by a call to \texttt{SNESSolve()}.  Finally the new values in \texttt{x}, presumably the converged solution, are printed at the command line using \texttt{VecView()} with a \texttt{STDOUT} viewer.

\vfill
\cinput{expcircle.c}{\CODELOC}{A first \pSNES-using code.  Solves nonlinear system \eqref{eq:nl:equation} with $\bF$ given in \eqref{eq:nl:expcircleF}.}{//START}{//END}{code:expcircle}

In order to match the calling sequence of \texttt{SNESSetFunction()}, \texttt{FormFunction()} must have a particular ``signature'' as a C function:
\begin{code}
PetscErrorCode FormFunction(SNES snes, Vec x, Vec F, void *ctx)
\end{code}
In particular, \texttt{FormFunction()} takes the input $\bx$ as the first \pVec argument and it generates output $\bF(\bx)$ as the second \pVec.\sidenote{For the second \pVec argument, note that a \pVec is in reality a \emph{pointer}, so passing a \pVec by value allows it to be modified.}

In other examples there may be additional information, such as parameters, passed to \texttt{FormFunction()} in a ``user context.''  This is why there is a fourth pointer argument ``\texttt{void*}'' to \texttt{FormFunction()}.  We will show how to pass parameters in the next code.

Looking inside \texttt{FormFunction()} in Code \ref{code:expcircle}, we see a new function for extracting values from, and setting values in, a \pVec.  Previously we have used \texttt{VecSetValues()} to set values at given indices (Chapter \ref{chap:ls}), or we have used a \pDMDA structured grid method of access (Chapter \ref{chap:st}).  Here we access the C array underlying the \pVec.  Because we only need to read entries of input \pVec \texttt{x}, we use the read-only array access method \texttt{VecGetArrayRead()} which supplies us with a read-only pointer ``\texttt{const double *ax}.''\sidenote{Because of the \texttt{const} qualifier, the C compiler can stop us from altering \texttt{ax[0]}, for example.  Try it!}  Since we are setting entries of \pVec \texttt{F}, we do nearly the same for it but we use an unrestricted pointer ``\texttt{double *aF}'' and function \texttt{VecGetArray()}.

Note that calls to \texttt{VecGetArray()} and \texttt{VecGetArrayRead()} are matched by \texttt{VecRestoreArray()} and \texttt{VecRestoreArrayRead()}, respectively.  These methods ``free-up''  the \pVecs so that other parts of the code can work on them, but they do not deallocate the \pVecs.\sidenote{\texttt{VecDestroy()} does that.}  In general:
\begin{quote}
\emph{Each \emph{\texttt{VecGetArray()}}-type call should be matched by the corresponding \emph{\texttt{VecRestoreArray()}} call once you are done with the array.}
\end{quote}

The actual content of \texttt{FormFunction()} is to implement formulas \eqref{eq:nl:expcircleF}.  \texttt{PetscExpReal()}, which computes the exponential function $e^x$, is just an alias for \texttt{exp()} from the standard library (i.e.~\texttt{math.h}).  Use of such \PETSc library functions means that the \PETSc configuration can link to consistent libraries just by including \texttt{petsc.h}.

It is time to run this example with \texttt{-snes\_monitor}, so as to count the iterations and show the residual norm $\|\bF(\bx_k)\|_2$:
\begin{cline}
$ cd c/ch4/
$ make expcircle
...
$ ./expcircle -snes_fd -snes_monitor
  0 SNES Function norm 2.874105323289e+00 
  1 SNES Function norm 8.591393113962e-01 
  2 SNES Function norm 1.609958353862e-01 
  3 SNES Function norm 1.106891696425e-02 
  4 SNES Function norm 6.618141730691e-05 
  5 SNES Function norm 2.420782802130e-09 
Vec Object: 1 MPI processes
  type: seq
0.319632
0.947542
\end{cline}
%$
Thus after 5 iterations the Newton method has reduced the residual norm by a factor of $10^9$ and stopped with solution $x_0=0.319632$ and $x_1=0.947542$.  Compare Figure \ref{fig:expcirclebasic}.

The above run uses option \texttt{-snes\_fd}, the purpose of which the reader may already see.  Clearly the Newton iteration \eqref{eq:nl:newton} requires the Jacobian, but we have only supplied the \pSNES with an implementation of function $\bF(\bx)$, not with $J_{\bF}(\bx)$.  As mentioned, the entries of the latter matrix are derivatives which can be approximated by finite differences.  Specifically, let $\delta\ne 0$ and let $\be_j\in \RR^N$ denote the standard unit vector with entry one in the $j$th position and zeros otherwise.  An entry in matrix $J=J_{\bF}(\bx)$ is approximated
\begin{equation}
J_{ij} = \frac{\partial F_i}{\partial x_j} \approx \frac{F_i(\bx+\delta \be_j) - F_i(\bx)}{\delta}.  \label{eq:nl:fdjac}
\end{equation}
When using \texttt{-snes\_fd}, \PETSc chooses $\delta$ internally and applies \eqref{eq:nl:fdjac}.  For example, $\delta = \sqrt{\eps}$, where $\eps$ is machine precision, gives a reasonably-accurate approximation if the inputs to $\bF$ are all of order approximately one and the function $\bF$ can be accurately-evaluated \citep{Kelley2003}.


\section{Inside \pSNES} \label{sec:insidesnes}

It is helpful to describe Newton iteration from the point of view of the actions taken by the \pSNES solver object.  In outline, it does these steps:
\begin{quote}
	\renewcommand{\labelenumi}{(\emph{\roman{enumi}})}
	\renewcommand{\labelenumii}{\emph{\alph{enumii}}.}
	\begin{enumerate}
	\item from the current iterate $\bx_k$, $\bF(\bx_k)$ is evaluated using a call-back function as set in \texttt{SNESSetFunction()}, e.g.~\texttt{FormFunction()} above,
	\item the Jacobian $J=J_{\bF}(\bx_k)$ is
	    \begin{enumerate}
	    \item computed and assembled by a call-back to user-supplied code, if it is available, as set using \texttt{SNESSetJacobian()}, e.g.~\texttt{FormJacobian()} in code \texttt{ecjacobian.c} below, \emph{or}
	    \item computed and assembled by evaluating $\bF(\bx_k+\delta \be_j)$ for $j=0,\dots,N-1$, thus  calling \texttt{FormFunction()} $N$ times, and then using formula \eqref{eq:nl:fdjac} $N^2$ times to compute all entries of $J$, \emph{or}
	    \item computed and assembled by calling \texttt{FormFunction()} substantially fewer than $N$ times to compute $\bF(\bx_k+\delta \bv)$ for special vectors $\bv$, by using a graph-coloring algorithm based on the sparsity pattern of $J$ to construct the vectors $\bv$, and using formula \eqref{eq:nl:fdcolorjac} below, \emph{or}
	    \item not assembled, but, in a Krylov iterative method for solving system \eqref{eq:nl:newton:eq}, the action $J \by$, of the Jacobian on vectors $\by$, is computed by finite-differences,
        \end{enumerate}
	\item linear system \eqref{eq:nl:newton:eq} is solved for $\bs$ by some \pKSP object, using whatever additional preconditioning matrix has been chosen,
	\item vector update \eqref{eq:nl:newton:update} is done, with possible reduction or expansion in the length of $\bs$ according to the line-search object, as addressed starting on page \pageref{sec:linesearch}, \emph{and}
	\item a convergence test is made, and we repeat at (\emph{i}) if not converged.
	\end{enumerate}
\end{quote}

Our single run above of \texttt{expcircle.c} used option (\emph{ii}) \emph{b} for the Jacobian, but option (\emph{ii}) \emph{d} also works.  Our next code will allow option (\emph{ii}) \emph{a} as well.  The graph-coloring technique (\emph{ii}) \emph{c} will be addressed starting on page \pageref{sec:nl:coloring}.

If you run \texttt{expcircle.c} without option \texttt{-snes\_fd} or \texttt{-snes\_mf} then you get an error message about an un-assembled matrix:
\begin{cline}
$ ./expcircle
[0]PETSC ERROR: --------------------- Error Message -------------------------
[0]PETSC ERROR: Object is in wrong state
[0]PETSC ERROR: Matrix must be assembled by calls to MatAssemblyBegin/End();
...
\end{cline}
%$
This message is somewhat opaque unless you are conscious of the need to form the Jacobian matrix at each Newton iteration.  That is, \emph{something} must supply a Jacobian at step (\emph{ii}), and the supplied-Jacobian-code case (\emph{ii})a does not work for \texttt{expcircle.c}.

The benefit of using options (\emph{ii})b--(\emph{ii})d above is that we do not need to write any error-prone code based on taking derivatives of our function $\bF$.  Avoiding writing and debugging Jacobian implementations may speed implementation by reducing the time \emph{you} spend on the task.  One possible disadvantage is likely to be apparent to the reader, namely that formula \eqref{eq:nl:fdjac} is only an \emph{approximation} of a Jacobian entry.  In most cases using a finite-difference-approximated Jacobian in the Newton step is not problematical \citep{Kelley2003}.

On the other hand, there is a significant performance problem in using \eqref{eq:nl:fdjac} naively for PDE-type applications of Newton's method.  In steps (\emph{i}) and (\emph{ii})b together we do $N+1$ calls to \texttt{FormFunction()} per Newton iteration.  This is a worrying amount of work if $N$ is large, as it would be for a system of nonlinear equations coming from discretizing a PDE.  We will therefore return to this issue later in the current Chapter, and provide more detail on (\emph{ii})c and (\emph{ii})d.

Evaluating $\bF$ can dominate the work in the Newton iteration, especially in systems where it is an intrinsically-expensive function to evaluate; \PETSc option \texttt{-log\_view} can help you see how many times $\bF$ is evaluated.  The work done in solving linear system \eqref{eq:nl:newton:eq} is the other main concern, which suggests that a good habit, to start right now, is to use \texttt{-log\_view} to see which kind of work actually dominates.  A furhter good habit is to use a \verb|--with-debugging=1| \PETSc configuration while developing your code, but also to maintain a \verb|--with-debugging=0| ``optimized'' configuration for use in evaluating performance, e.g.~when looking at the timing results from \texttt{-log\_view}.


\section{Residual norm in the Newton iteration}

Now we return to running \texttt{expcircle.c}.  Option \texttt{-snes\_rtol} specifies by what factor the \pSNES should try to reduce the residual norm.  The default accuracy corresponds to \texttt{-snes\_rtol 1.0e-8}; this default value can be printed by running
\begin{cline}
$ ./expcircle -snes_fd -help | grep snes_rtol
\end{cline}
%$
Thus the example
\begin{cline}
$ ./expcircle -snes_fd -snes_monitor -snes_rtol 1.0e-14
\end{cline}
%$
asks for much more accuracy.  It may be a surprise that asking for a further $10^6$ reduction in residual norm requires only one more iteration, namely six iterations this time, but this is typical of the Newton iteration in the best cases.  Instead of showing the Newton iterations again as text output, Figure \ref{fig:newtonconvbasic} shows the residual norms in a graph with log-scaling on the $y$-axis.

\begin{figure}
\includegraphics[width=0.8\textwidth]{figs/newtonconvbasic}
\caption{The characteristic look of the quadratic convergence of the Newton iteration: the residual norm drops abruptly.}
\label{fig:newtonconvbasic}
\end{figure}

The residual norm drops abruptly in the Figure, reflecting the hoped-for best-case behavior of Newton iteration.  In fact, the per-iteration decrease is substantial in the sense that the \emph{numerical error} is proportional to the \emph{square} of the numerical error at the previous iteration, and this is the main reason the Newton iteration is so powerful.

The next theorem will express such best-case behavior, but before we state it let us pause to define ``numerical error''.  If $\bx^*$ denotes a solution of \eqref{eq:nl:equation} then the \emph{numerical error} at iteration $k$ is
\begin{equation}
\be_k = \bx_k-\bx^*.  \label{eq:nl:errordefn}
\end{equation}
This is the quantity which we want to send to zero.  In particular, we want to stop the Newton iteration when $\bx_k$ is a good approximation of $\bx^*$.

Of course, $\be_k$ is just as hard to compute as $\bx^*$; we do not generally have exact access to either one.  Instead we have, available for inspection when needed, the iterate itself $\bx_k$ and the computable residual $\bF(\bx_k)$.  Both parts of the following theorem are important.

\begin{theorem} \citep[Theorems 1.1 and inequalities (1.13)]{Kelley2003}
Suppose that $\bF:\RR^N\to\RR^N$ is differentiable, $J_{\bF}$ is Lipschitz near $\bx^*$, and $J_{\bF}(\bx^*)$ is a nonsingular matrix.  Let $\|\cdot\|$ denote a vector norm and its induced matrix norm, and let $\kappa(A)=\|A^{-1}\| \|A\|$ denote the condition number of an invertible matrix $A$.  If $\bx_0$ is sufficiently close to $\bx^*$ then, in exact arithmetic,
\renewcommand{\labelenumi}{(\roman{enumi})}
\begin{enumerate}
\item there is $K\ge 0$ such that for all $k$ sufficiently large,
\begin{equation}
	\|\be_{k+1}\| \le K \|\be_k\|^2, \label{eq:nl:quadraticconvergence}
\end{equation}
\item and if $\kappa = \kappa\left(J_{\bF}(\bx^*)\right)$ then
\begin{equation}
	\frac{\|\be_k\|}{4 \kappa \|\be_0\|} \le \frac{\|\bF(\bx_k)\|}{\|\bF(\bx_0)\|} \le \frac{4 \kappa \|\be_k\|}{\|\be_0\|}. \label{eq:nl:errorresidualnormequiv}
\end{equation}
\end{enumerate}
\end{theorem}

\medskip
By definition, a sequence $\{\bx_k\}$ in $\RR^N$ \emph{converges quadratically to} $\bx^*$ if the sequence of errors $\{\be_k\}$ satisfies \eqref{eq:nl:quadraticconvergence} for some $K\ge 0$.  Thus the Theorem says that, under strong assumptions about the regularity and nonsingularity of the Jacobian, the iterates converge quadratically to a solution of \eqref{eq:nl:equation}.  Heuristically, once $\|\be_k\|$ gets reasonably-small then the number of correct digits in $\bx_k$ \emph{doubles} with each additional iteration.

We seem to see quadratic convergence in Figure \ref{fig:newtonconvbasic}, but actually it shows the residual norm $\|\bF(\bx_k)\|_2$ and not the error norm $\|\be_k\|_2$.  However, the second part of the Theorem says that the residual decrease at the $k$th iteration (i.e.~$\|\bF(\bx_k)\|/\|\bF(\bx_0)\|$) is within a factor, determined by the conditioning of the Jacobian at the solution, of the error decrease ($\|\be_k\|/\|\be_0\|$).

The Theorem confirms that residual norm decay like that shown in Figure \ref{fig:newtonconvbasic} corresponds to quadratic convergence of $\bx_k$ to a solution $\bx^*$.  If we want to reduce the (generally-unknowable) error $\be_k$ by a given amount then it can suffice to reduce the residual norm by a comparable amount.  The factor $4 \kappa$ by which the two relative norms differ in \eqref{eq:nl:errorresidualnormequiv} is large if the conditioning of the Jacobian at the solution is poor, but, just as in the linear case, a large Jacobian condition number would also mean lost precision in solving \eqref{eq:nl:equation} by \emph{any} numerical means.\sidenote{Recall the numerical facts-of-life in Chapter \ref{chap:ls}.}

In \PETSc, the amount of residual norm reduction is exactly what the option \texttt{-snes\_rtol} controls, that is, the iteration continues until
    $$\frac{\|\bF(\bx_k)\|_2}{\|\bF(\bx_0)\|_2} \le \text{\texttt{snes\_rtol}}.$$

To give the more complete story, however, note there are \emph{three} \pSNES tolerances, as listed in Table \ref{tab:snestolerances}.  The iteration stops as soon as one of these conditions is satisfied.  Option \texttt{-snes\_converged\_reason} reports which was active, using the name given in Table \ref{tab:snestolerances}.  For example,
\begin{cline}
$ ./expcircle -snes_fd -snes_converged_reason
Nonlinear solve converged due to CONVERGED_FNORM_RELATIVE iterations 5
...
\end{cline}
%$
Also note that $\bs_k$, the ``step'' at iteration $k$, denotes the solution to linear system \eqref{eq:nl:newton:eq}.

\medskip
\begin{table}
\begin{tabular}{lll}
\underline{Option}\hspace{0.2in} & \underline{Name}\hspace{0.2in} & \underline{Condition}\hspace{0.2in} \\
\texttt{-snes\_rtol X} & relative (\texttt{FNORM\_RELATIVE}) & $\|\bF(\bx_k)\|_2 \le \text{\texttt{X}}\, \|\bF(\bx_0)\|_2$ \\
\texttt{-snes\_atol X} & absolute (\texttt{FNORM\_ABS}) & $\|\bF(\bx_k)\|_2 \le \text{\texttt{X}}$ \\
\texttt{-snes\_stol X} & step-length (\texttt{SNORM\_RELATIVE}) & $\|\bs_k\|_2 \le \text{\texttt{X}}\, \|\bx_k\|_2$
\end{tabular}
\caption{The three ways \pSNES can succeed, thereby stopping the Newton iteration.} \label{tab:snestolerances}
\end{table}

\medskip
The defaults for the three tolerances in the Table are \texttt{X} $=10^{-8},10^{-50},10^{-8}$ respectively.  One can force the \pSNES to use a subset of the stopping criteria by setting the tolerance to zero (\texttt{X} $=0$) in the unwanted condition(s).


\section{Convergence difficulties} \label{sec:divergence}

So far we have portrayed the Newton iteration in optimistic terms, but it is not magic and things can go wrong.  First note a key hypothesis in the above Theorem, namely that $\bx_0$ needs to be sufficiently close to $\bx^*$.  Even on well-behaved nonlinear equations, if $\bx_0$ is far from the solution then the iteration may take many steps before $\|\be_k\|$ becomes small enough so that quadratic convergence \eqref{eq:nl:quadraticconvergence} ``kicks in''.  For example, Figure \ref{fig:newtonconvdelayed} shows what happens if we use initial iterate $\bx_0=[10\,\, 10]^\top$ in the above Example.  About 16 iterations of slow progress\sidenote{From the constant slope shown in the Figure, this is evidently \emph{linear} convergence in which the residual norm is reduced by a constant factor.} is needed before the iterate $\bx_k$ enters the region around $\bx^*$ where the conclusions of the Theorem apply.  This region is sometimes known as the ``ball of quadratic convergence.''

\begin{figure}
\includegraphics[width=0.8\textwidth]{figs/newtonconvdelayed}
\caption{Even in for well-behaved systems $\bF(\bx)=0$, if the initial iterate $\bx_0$ is far from the solution then quadratic convergence can be postponed for many iterations.}
\label{fig:newtonconvdelayed}
\end{figure}

Actual decrease in residual norm, as displayed in Figures \ref{fig:newtonconvbasic} and \ref{fig:newtonconvdelayed}, is also not guaranteed in general.  Indeed there is nothing intrinsic about \eqref{eq:nl:newton} that implies $\|\bF(\bx_{k+1})\| \le \|\bF(\bx_k)\|$, and on some equations the Newton iteration as it stands actually diverges when starting from some initial states; for an example, see Exercise \ref{chap:nl}.\ref{exer:nl:newtonatan}.  However, the ``line-search'' methods, described starting on page \pageref{sec:linesearch}, enforce residual norm decrease, or stop if it cannot be achieved.  The line-search method is said to ``globalize'' the convergence \citep{Kelley2003} in the sense of allowing convergence from a larger set of initial states.

Many real-world problems do not have the smoothness needed to apply the above Theorem.  In such cases regularization, continuation, or other procedures may be needed to make the Newton method effective.


\section{Exact Jacobians and passing parameters}

We have yet to exploit two critical possibilities when using a \pSNES, namely passing parameters through the call-back mechanism, so that they can be used inside the residual- and Jacobian-evaluation functions, and providing a user-written Jacobian function $J_{\bF}(\bx)$.  Codes \ref{code:ecjacobianI} and \ref{code:ecjacobianII} show  \texttt{ecjacobian.c}, which implements both of these capabilities.  It is a ``model use'' of \pSNES.\sidenote{For cases without a structured grid of the \pDMDA type (Chapter \ref{chap:st}), anyway.  Compare \texttt{reaction.c} below.}

Code \ref{code:ecjacobianI} starts with a declaration of a C \texttt{struct} called \texttt{AppCtx} (``application context'').  It has just one element, the real parameter $b$ which appears in formulas \eqref{eq:nl:expcircleF} and \eqref{eq:nl:ecjacobian}, so a \texttt{struct} is not necessary, but in future examples there will be more than one parameter to pass.  Next, the method \texttt{FormFunction()} is almost the same as the one in \texttt{expcircle.c} (Code \ref{code:expcircle}), but the value of $b$ comes from the \texttt{struct} instead of being hard-wired as before.  In detail, the argument \texttt{void *ctx} is ``cast'' in the sense of the C language \citep{KernighanRitchie1988} to a pointer of type \texttt{AppCtx*}, and then the parameter is extracted, via the pointer, by ``\texttt{user->b}.''\sidenote{Note ``\texttt{user->b}'' and ``\texttt{(*user).b}'' are equivalent expressions in C.}  This awkward-seeming method of passing parameters allows the signature of \texttt{FormFunction()} to be precisely as before.

\cinputpart{ecjacobian.c}{\CODELOC}{Solves the same nonlinear system as \texttt{expcircle.c}, but with an exact Jacobian and parameter passing.}{I}{//START}{//END}{code:ecjacobianI}

The function \texttt{FormJacobian()} in Code \ref{code:ecjacobianI} is new.  It has similar structure and semantics to \texttt{FormFunction()}, but a new signature for this kind of call-back, namely
\begin{code}
PetscErrorCode FormJacobian(SNES snes, Vec x, Mat P, Mat J, void *ctx)
\end{code}
Input \pVec \texttt{x} and pointer \texttt{void *ctx} have the same meaning as in \texttt{FormFunction()}.

The difference is that we must set a \pMat as output, based on formula \eqref{eq:nl:ecjacobian} in this case.  In setting up the output \pMat the roles of \texttt{MatSetValues()}, real array \texttt{v[4]} for the entries themselves, and integer arrays \texttt{row[2]} and \texttt{col[2]} as global indices, are all the same as in Chapter \ref{chap:ls}.  Also as before, when reading \texttt{x} we can use \texttt{VecGetArrayRead()} and \texttt{VecRestoreArrayRead()}.

\cinputpart{ecjacobian.c}{\CODELOC}{This \texttt{main()} function allocates a \pMat to hold the Jacobian.}{II}{//STARTMAIN}{//ENDMAIN}{code:ecjacobianII}

An interesting detail appears here, however.  There are actually \emph{two} output \pMats for \texttt{FormJacobian()} to set.  The first, called \texttt{J} here, corresponds to the Jacobian matrix itself, which we want to supply in the no-option case.  The second, called \texttt{P} here, is the ``material'' we supply from which \PETSc can build a preconditioner.  Here function \texttt{FormJacobian()} sets the entries of \pMat \texttt{P} to the correct, accurate derivatives of $\bF$, but \texttt{P}  might, in other cases, be a poor approximation of the Jacobian.  When we are done setting entries, we then assemble the \pMat \texttt{P}.  In the no option case we have also set the entries of \pMat \texttt{J}, because it is the same \pMat object as \texttt{P}.  However, and this applies if options \texttt{-snes\_fd} or \texttt{-snes\_mf} or \texttt{-snes\_mf\_operator} are used, if the \pMat \texttt{J} points to a different object then we also assemble it.\sidenote{This odd action is because \texttt{J} might represent a \emph{different} (unassembled) approximation of the Jacobian, e.g.~it is a finite-difference action in the \texttt{-snes\_mf\_operator} usage described later.  In the current case we do not want to modify \texttt{J}, but we do need to tell \PETSc that we are done with it.  Thus these apparently-odd actions on \texttt{J} and \texttt{P} allow \emph{preconditioned} Jacobian-free Newton-Krylov methods to work; see page \pageref{sec:JFNK} below.}

Looking at \texttt{main()} in Code \ref{code:ecjacobianII}, note that we create and configure a $2\times 2$ \pMat \texttt{J} to hold the Jacobian.  Our use of \texttt{MatCreate(), MatSetSizes(), MatSetFromOptions(),} and \texttt{MatSetUp()} on \texttt{J} mimics what we did for linear systems in Chapter \ref{chap:ls}.  However, this time when we set-up the \pSNES we pass \texttt{J} for two arguments,
\begin{code}
SNESSetJacobian(snes,J,J,FormJacobian,&user);
\end{code}
This means we provide the allocated space in \texttt{J} as both the Jacobian and preconditioner matrices, that is, for both the second and third \pMat arguments.  We are telling \PETSc that we have an exact Jacobian and that we want the preconditioner to be built from it too.

Now that we have assembled an exact Jacobian, we see that it and its finite-difference approximation produce nearly the same result on this small and well-behaved example:
\begin{cline}
$ make ecjacobian
...
$ ./ecjacobian -snes_monitor
  0 SNES Function norm 2.874105323289e+00 
  1 SNES Function norm 8.591392822370e-01 
  2 SNES Function norm 1.609958166309e-01 
  3 SNES Function norm 1.106891138388e-02 
  4 SNES Function norm 6.618107497046e-05 
  5 SNES Function norm 2.419259135755e-09 
...
$ ./ecjacobian -snes_monitor -snes_fd
  0 SNES Function norm 2.874105323289e+00 
  1 SNES Function norm 8.591393113962e-01 
  2 SNES Function norm 1.609958353862e-01 
  3 SNES Function norm 1.106891696425e-02 
  4 SNES Function norm 6.618141730691e-05 
  5 SNES Function norm 2.420782802130e-09 
...
\end{cline}
%$

\PETSc also helps with debugging exact-Jacobian code through option \texttt{-snes\_type test}.  Specifically, for a small selection of inputs $\by$, the finite-difference approximation of the Jacobian $J_{\bF}(\by)$, which needs only \texttt{FormFunction()} evaluations, is compared to the result of \texttt{FormJacobian()} evaluations.  Oddly, this option generates an error message, but at least it causes the run to stop after the test results are shown.  The user must read the output and decide if the comparison shows that the implemented Jacobian is good:
\begin{cline}
$ ./ecjacobian -snes_type test
Testing hand-coded Jacobian, if the ratio is
O(1.e-8), the hand-coded Jacobian is probably correct.
Run with -snes_test_display to show difference
of hand-coded and finite difference Jacobian.
Norm of matrix ratio 1.9182e-08, difference 1.52973e-07 (user-defined state)
Norm of matrix ratio 1.21378e-08, difference 3.64505e-08 (constant state -1.0)
Norm of matrix ratio 1.9182e-08, difference 1.52973e-07 (constant state 1.0)
[0]PETSC ERROR: --------------------- Error Message ------------------------
[0]PETSC ERROR: Object is in wrong state
[0]PETSC ERROR: SNESTest aborts after Jacobian test: it is NORMAL behavior.
...
[0]PETSC ERROR: ----------------End of Error Message -------
...
\end{cline}
%$
See Exercise \ref{chap:nl}.\ref{exer:nl:snestestdisplay} for a bit more on this capability.


\section{Example: a nonlinear diffusion-reaction equation}

A two-point boundary value problem gets us started on nonlinear PDE problems.  Suppose $u(x)$ is the density of some substance.  An equation of the form
\begin{equation}
- u'' - R(u) = f(x)  \label{eq:nl:diffusionreaction}
\end{equation}
models a time-independent combination of \emph{diffusion} ($u''$) and \emph{reaction} ($R(u)$) processes.  The right-hand side $f(x)$ models an additional location-dependent \emph{source}.

This problem is clearest if we note \eqref{eq:nl:diffusionreaction} is the steady-state of the time-dependent model
\begin{equation}
u_t = u_{xx} + R(u) + f(x),  \label{eq:nl:drtimedependent}
\end{equation}
which generalizes the one-dimensional, time-evolving heat equation $u_t = u_{xx}$.  A positive sign for $R(u)+f(x)$ clearly models the increase of $u$, and a negative sign the decrease.  If $u$ represents temperature, for instance, in which case $R(u)$ represents a heat-producing or heat-absorbing temperature-dependent reaction, according to its sign, then \eqref{eq:nl:diffusionreaction} models the equilibrium temperature distribution.

We should be concerned about the solvability of \eqref{eq:nl:diffusionreaction} in the case where $R$ is an \emph{increasing} function of $u$.  If $R$ is positive and increasing then the ability of the diffusion term to ``damp-out'' maxima may be exceeded by the increasing production from large values of $u$, so that these terms in \eqref{eq:nl:diffusionreaction} cannot balance.  If $R$ is negative and increasing then the analogous concern applies to minima of $u$.  These concerns are demonstrated by the example $R(u) = \lambda e^u$ with $\lambda>0$ in Exercise 4.\ref{exer:nl:bratu}; that problem is not solvable for sufficiently-large $\lambda$.

However, if $R$ is a non-increasing function then our worries go away.  In fact, equation \eqref{eq:nl:diffusionreaction} is a nonlinear elliptic PDE,\sidenote{By mild abuse of the letter ``P''.} but in one-dimension.  Elliptic PDE techniques show the problem is well-posed in that case \citep[pages 93-94]{KinderlehrerStampacchia1980}.  To be concrete, consider Dirichlet boundary conditions
\begin{equation}
u(0)=\alpha \quad \text{and} \quad u(1)=\beta  \label{eq:nl:drbcs}
\end{equation}
on the convenient interval $x\in[0,1]$.  (For other intervals simply shift and scale $x$ as needed.)  If $R$ is continuous and non-increasing then the nonlinear operator in \eqref{eq:nl:diffusionreaction} is strictly-monotone \citep[page 83]{KinderlehrerStampacchia1980}.  Because \eqref{eq:nl:diffusionreaction} is also coercive on the appropriate function space,\sidenote{Namely the Sobolev space $H_0^1[0,1]$, after a linear change of variables to set $\alpha=\beta=0$.} which says intuitively that the highest-order (diffusion) term is effective at damping out large variations in $u$ which correspond to a large norm $\|u'\|_2$, abstract arguments show unique existence of a solution.

So now consider the example
\begin{equation}
-u'' + \rho \sqrt{u} = 0. \label{eq:nl:drsqrt}
\end{equation}
% R(u) = - rho sqrt(|u|) decreasing so -u''+F(u)=0 with F increasing
This is of form \eqref{eq:nl:diffusionreaction} with $R(u) = - \rho \sqrt{u}$ and $f(x)=0$.  Because $R$ is non-increasing and continuous if $\rho>0$, the corresponding Dirichlet problem is well-posed.

We will use \eqref{eq:nl:drsqrt} as a first example because we want to verify our numerical solution using an exact solution.   Exercise 5.50 of \citet[page 240]{Ockendonetal2003} shows how to find the solution.  One observes that both second-derivative and square-root operations convert certain 4th degree polynomials into quadratic polynomials.  Substituting $u(x)=M(x+1)^4$ into \eqref{eq:nl:drsqrt} finds $M=(\rho/12)^2$.  We also obtain the boundary conditions from the exact solution: set $\alpha=M$ and $\beta=16 M$.


\section{Numerical method and implementation} \label{sec:nl:implementation}

On an $N$ point grid, the centered, $O(h^2)$ local truncation error \citep{MortonMayers2005} finite-difference scheme for \eqref{eq:nl:diffusionreaction} is
\begin{equation}
- \frac{u_{j+1} - 2 u_j + u_{j-1}}{h^2} - R(u_j) = f(x_j)   \label{eq:nl:drfdscheme}
\end{equation}
where $h=1/(N-1)>0$, $x_j = j h$ for $j=0,1,\dots,N-1$, and $u_j \approx u(x_j)$.  Our program \texttt{reaction.c} shown in Codes \ref{code:reactionI} and \ref{code:reactionII} implements this scheme.  We use a \pSNES object for the Newton iteration and a \pDMDA object for the grid.

Three functions appear in Code \ref{code:reactionI}.  The first one (\texttt{InitialAndExact()}) computes both the initial iterate\sidenote{A linear function connecting the boundary conditions.} and the exact solution.  Then come residual (\texttt{FormFunctionLocal()}) and Jacobian (\texttt{FormJacobianLocal()}) evaluation functions.  These two are ``call-back'' functions which the \pSNES will use when it needs the values of the residual and Jacobian at a given location in the space of candidate solutions.

The functions in Code \ref{code:reactionI} have common features to their signatures.  They each have arguments of type \texttt{DMDALocalInfo*} and \texttt{double*}.  This reflects two ideas which are worth noting:
\renewcommand{\labelenumi}{(\emph{\roman{enumi}})}
\begin{enumerate}
\item First, the inputs are C array pointers instead of \pVecs.  Our implementation of \texttt{InitialAndExact()} explains how this works.  Before calling it, in Code \ref{code:reactionII} below, we use \texttt{DMDAVecGetArray()} on \pVecs \texttt{u} and \texttt{uexact}.  This gives pointers of type \texttt{double*} which are handed to \texttt{InitialAndExact()}.  Also, a \texttt{DMDALocalInfo} structure\sidenote{Figure \ref{fig:localpartofgrid} shows the information carried by \texttt{DMDALocalInfo}.} is passed in so that the local part of the grid and the global grid size (i.e.~\texttt{info.mx}) can be accessed through this \texttt{struct} without needing the \pDMDA object itself.  The \pSNES calls \texttt{FormFunctionLocal()} and \texttt{FormJacobianLocal()} in the same way.
\item Second, the pointers of type \texttt{double*} passed into \texttt{FormFunctionLocal()} and \texttt{FormJacobianLocal()} have valid ghosts.  That is, expressions ``\texttt{u[i+1]}'' and ``\texttt{u[i+1]}'' are valid ways of dereferencing \texttt{double *u}.  This feature is used in \texttt{FormFunctionLocal()} because indices $i-1$ and $i+1$ appear in finite-difference equation \eqref{eq:nl:drfdscheme}.  Though we never see the corresponding \pVec, ``under the hood'' the \pSNES has allocated a local \pVec with valid ghost locations, then communicated between processes to update those ghost values, and then used \texttt{DMDAVecGetArray()}/\texttt{DMDAVecRestoreArray()} to get an array.  By contrast, the solution vector \texttt{u} we provide to the \pSNES, in Code \ref{code:reactionII} at the \texttt{SNESSolve()} call, has no ghosts because it was allocated with \texttt{DMCreateGlobalVector()}.
\end{enumerate}

In \texttt{main()} in Code \ref{code:reactionII} we use \texttt{DMDASNESSetFunctionLocal()} and  \texttt{DMDASNESSetJacobianLocal()}.  These are structured-grid-aware versions of \texttt{SNESSetFunction()} and \texttt{SNESSetJacobian()}, respectively, used earlier in \texttt{ecjacobian.c} (Code \ref{code:ecjacobianII}).  Note that, while we do set entries in a Jacobian, in \texttt{reaction.c} we never explicitly allocate a \pMat to hold it.  The \pDMDA object, however, has enough information about the grid and stencil to allocate it internally.

\cinputpart{reaction.c}{\CODELOC}{Functions describing the initial iterate, the exact solution, finite-difference equation \eqref{eq:nl:drfdscheme}, and the derivatives of \eqref{eq:nl:drfdscheme}.}{I}{//CALLBACK}{//ENDCALLBACK}{code:reactionI}

\cinputpart{reaction.c}{\CODELOC}{In \texttt{main()} we create a \pDMDA, some \pVecs, and a \pSNES.  We hand call-back functions to the \pSNES.  Then we solve the equation.}{II}{//MAIN}{//ENDMAIN}{code:reactionII}

Though the previous paragraphs adequately summarize program \texttt{reaction.c}, the diagram in Figure \ref{fig:nl:reactionstack} communicates the overall structure.  Our program uses six different major \PETSc types, but we apply explicit \texttt{Create()}/configure/act/\texttt{Destroy()} sequences to only three of these (\pDMDA, \pSNES, \pVec).  We do set entries in a \pMat, but it is allocated internally by the \pDMDA object.  However, one must remember that each step of Newton's method solves a linear system, and thus that \pKSP and \pPC objects exist ``inside'' the \pSNES, as revealed by \texttt{-snes\_view} in particular.  When seeking better performance on fine grids, exposure and control of these linear solver objects is essential.

\begin{figure}
\begin{tikzpicture}[scale=0.8,
                    >={Latex[length=2mm]},
  component/.style={
     rectangle,draw,fill=white,align=center,line width=1pt},
  userfcn/.style={
     rounded corners,draw,fill=white,draw,align=center,line width=1pt,font={\itshape,\normalsize}}]

\draw[line width=1pt] (3,7) node[userfcn,minimum width=90mm] (usercode) {user code \\ \vspace{15mm}};

\draw[line width=1pt] (1,7.2) node[userfcn] (rescode) {residual \emph{\texttt{FormFunctionLocal()}}};
\draw[line width=1pt] (5,6.2) node[userfcn] (jaccode) {Jacobian \emph{\texttt{FormJacobianLocal()}}};

\draw[line width=1pt] (-1,4) node[component] (snes) {\complabel{\pSNES}{nonlinear solver}};
\draw[line width=1pt] (-1,2) node[component] (ksp)  {\complabel{\pKSP}{linear solver}};
\draw[line width=1pt] (-1,0) node[component] (pc)   {\complabel{\pPC}{preconditioner}};

\draw[line width=1pt] (7,4) node[component] (dmda) {\complabel{\pDMDA}{structured grid}};

\draw[line width=1pt] (2,0) node[component] (matj)   {\usedlabel{\pMat}{Jacobian}};
\draw[line width=1pt] (6,0) node[component] (vecs)   {\pVecs \\ \footnotesize  \emph{solution, other fields}};

\path
   ([xshift=-10em]usercode.south) edge[->] node {} (snes)
   ([xshift=10em]usercode.south) edge[->] node {} (dmda)
   ([xshift=0em]usercode.south) edge[->] node {} (vecs)

   (rescode.south) edge[->] node {} ([xshift=-2em]vecs.north)

   ([xshift=-1em]jaccode.south) edge[->] node {} (matj)
   ([xshift=1em]jaccode.south) edge[->,bend right] node {} ([xshift=1em]vecs.north)

   (snes) edge node {} (ksp)
   (ksp) edge node {} (pc)

   ([yshift=1em]snes.east) edge[->,dotted,bend right] node {} ([xshift=-2em]rescode.south)
   ([yshift=-1em]snes.east) edge[->,dotted,bend right] node {} ([xshift=-2em]jaccode.south);
\end{tikzpicture}
\caption{An overview of \texttt{reaction.c}, our first structured-grid solution of a nonlinear PDE using \PETSc.  Solid arrows should be read as ``user code acts directly on.''  Dotted arrows are call-backs.  Compare Figure \ref{fig:st:linearstack}.}
\label{fig:nl:reactionstack}
\end{figure}


\section{Convergence under grid refinement}

On a modestly-refined grid we can compare the number of \texttt{reaction.c} Newton iterations from both analytical and finite-difference Jacobians.  Recall that the resolution of a structured grid can be set with \texttt{-da\_refine K}.  In fact the results are identical after 6 levels of refinement:
\begin{cline}
$ ./reaction -snes_converged_reason -da_refine 6
Nonlinear solve converged due to CONVERGED_FNORM_RELATIVE iterations 3
on 513 point grid:  |u-u_exact|_inf/|u|_inf = 4.62255e-08
$ ./reaction -snes_converged_reason -da_refine 6 -snes_fd
Nonlinear solve converged due to CONVERGED_FNORM_RELATIVE iterations 3
on 513 point grid:  |u-u_exact|_inf/|u|_inf = 4.62255e-08
\end{cline}
One can also see the Newton iterates graphically by
\begin{cline}
$ ./reaction -da_refine 6 -snes_monitor_solution -draw_pause 1
\end{cline}
%$
One sees that the first Newton step already moves $\bx_k$ close to the solution (not shown).

\begin{figure}
\includegraphics[width=0.8\textwidth]{figs/reaction-conv}
\caption{This convergence evidence suggests \texttt{reaction.c} is correctly-implemented.}
\label{fig:nl:reaction-conv}
\end{figure}

Noting that our finite difference method has local truncation error $O(h^2)$, and because an exact solution allows computation of the numerical error, we should generate convergence data to check the implementation.  The result of this Bash loop
\begin{cline}
$ for N in 0 2 4 6 8 10 12 14 16; do
>   ./reaction -da_refine $N -snes_rtol 1.0e-10; done
on 9 point grid:  |u-u_exact|_inf/|u|_inf = 0.000188753
on 33 point grid:  |u-u_exact|_inf/|u|_inf = 1.1825e-05
...
on 131073 point grid:  |u-u_exact|_inf/|u|_inf = 7.05476e-13
on 524289 point grid:  |u-u_exact|_inf/|u|_inf = 6.04273e-12
\end{cline}
is shown in Figure \ref{fig:nl:reaction-conv}.  If we ignore the result on the finest grid then the convergence rate is $O(h^2)$.  We also see consistent evidence of quadratic convergence by adding \texttt{-snes\_monitor} to the above runs.  Thus we conclude our implementation is correct.

Before dropping the topic, however, let us not forget the finest grid result in Figure  \ref{fig:nl:reaction-conv}.  At this point the error stopped decreasing at the expected rate.  It is time for another numerical ``fact of life'':
\renewcommand{\labelenumi}{Fact \Roman{enumi}.}
\begin{enumerate}
\setcounter{enumi}{4}
\item \emph{Error stagnation always occurs at \emph{some} level of refinement.}  For a given floating-point precision, in going down the refinement path for your PDE numerical scheme, at some point the accumulated round-off error will dominate the error from truncation.  You cannot get better accuracy than this point without changing the floating-point precision.  (This point may never be reached on 2D and 3D examples.)
\end{enumerate}



\section{Finite-difference Jacobians by ``coloring''} \label{sec:nl:coloring}

When we look closer at the \texttt{-snes\_fd} results from finite-difference Jacobians we see that it is not really working.  There are far too many function evaluations per Newton step to allow use for PDE problems at practical levels of refinement.  The underlying difficulty is that, in contrast to the earlier fixed-dimensional nonlinear examples in the current Chapter, discretizing a PDE generates an arbitrarily-large number of unknowns.  However, the exploitable property of these nonlinear systems, which arise from discretizing PDEs, is that they have a small number of unknowns per equation.

To show the performance target for an improved finite-difference Jacobian algorithm, note that an analytical Jacobian run on a grid of about 8000 nodes requires three Newton iterations and $0.04$ seconds:
\label{etc:nl:bestreaction}
\begin{cline}
$ timer ./reaction -snes_converged_reason -da_refine 10
Nonlinear solve converged due to CONVERGED_FNORM_RELATIVE iterations 3
on 8193 point grid:  |u-u_exact|_inf/|u|_inf = 1.75603e-10
real 0.04
\end{cline}
%$
This run used four $\bF$ evaluations and three $J_{\bF}$ evaluations.\sidenote{Use ``\dots \texttt{-log\_view | grep Eval}'' to confirm.}

By contrast the finite-difference-Jacobian case simply fails at first:
\begin{cline}
$ ./reaction -snes_converged_reason -da_refine 10 -snes_fd
Nonlinear solve did not converge due to DIVERGED_FUNCTION_COUNT iterations 1
on 8193 point grid:  |u-u_exact|_inf/|u|_inf = 0.0049428
\end{cline}
%$
The problem is that a single Jacobian evaluation using finite-difference formula \eqref{eq:nl:fdjac} requires $N+1=8194$ evaluations of $\bF$ per Newton iteration, one for each column of $J_{\bF}$ plus one for the right side of \eqref{eq:nl:newton:eq}.  Thus \pSNES's default maximum $\bF$ evaluation count of 10000 is exceeded with just two iterations.\sidenote{``\texttt{./reaction -help |grep snes\_}'' gets the default value for option \texttt{-snes\_max\_funcs}.}  If we raise the function-count limit then we do get convergence, but at more than 100 times increase in run time compared to the analytical Jacobian run:
\begin{cline}
$ timer ./reaction -snes_converged_reason -da_refine 10 -snes_fd \
> -snes_max_funcs 100000
Nonlinear solve converged due to CONVERGED_FNORM_RELATIVE iterations 3
on 8193 point grid:  |u-u_exact|_inf/|u|_inf = 1.75635e-10
real 9.04
\end{cline}
%$
In this run we have evaluated $\bF$ about $3N \approx 24000$ times, far too many to be sustainable when we go to two and three dimensional PDE problems.

Fortunately there is no need to abandon the finite-difference Jacobian idea.  (The idea definitely saves programmer effort at the initial prototyping stage!)  The additional tool needed to make it work is applicable to many PDE numerical schemes, especially on structured grids.  Namely, we can exploit the fact that a small set of grid values---a small stencil---is used in approximating the PDE at each location.  Finite difference, element, and volume schemes for discretizing PDES all normally have this property.\sidenote{Spectral methods \citep{Trefethen2000}, by contrast, have a large, sometimes global, stencil, and generally do not benefit.}

The idea is to ``color'' the nodes, and thus also color the unknowns of the problem and the columns of the Jacobian, with a small set of colors so that each scalar equation in system \eqref{eq:nl:equation} relates nodes (unknowns) with distinct colors.  The ``colors'' are in fact merely integers $\{0,1,2,\dots,c-1\}$; here $c$ is the number of colors used.  From the coloring, evaluations of $\bF$ can be redesigned, relative to the basic formula \eqref{eq:nl:fdjac}, to allow the computation of many columns of $J_{\bF}(\bx)$ simultaneously; this is formula \eqref{eq:nl:fdcolorjac} below.

The number of $\bF$ evaluations in this context is one more than the number of colors needed for a (proper) vertex coloring \citep[e.g.~as defined in][]{ChartrandLesniakZhang2011} of a graph associated to the Jacobian matrix.\sidenote{This was first observed by \citet{ColemanMore1983}; see also the review by \citet{Gebremedhinetal2005}.}  Algorithms exist in \PETSc for such coloring-based finite-difference Jacobian approximations.  For \pDMDA-based structured grids we can invoke them by run-time options.  Before doing such runs, however, let's be clear on the coloring in a small example.

Consider the $N=7$ point grid that comes from this command:
\begin{cline}
$ ./reaction -da_grid_x 7
\end{cline}
%$
Figure \ref{fig:stencilreaction} shows this grid and the three-point stencil for the numerical discretization scheme \eqref{eq:nl:drfdscheme}.  The stencil (equation) at $x_j$ generically involves three unknowns, namely $u_{j-1}$, $u_j$, and $u_{j+1}$.  Thus the $j$th row of the Jacobian $J$ has three nonzero entries.  Exceptions occur at the boundary nodes, where only two unknowns are involved.  In any case, option \texttt{-mat\_view ::ascii\_dense} allows one to see the Jacobian matrices themselves during the Newton iteration.

\begin{figure}
\begin{tikzpicture}[scale=1.0]
  \node at (0.5,9.3) {$x_0$};
  \node at (1.5,9.3) {$x_1$};
  \node at (2.5,9.3) {$x_2$};
  \node at (3.5,9.3) {$x_3$};
  \node at (4.5,9.3) {$x_4$};
  \node at (5.5,9.3) {$x_5$};
  \node at (6.5,9.3) {$x_6$};
  \draw[line width=0.7pt] (0.5,9.0) -- (6.5,9.0);
  \filldraw (0.5,9.0) circle (2.0pt);
  \filldraw (1.5,9.0) circle (2.0pt);
  \filldraw (2.5,9.0) circle (2.0pt);
  \filldraw (3.5,9.0) circle (4.0pt);
  \filldraw (4.5,9.0) circle (4.0pt);
  \filldraw (5.5,9.0) circle (4.0pt);
  \filldraw (6.5,9.0) circle (2.0pt);
  \draw[line width=2.5pt] (3.5,9.0) -- (5.5,9.0);
\end{tikzpicture}


\caption{\texttt{reaction.c} uses discretization \eqref{eq:nl:drfdscheme} at each interior node of the grid.  This corresponds to a three-point stencil, as shown.}
\label{fig:stencilreaction}
\end{figure}

We take the unknowns, or the corresponding grid nodes, to be the vertices of a graph $G=G(J)$, as shown in the top of Figure \ref{fig:fdcolorreaction}.  Graph $G$ has an edge between two vertices if the corresponding unknowns both appear in at least one equation, i.e.~in one instance of equation \eqref{eq:nl:drfdscheme} in this case.

\begin{figure}
\begin{tikzpicture}[scale=1.0]
\node at (-2.25,8.0) {colored graph $G(J)$:};
  \node at (0.5,8.4) {$0$};
  \node at (1.5,8.4) {$1$};
  \node at (2.5,8.4) {$2$};
  \node at (3.5,8.4) {$0$};
  \node at (4.5,8.4) {$1$};
  \node at (5.5,8.4) {$2$};
  \node at (6.5,8.4) {$0$};
  \draw (0.5,8.0) circle (4.0pt);
  \draw[line width=0.6pt] (0.65,8.0) -- (1.35,8.0);
  \draw (1.5,8.0) circle (4.0pt);
  \draw[line width=0.6pt] (1.65,8.0) -- (2.35,8.0);
  \draw (2.5,8.0) circle (4.0pt);
  \draw[line width=0.6pt] (2.65,8.0) -- (3.35,8.0);
  \draw (3.5,8.0) circle (4.0pt);
  \draw[line width=0.6pt] (3.65,8.0) -- (4.35,8.0);
  \draw (4.5,8.0) circle (4.0pt);
  \draw[line width=0.6pt] (4.65,8.0) -- (5.35,8.0);
  \draw (5.5,8.0) circle (4.0pt);
  \draw[line width=0.6pt] (5.65,8.0) -- (6.35,8.0);
  \draw (6.5,8.0) circle (4.0pt);
  \draw[line width=0.75pt] (0.6,7.9) .. controls (1.4,7.5) and (1.6,7.5) .. (2.4,7.9);
  \draw[line width=0.75pt] (1.6,7.9) .. controls (2.4,7.5) and (2.6,7.5) .. (3.4,7.9);
  \draw[line width=0.75pt] (2.6,7.9) .. controls (3.4,7.5) and (3.6,7.5) .. (4.4,7.9);
  \draw[line width=0.75pt] (3.6,7.9) .. controls (4.4,7.5) and (4.6,7.5) .. (5.4,7.9);
  \draw[line width=0.75pt] (4.6,7.9) .. controls (5.4,7.5) and (5.6,7.5) .. (6.4,7.9);
\node at (-2.4,6.9) {colored columns of $J$:};
  %\draw[xstep=1.0,ystep=1.0,black,thin] (0.0,0.0) grid (7.0,7.0);
  \node at (0.5,7.0) {$0$};
  \node at (0.5,6.2) {$0$};
  \node at (1.5,7.0) {$1$};
  \node at (1.5,6.2) {$1$};
  \node at (1.5,5.4) {$1$};
  \node at (2.5,6.2) {$2$};
  \node at (2.5,5.4) {$2$};
  \node at (2.5,4.6) {$2$};
  \node at (3.5,5.4) {$0$};
  \node at (3.5,4.6) {$0$};
  \node at (3.5,3.8) {$0$};
  \node at (4.5,4.6) {$1$};
  \node at (4.5,3.8) {$1$};
  \node at (4.5,3.0) {$1$};
  \node at (5.5,3.8) {$2$};
  \node at (5.5,3.0) {$2$};
  \node at (5.5,2.2) {$2$};
  \node at (6.5,3.0) {$0$};
  \node at (6.5,2.2) {$0$};
\end{tikzpicture}


\caption{Graph $G(J)$ is built from the stencil and the grid, with an edge for every pair of unknowns that appears in an equation in system \eqref{eq:nl:equation}.  Coloring this graph---$c=3$ colors suffice in this case---also assigns colors to the columns of $J$.}
\label{fig:fdcolorreaction}
\end{figure}

Now suppose we color the vertices so that any two adjacent vertices have distinct colors.  The coloring in Figure \ref{fig:fdcolorreaction} uses $c=3$ colors.  In the case of this simple graph, $c=3$ is the minimum number needed, and in this case $\chi(G)=c$ where $\chi(G)$ is the \emph{chromatic number} of $G$ \citep{ChartrandLesniakZhang2011}.

The case of a structured 2D grid, as used in Chapter \ref{chap:st} with a finite-difference scheme \eqref{poissonsquareFD} for the Poisson equation \eqref{poissonsquare}, further illustrates the coloring idea.  The stencil of the scheme involves five unknowns $u_{i,j+1}, u_{i-1,j}, u_{i,j}, u_{i+1,j}, u_{i,j-1}$ (Figure \ref{fig:unitsquaregridstencil}).  Therefore the graph $G(J)$ in this case includes a complete graph on five vertices, a $K_5$ \citep{ChartrandLesniakZhang2011}, as a subgraph at each (generic interior) node in the structured grid, as shown in Figure 4.7.  Clearly $\chi(G(J))\ge 5$.  When \PETSc uses a coloring algorithm then it finds that indeed five colors suffice to color $G(J)$.\sidenote{At least in refined-grid cases using the default incidence-degree-ordering.  Certain coarse grids have lower chromatic number.}  We will be able to verify this behavior when we have a \pSNES-based 2D PDE example to work with, as in the next Chapter.
% for number of colors on graph in fig:colorstencilplane, in refined-grid cases, compare
%   $ ./ex5 -da_refine 4 -log_view -snes_fd_color | grep SNESFunctionEval
% which gives 19 function evals and
%   $ ./ex5 -da_refine 4 -snes_converged_reason -snes_fd_color
% which gives 3 iterations
% we conclude  19=N(c+1)+1=3(c+1)+1  so c = 5

\begin{figure}
\begin{tikzpicture}[scale=7.0]
% mesh
  \draw[xstep=0.25,ystep=0.166667,black,thin,dashed] (0.0,0.333334) grid (1.0,1.0);
% stencil
  \filldraw (0.50,0.666667) circle (0.6pt);
  \filldraw (0.75,0.666667) circle (0.6pt);
  \filldraw (1.00,0.666667) circle (0.6pt);
  \filldraw (0.75,0.5) circle (0.6pt);
  \filldraw (0.75,0.833333) circle (0.6pt);
% subgraph K_5
  \draw[line width=2.5pt] (0.50,0.666667) -- (1.00,0.666667);
  \draw[line width=2.5pt] (0.75,0.5)  -- (0.75,0.833333);
  \draw[line width=0.75pt] (0.5,0.666667) -- (0.75,0.833333) -- (1.0,0.666667) -- (0.75,0.5) -- cycle;
  \draw[line width=0.75pt] (0.5,0.666667) .. controls (0.55,0.72) and (0.7,0.72) .. (0.75,0.666667);
  \draw[line width=0.75pt] (0.75,0.666667) .. controls (0.8,0.72) and (0.95,0.72) .. (1.0,0.666667);
  \draw[line width=0.75pt] (0.75,0.833333) .. controls (0.8,0.8) and (0.8,0.7) .. (0.75,0.666667);
  \draw[line width=0.75pt] (0.75,0.666667) .. controls (0.8,0.633333) and (0.8,0.533333) .. (0.75,0.5);
  \draw[line width=0.75pt] (0.5,0.666667) .. controls (0.7,1.0) and (0.8,1.0) .. (1.0,0.666667);
  \draw[line width=0.75pt] (0.75,0.833333) .. controls (0.3,0.8) and (0.3,0.533333) .. (0.75,0.5);
\end{tikzpicture}


\caption{For the 2D finite-difference scheme used in Chapter \ref{chap:st}, the graph $G(J)$ has a $K_5$ at every node (thin lines) because the stencil (thick lines) involves five unknowns.}
\label{fig:colorstencilplane}
\end{figure}

Optimally-coloring a graph is a hard problem which \PETSc does not attempt to solve.  \PETSc includes ``incidence-degree ordering'' [\emph{default}] and ``smallest-last ordering'' graph-coloring algorithms which provide $c$-colorings for which $c \le \alpha N^{1/2} \chi(G(J))$ \citep{ColemanMore1983}, for some constant $\alpha>0$, where $N$ is the number of equations in \eqref{eq:nl:equation}.  These algorithms also do well in the informal sense that $c$ is close to $\chi(G(J))$ for a large selection of test matrices \citep{ColemanMore1983}.  Furthermore, these algorithms run in time determined by the number of nonzeros in rows of $J$.  In practice we get a reasonably-good coloring in substantially-less than $O(N^2)$, and often in $O(N)$, time for problems coming from discretized PDEs.  (Recall we are trying to improve on the existing $O(N^2)$ cost of the \texttt{-snes\_fd} method.)

The graph coloring also implies a coloring of the columns of $J$, as shown in Figure \ref{fig:fdcolorreaction}.  (Each row has distinct colors, which is one way to describe our purpose here.)  The first step in modifying the finite-difference formulas is to generate vectors $\bv_0, \bv_1, \dots, \bv_{c-1} \in \RR^N$ by the rule that $\bv_k$ has a $1$ in entry $j$ if $k$ is the color of unknown $j$, and is otherwise zero.  In the $c=3$ case shown in Figure \ref{fig:fdcolorreaction},
\begin{equation}
\bv_0 = \begin{bmatrix} 1 \\ 0 \\ 0 \\ 1 \\ 0 \\ 0 \\ 1 \end{bmatrix}, \qquad
\bv_1 = \begin{bmatrix} 0 \\ 1 \\ 0 \\ 0 \\ 1 \\ 0 \\ 0 \end{bmatrix}, \qquad
\bv_2 = \begin{bmatrix} 0 \\ 0 \\ 1 \\ 0 \\ 0 \\ 1 \\ 0 \end{bmatrix}.
 \label{eq:nl:coloringvecs}
\end{equation}
We may define a function $k = k(j)$ which maps from node (unknown) index $j$ to color $k$ so that $(\bv_k)_j = \delta_{k,k(j)}$; in this case $k(j) = j\mod 3$.

Then we replace \eqref{eq:nl:fdjac} with
\begin{equation}
J_{ij} = \frac{\partial F_i}{\partial x_j} \approx \frac{F_i(\bx+\delta \bv_{k(j)}) - F_i(\bx)}{\delta}.  \label{eq:nl:fdcolorjac}
\end{equation}
The right sides of \eqref{eq:nl:fdjac} and \eqref{eq:nl:fdcolorjac} compute exactly the same entries $J_{ij}$, but \eqref{eq:nl:fdcolorjac} requires far fewer evaluations of $\bF$.  In particular, all columns of $J$ with color $k$ are computed by \eqref{eq:nl:fdcolorjac} just using the \emph{smallest} $j$ for which $k(j)=k$.  Thus, given a $c$-coloring of $G(J)$ there are exactly $c$ evaluations $\bF(\bx+\delta \bv_{k(j)})$, plus one more for $\bF(\bx)$, to fill $J$ using \eqref{eq:nl:fdcolorjac}.

The news is good when we actually try it.  The result is almost as fast in this 1D PDE case as using the analytical Jacobian:
\begin{cline}
$ timer ./reaction -snes_converged_reason -da_refine 10 -snes_fd_color
Nonlinear solve converged due to CONVERGED_FNORM_RELATIVE iterations 3
on 8193 point grid:  |u-u_exact|_inf/|u|_inf = 1.75633e-10
real 0.06
\end{cline}
%$
Compare the run on page \pageref{etc:nl:bestreaction}.

Counting the number of function evaluations in the various cases is a good idea, and there is no need to alter \texttt{reaction.c} to do so: use \texttt{-log\_view} and look at the output.\sidenote{Do not add print statements!}  Because we want the ``\texttt{SNESFunctionEval}'' lines, \texttt{grep} extracts the count for the runs above:
\begin{cline}
$ ./reaction -da_refine 10 -snes_fd -snes_max_funcs 100000 -log_view | grep SNESFunctionEval
SNESFunctionEval   24586 1.0 3.3170e+00 1.0 ...
$ ./reaction -da_refine 10 -snes_fd_color -log_view | grep SNESFunctionEval
SNESFunctionEval      13 1.0 4.7762e-03 1.0 ...
$./reaction -da_refine 10 -log_view | grep SNESFunctionEval
SNESFunctionEval       4 1.0 9.3341e-04 1.0 ...
\end{cline}
%$
The number of $\bF$ evaluations was 24586, 13, and 4, respectively.


\section{Jacobian-free Newton-Krylov (JFNK)} \label{sec:JFNK}

Besides the finite-difference Jacobian approach above, using equation \eqref{eq:nl:fdjac} or \eqref{eq:nl:fdcolorjac} to compute the entries of the matrix, there is another approach which requires no assembled Jacobian matrix at all.  It approximates matrix-vector products with the Jacobian matrix by a finite difference formula, and uses a Krylov-type method to solve \eqref{eq:nl:newton:eq} from the vector subspace
\begin{equation}
    \mathcal{K} = \operatorname{span}\{\br,J\br,J^2\br,\dots,J^{m-1}\br\}. \label{eq:nl:krylovagain}
\end{equation}
Here $J=J_{\bF}(\bx_k)$ is the Jacobian at iterate $k$ and $\br$ is a (linear) residual in equation \eqref{eq:nl:newton:eq}.

It goes by the name ``Jacobian-free Newton-Krylov'' \citep{KnollKeyes2004} or just ``JFNK''.  To give more detail, suppose we use initial estimate $0$ for the solution to \eqref{eq:nl:newton:eq}.  The residual is $\br = - \bF(\bx) - J 0 = - \bF(\bx)$.  To compute $J\br$, $J^2\br=J(J\br)$, \dots in \eqref{eq:nl:krylovagain}, recall definition \eqref{eq:nl:expandF} of the derivative of $\bF$.  It implies that if $\bv$ is any vector then
\begin{equation}
J \bv \approx \frac{\bF(\bx+\delta \bv) - \bF(\bx)}{\delta} \label{eq:nl:JFNKbasic}
\end{equation}
for small $\delta \ne 0$.  Approximating the Jacobian-vector product by \eqref{eq:nl:JFNKbasic} avoids evaluation of any entries of $J$.

Computing $\br = - \bF(\bx)$, to get the Krylov method started, requires evaluating $\bF$.  Then each successive vector in basis \eqref{eq:nl:krylovagain} requires an additional evaluation of $\bF$, namely
\begin{equation}
J^{\ell} \br \approx \frac{\bF(\bx+\delta J^{\ell-1}\br) - \bF(\bx)}{\delta} \label{eq:nl:JFNKiteration}
\end{equation}
for $\ell=1,\dots,m$.  Equation \eqref{eq:nl:JFNKiteration}, the central calculation in JFNK, uses $m$ evaluations of $\bF$ to compute the whole Krylov basis \eqref{eq:nl:krylovagain}.  By contrast, use of finite difference formula \eqref{eq:nl:fdjac} to compute a generic $N\times N$ Jacobian $J$ requires $N$ evaluations of $\bF$.  Thus JFNK can be efficient if \eqref{eq:nl:newton:eq} is solved to desired tolerance in $m\ll N$ Krylov iterations.

However, there are a few basic points to consider.  First, once you have a matrix you can do many things, such as incomplete matrix factorizations, other than computing a matrix-vector product or a Krylov basis, so giving up on matrices may be premature.  Second, and related, we may need to precondition the linear equation \eqref{eq:nl:newton:eq}, in which case we want the Krylov space for the linear operator $M^{-1}J$, not for $J$ itself.  Also, in cases such as structured-grid PDE schemes where it can be applied, we have seen that coloring greatly reduces the cost of using \eqref{eq:nl:fdjac} to assemble $J$.  JFNK is a good strategy to the extent that it actually \emph{works}, and to the extent it beats the competition.

Note JFNK is implemented in \PETSc and invoked by option \texttt{-snes\_mf}, where ``\texttt{mf}'' in stands for ``matrix-free.''\sidenote{The \PETSc API provides functions for even more flexibility and generality than are available at the command line with options \texttt{-snes\_mf}, \texttt{-snes\_fd}, \texttt{-snes\_fd\_color}, and so on.}  We shall see below, in the important and even essential preconditioned case, that there may be a matrix involved in JFNK anyway.  The method never involves fully-assembling the Jacobian matrix itself, however, and thus the ``Jacobian-free'' label is justified.

The issues above can be pursued in more detail on a concrete example, so let us give it a try on a very coarse grid in the 1D diffusion-reaction PDE example \texttt{reaction.c}:
\begin{cline}
$ ./reaction -snes_converged_reason -snes_mf
Nonlinear solve converged due to CONVERGED_FNORM_RELATIVE iterations 3
on 9 point grid:  |u-u_exact|_inf/|u|_inf = 0.000188753
\end{cline}
%$
So far so good.

Any enthusiasm is rather damped when we try refined grids.  Noting $J$ is symmetric in this case,\sidenote{Well \dots nearly so.  See exercise 4.\ref{exer:nl:symmetrizeJ}.} we try the CG method (Chapter \ref{chap:ls}) and ask \PETSc for both the number of Newton and Krylov iterations.  On 500 and 1000 point grids the performance is poor in the sense that too many ``inner'' Krylov iterations are needed:
\begin{cline}
$ ./reaction -snes_converged_reason -snes_mf -ksp_converged_reason -ksp_type cg -da_refine 6
  Linear solve converged due to CONVERGED_RTOL iterations 511
  Linear solve converged due to CONVERGED_RTOL iterations 504
  Linear solve converged due to CONVERGED_RTOL iterations 508
Nonlinear solve converged due to CONVERGED_FNORM_RELATIVE iterations 3
on 513 point grid:  |u-u_exact|_inf/|u|_inf = 4.62255e-08
$ ./reaction -snes_converged_reason -snes_mf -ksp_converged_reason -ksp_type cg -da_refine 7
  Linear solve converged due to CONVERGED_RTOL iterations 1023
  Linear solve converged due to CONVERGED_RTOL iterations 1006
  Linear solve converged due to CONVERGED_RTOL iterations 1015
  Linear solve converged due to CONVERGED_RTOL iterations 950
Nonlinear solve converged due to CONVERGED_FNORM_RELATIVE iterations 4
on 1025 point grid:  |u-u_exact|_inf/|u|_inf = 1.15576e-08
\end{cline}
These large iteration counts, which illustrate the well-known doubling of CG iterations with each doubling of dimension, also arose in Chapter \ref{chap:ls} when we looked at unpreconditioned Krylov methods.

Observe that the default GMRES solver is no better; we simply get a linear solve failure:
\begin{cline}
$ ./reaction -snes_converged_reason -snes_mf -ksp_converged_reason -da_refine 7
  Linear solve did not converge due to DIVERGED_ITS iterations 10000
Nonlinear solve did not converge due to DIVERGED_LINEAR_SOLVE iterations 0
on 1025 point grid:  |u-u_exact|_inf/|u|_inf = 0.217386
\end{cline}
%$

Here we see an excellent reminder to apply the Krylov solver to a \emph{preconditioned} form of linear equation \eqref{eq:nl:newton:eq}.  In this case left- and right-sided preconditioning, equations \eqref{introleftpre} and \eqref{introrightpre} in Chapter \ref{chap:ls}, are
\begin{align}
(M^{-1} J) \bs &= - M^{-1} \bF(\bx_k), \label{eq:nl:leftpre} \\
(J M^{-1}) (M \bs) &= - \bF(\bx_k), \label{eq:nl:rightpre}
\end{align}
respectively.  Regarding their implementation in the Krylov iteration, in \eqref{eq:nl:leftpre} the action of $M^{-1} J$ on some vector $\bv$ is a straightforward composition of \eqref{eq:nl:JFNKbasic} followed by application of $M^{-1}$.  In right preconditioning \eqref{eq:nl:rightpre} the action of $J M^{-1}$ on $\bv$ is computed by first applying $M^{-1}$ to $\bv$, namely by solving $M \by = \bv$, and then using \eqref{eq:nl:JFNKbasic}:
\begin{equation}
(J M^{-1}) \bv \approx \frac{\bF(\bx+\delta \by) - \bF(\bx)}{\delta} \label{eq:nl:JFNKwithrightpre}
\end{equation}

Also recall that preconditioning matrices $M$ and $M^{-1}$ are usually never assembled.  Even if an assembled preconditioner-material matrix $P$ is present, $M$ is generally constructed nontrivially from $P$.  For example, if we supply some assembled matrix $P$ as the preconditioner material, but we ask for ILU($0$) preconditioning, then $M$ is actually the product of the ILU($0$) factors of $P$, not $P$ itself.


\section{Jacobian cases} \label{sec:jacobiancases}

\def\checkmark{\tikz\fill[scale=0.4](0,.35) -- (.25,0) -- (.7,.8) -- (.25,.15) -- cycle;}
\def\bigcheckmark{\tikz\fill[scale=0.6](0,.35) -- (.25,0) -- (.7,.8) -- (.25,.15) -- cycle;}

At this point we risk overwhelming the reader with options, so we pause to review the possibilities before testing them on the \texttt{reaction.c} problem.

Table \ref{tab:snesjacobianoptions:needed} summarizes options relating to residual and Jacobian call-backs in \pSNES-using codes.  If no Jacobian or approximate Jacobian routine is provided in the user-written code then only finite-difference evaluation of an assembled Jacobian matrix (i.e.~\texttt{-snes\_fd} or \texttt{-snes\_fd\_color}) or finite-difference evaluation of the Jacobian-vector product inside the Krylov method (\texttt{-snes\_mf}, i.e.~JFNK) are available.  If a Jacobian routine \emph{is} provided then it is used in the Newton iteration (when no alternative option is given).

However, option \texttt{-snes\_mf\_operator}, which we have not mentioned until now, tells the \pSNES to only use the provided Jacobian when preconditioning the JFNK Jacobian-vector product.  This last option may give quadratic convergence even if the provided Jacobian is inexact, that is, even if the ``material'' $P$ which is used to build $M$ in formulas \eqref{eq:nl:leftpre}--\eqref{eq:nl:JFNKwithrightpre} is only an approximation of the Jacobian $J$.

\begin{table}
\begin{tabular}{lllll}
 &\underline{\small no option}\hspace{0.0in} & \underline{\small\hspace{-0.05in}$\begin{array}{ll} \text{\texttt{-snes\_fd}} \\ \text{\texttt{-snes\_fd\_color}}\end{array}$} & \underline{\small \texttt{-snes\_mf}} & \underline{\small \texttt{-snes\_mf\_operator}} \\
only $\bF$      & error           & $\bigcheckmark$ & $\bigcheckmark$ & error \\
$\bF$ and $P$   & $\checkmark$ & $\checkmark$    & $\checkmark$    & $\bigcheckmark$ \\
$\bF$ and $J$   & $\bigcheckmark$ & $\checkmark$    & $\checkmark$    & $\checkmark$
\end{tabular}
\caption{Jacobian options when using \pSNES.  The row labels describe which mathematical objects have been provided through user-implemented code.  Symbol ``$P$'' denotes an approximate Jacobian while ``$J$'' denotes the actual Jacobian.  A larger check mark shows recommended usage.} \label{tab:snesjacobianoptions:needed}
\end{table}

From the code side, at a minimum a method for $\bF$ must be implemented in all cases, as there is no other way for \PETSc to know what equations you are solving.\sidenote{Except in the case where the residual function $\bF$ itself represents a derivative (gradient) of an objective function.  See Chapter \ref{chap:of}.}   There are two ways of providing $\bF$:  If the problem is based on a structured grid, as in \texttt{reaction.c} above, use \texttt{DMDASNESSetFunctionLocal()}.  In general, use \texttt{SNESSetFunction()}.  If only $\bF$ is provided, do not create or preallocate a \pMat for the Jacobian, as this is done internally by the \pSNES for \texttt{-snes\_fd} and \texttt{-snes\_fd\_color} runs, while no assembled Jacobian \pMat exists for \texttt{-snes\_mf} runs.

If an exact ($J$) or approximate ($P$) Jacobian function is implemented then there are two cases:
\begin{quote}
  \renewcommand{\labelenumi}{(\roman{enumi})}
  \begin{enumerate}
  \item On a structured grid, provide the function (e.g.~``\texttt{FormJacobianLocal()}'') using \texttt{DMDASNESSetJacobianLocal()}.  In this case the \pMat holding the Jacobian is internal, and so it is not created or preallocated by the user.
  \item In a general, first declare a \pMat variable, say ``\texttt{J}'', and then call \texttt{MatCreate()}, \texttt{MatSetSizes()}, and \texttt{MatSetFromOptions()} on it.  Then call either \texttt{MatSetUp()} or \texttt{MatXAIJPreallocate()} on \texttt{J}, to allocate space for the assembled Jacobian matrix.  Then provide the call-back code for function $J$ or $P$ (e.g.~``\texttt{FormJacobian()}'') and the \pMat \texttt{J} for both \pMat arguments through \texttt{SNESSetJacobian()}.
  \end{enumerate}
\end{quote}
In either case (i) or (ii), the user's Jacobian code only needs to set values in the second ``preconditioner'' \pMat argument.  (This assumes in (ii) that \texttt{J} is provided for both \pMat arguments of \texttt{SNESSetJacobian()}, as described.)  The code should, however, \emph{assemble} both \pMat arguments.

As practical advice for the debugging stage, you know you have correctly- and fully-implemented the Jacobian, and you have found an adequate initial iterate, if all four options (columns) in Table \ref{tab:snesjacobianoptions:needed} work and each gives apparent quadratic convergence when looking at the residual norm decay shown by \texttt{-snes\_monitor}.  As the reader may check, this is the situation for the last two codes \texttt{ecjacobian.c} and \texttt{reaction.c}.

Another way to compare the options, especially in a structured PDE case where option \texttt{-snes\_fd\_color} might be effective, is shown in Table \ref{tab:snesjacobianoptions:evals}.  This Table supposes, rather unrealistically, that the number $q$ of Newton iterations is independent of the option.  For the last column it also assumes, again unrealistically, that the number of Krylov iterations is the same at each Newton iteration (in JFNK).  The Table two simple points:\begin{itemize}
\item Improved performance with \texttt{-snes\_fd\_color} comes from having a $c$-coloring where $c \ll N$.
\item Improved performance with \texttt{-snes\_mf\_operator} comes from reducing the number $m$ of Krylov iterations.  We know from Chapter \ref{chap:ls} that this is primarily an issue of choosing the right preconditioner so that $m_2 \ll m_1$.
\end{itemize}

\begin{table}
\begin{tabular}{lccc}
 &\underline{no option}
                       & \small\underline{$\begin{array}{ll} \text{\texttt{-snes\_fd}} \\ \text{\texttt{-snes\_fd\_color}} \end{array}$}
                                    & \small\underline{$\begin{array}{ll} \text{\texttt{-snes\_mf}} \\ \text{\texttt{-snes\_mf\_operator}} \end{array}$} \vspace{0.2in} \\
residual $\bF$        & $q+1$ & $\begin{array}{cc} q(N+1)+1 \\ q(c+1)+1 \end{array}$ & $\begin{array}{cc} q m_1 \\ q m_2 \end{array}$ \vspace{0.1in} \\
Jacobian $J$ or $P$    & $q$ & $\begin{array}{cc} 0 \\ 0 \end{array}$ & $\begin{array}{cc} 0 \\ q \end{array}$
\end{tabular}
\caption{Jacobian options compared by number of residual and Jacobian evaluations.  Here $q$ is the number of Newton iterations, $N$ is the dimension of the problem, $c$ is the number of colors on $G(J)$, $m_1$ is the dimension of the Krylov space for $J$, and $m_2$ is the Krylov space dimension for the preconditioned operator $M^{-1}J$.} \label{tab:snesjacobianoptions:evals}
\end{table}


\section{Testing JFNK with preconditioning} \label{sec:testsnesmfoperator}

We now do refined-grid runs of \texttt{reaction.c} to show the \texttt{-snes\_mf\_operator} option in action.  We have already seen that the \texttt{-snes\_fd\_color} option is effective for reducing evaluations of $\bF$ if a Jacobian is not implemented, but in the same case the un-preconditioned JFNK method \texttt{-snes\_mf} has serious difficulties as it requires unreasonable numbers of Krylov iterations and function evaluations.  Now we can show that \texttt{-snes\_mf\_operator} using only a rough approximation of the Jacobian as a preconditioner gives good performance.

Specifically, suppose we modify this line in \texttt{reaction.c} (Code \ref{code:reactionI}),
\begin{code}
    col[1] = i;    v[1] = 2.0 - h*h * dRdu;
\end{code}
to remove ``\texttt{- h*h * dRdu}'', corresponding to the nonlinear term $\rho \sqrt{u}$ in \eqref{eq:nl:drsqrt}, to obtain
\begin{code}
    col[1] = i;    v[1] = 2.0;
\end{code}
This change, which we call a ``\texttt{J->P}'' below, keeps the tridiagonal sparsity pattern of the Jacobian.  Furthermore it preserves many characteristics of the spectrum of the linearizations of the operator in \eqref{eq:nl:drsqrt} by keeping the highest-order term $u''$.

With this change, convergence is slowed for no option, that is, when we try to use the new approximate Jacobian as though it were exact.  Specifically, on a coarse grid the number of iterations goes from 4 to 15, and the residual norms suggest that convergence is no longer quadratic:
\begin{cline}
$ ./reaction -da_refine 4 -snes_monitor    # before change
  0 SNES Function norm 1.671129624018e-02 
  1 SNES Function norm 3.609252641302e-04 
  2 SNES Function norm 4.167490508951e-07 
  3 SNES Function norm 4.935230509260e-13 
on 129 point grid:  |u-u_exact|_inf/|u|_inf = 7.39662e-07
$ ./reaction -da_refine 4 -snes_monitor    # with J->P change
  0 SNES Function norm 1.671129624018e-02 
  1 SNES Function norm 3.822032062916e-03 
...
 14 SNES Function norm 3.879363487638e-10 
 15 SNES Function norm 1.119521798815e-10 
on 129 point grid:  |u-u_exact|_inf/|u|_inf = 7.38159e-07
\end{cline}
This loss of Newton method performance from a significantly incorrect Jacobian is expected in theory \citep{Kelley2003}.

However, for this modified Jacobian case, \texttt{-snes\_mf\_operator} is now a fast option for higher-resolution grids, fully competitive with the exact Jacobian and finite-difference-by-coloring cases already seen.  Again on a $8000$ point grid, using the approximate Jacobian ``as is'' causes too many Newton iterations and less-than-quadratic convergence:
\begin{cline}
$ timer ./reaction -snes_converged_reason -da_refine 10    # with J->P change
Nonlinear solve converged due to CONVERGED_FNORM_RELATIVE iterations 15
on 8193 point grid:  |u-u_exact|_inf/|u|_inf = 1.36236e-09
real 0.13
\end{cline}
%$
Now we try \texttt{-snes\_mf\_operator}, which is designed for this approximate-Jacobian situation:
\begin{cline}
$ timer ./reaction -snes_converged_reason -da_refine 10 -snes_mf_operator  # with J->P change
Nonlinear solve converged due to CONVERGED_FNORM_RELATIVE iterations 4
on 8193 point grid:  |u-u_exact|_inf/|u|_inf = 1.80588e-10
real 0.05
\end{cline}
%$
The number of iterations and the time are significantly reduced, and both are comparable to the exact Jacobian case (page \pageref{etc:nl:bestreaction}).

One might give the following summary advice on \pSNES usage:

\begin{quote}
Before implementing a Jacobian, try finite-difference evaluation \texttt{-snes\_fd} first, including a look at whether coloring (option \texttt{-snes\_fd\_color}) applies to your case.  As a general rule, it can be applied and it is often effective on a structured grid, but on an unstructured grid (Chapters \ref{chap:un} and \ref{chap:dp}) the coloring method requires additional work.  Note JFNK with no preconditioning (option \texttt{-snes\_mf}) is rarely effective, though easy to try.  Now consider implementing the exact Jacobian $J$.  If that seems like too much work or is too error-prone, consider a simpler approximate Jacobian $P$ used with option \texttt{-snes\_mf\_operator}.  In PDE cases, in particular, $P$ might only capture the highest-order derivatives in $J$.  To test convergence with $P$, compare ``no option'' runs which use $P$ as though it were $J$, but wherein less-than-quadratic convergence is expected, and \texttt{-snes\_mf\_operator} runs where quadratic convergence should be recovered.
\end{quote}


\section{Line search Newton methods} \label{sec:linesearch}

Once the Newton step $\bs_k$ solving equation \eqref{eq:nl:newton:eq} is computed, the new iterate $\bx_{k+1} = \bx_k + \bs_k$ from \eqref{eq:nl:newton:update} may not be what we want.  For example, the residual norm $\|\bF(\bx_k + \bs_k)\|$ may exceed $\|\bF(\bx_{k})\|$, suggesting that no progress is being made in solving \eqref{eq:nl:equation}.  Improving this situation is the problem of \emph{globalizing} the convergence of the Newton iteration, so that the iterates $\bx_k$ are more likely to head toward the sometimes-small region of quadratic convergence around the solution to \eqref{eq:nl:equation}.

The \emph{line search} globalization technique \citep{DennisSchnabel1983} is to replace \eqref{eq:nl:newton:update} with
\begin{equation}
\bx_{k+1} = \bx_k + \lambda_k \bs_k.  \label{eq:nl:linesearchupdate}
\end{equation}
That is, we accept $\bs_k$ as a search direction for solving \eqref{eq:nl:newton:eq}, but we may use an actual step of less ($\lambda_k < 1$), or sometimes more ($\lambda_k > 1$), than in \eqref{eq:nl:newton:update}.  The question, of course, is how to choose $\lambda_k$.

As we are solving a system of nonlinear equations \eqref{eq:nl:equation}, there is no pre-existing sense of ``better'' or ``worse'' approximations, as there is in an optimization context.  But we may introduce a particular \emph{merit function} \citep{NocedalWright2006} to provide such a sense, namely
\begin{equation}
\phi(\bx) = \frac{1}{2} \|\bF(\bx)\|_2^2.  \label{eq:nl:linesearchmeritfunction}
\end{equation}
This merit function is used in all \pSNES line searches, unless an objective function is provided to the \pSNES (next Chapter).

To determine $\lambda_k$ one could perhaps solve the one-dimensional optimization problem
\begin{equation}
\min_{0<\lambda} \phi(\bx_k + \lambda \bs_k),  \label{eq:nl:linesearchperhapsopt}
\end{equation}
thereby making the residual 2-norm as small as possible along the search direction.  However, solving this optimization problem accurately is rarely-justified because we are merely using the solution to compute the next Newton iterate; we hope the Newton iteration itself, with $\lambda=1$, generates rapid convergence.

A practical line-search tries candidate values of $\lambda=\lambda_k$, testing them for whether they reduce $\phi(\bx_k + \lambda \bs_k)$.  The initial try is usually $\lambda=1$.  A \emph{sufficient  decrease} condition with parameter $\alpha\in(0,1)$ is to require
\begin{equation}
\phi(\bx_k + \lambda \bs_k) \le \phi(\bx_k) + \alpha \lambda\, \bs_k^\top \grad \phi(\bx_k).  \label{eq:nl:linesearchsuffdecrease}
\end{equation}
The \PETSc default for $\alpha$ is $10^{-4}$, corresponding to an easy-to-satisfy reduction criterion, and the Newton iteration is stopped with an error if such a decrease cannot be found.

The quantity $\bs_k^\top \grad \phi(\bx_k)$ in \eqref{eq:nl:linesearchsuffdecrease} is an initial slope of $\phi(\bx)$ as we move in the search direction---it is the directional derivative in the direction $\bs_k$.  In fact, for merit function \eqref{eq:nl:linesearchmeritfunction} it is always negative, as we see by using \eqref{eq:nl:newton:eq}:
\begin{align}
\bs_k^\top \grad \phi(\bx_k) &= \bs_k^\top J_{\bF}(\bx_k)^\top \bF(\bx_k) = \left(J_{\bF}(\bx_k) \bs_k\right)^\top \bF(\bx_k) \label{eq:nl:linesearchinitslope} \\
  &= - \|\bF(\bx_k)\|_2^2. \notag
\end{align}
Thus $\bs_k$ is a descent direction for merit function \eqref{eq:nl:linesearchmeritfunction}.  Also notice that sufficient decrease condition \eqref{eq:nl:linesearchsuffdecrease} can be evaluated using only residual evaluations $\bF(\bx_k)$ and $\bF(\bx_k + \lambda \bs_k)$.

Merit function \eqref{eq:nl:linesearchmeritfunction} has another nice property.  Though $\phi(\bx)$ may have local minima $\bx'$ other than solutions $\bx^*$ to \eqref{eq:nl:equation}, the Jacobian must also be singular at such locations.  In fact,
\begin{equation}
0 = \grad \phi (\bx') = J_{\bF}(\bx')^\top \bF(\bx'),
\end{equation}
so $J_{\bF}(\bx')^\top$ has a nonzero null-space if $\bF(\bx')\ne 0$.  Thus, for example, no non-solution local minima arise in a region where there is a finite bound on the condition number of the Jacobian.

The line search options available in \PETSc are summarized in Table \ref{tab:nl:linesearchoptions}.  The Table has only minimal information, so for more details see \citep[Chapter 6]{DennisSchnabel1983} and the \PETSc source code itself.

\begin{table}
\begin{tabular}{ll}
\underline{Name} & \underline{Summary} \vspace{0.05in} \\ \vspace{0.1in}
\texttt{basic} & No line search.  Use full Newton step $\lambda_k=1$ in \eqref{eq:nl:linesearchupdate}. \\ \vspace{0.1in}
\texttt{bt} [\emph{default}] & \begin{minipage}[t]{0.8\textwidth}
Polynomial-fit back-tracking \citep[section 6.3.2]{DennisSchnabel1983}.  A quadratic or cubic [\emph{default}] polynomial is built up as a model for \mbox{$\phi(\bx_k+\lambda \bs_k)$}, though initially the model is always quadratic.  Successive values $\lambda$ are required to decrease.
\end{minipage} \\ \vspace{0.1in}
\texttt{cp} & \begin{minipage}[t]{0.83\textwidth}
Assumes $\bF$ is the gradient of an unknown objective function, i.e.~\mbox{$\bF=\grad g$}.  A critical point of $g$ is then found along the search direction by the secant method.
\end{minipage} \\ \vspace{0.1in}
\texttt{l2} & \begin{minipage}[t]{0.83\textwidth}
Secant-line minimization along the search direction, starting with testing \mbox{$\lambda=1/2$} and \mbox{$\lambda=1$}. Repeated a fixed number of times.
\end{minipage}
\end{tabular}
\caption{Line search methods (types) for \pSNES.  Use \texttt{-snes\_linesearch\_type X} to choose one.  Only types \texttt{bt} and \texttt{l2} use an objective function, supposing it is provided by a call to \texttt{SNESSetObjective()}.  Sufficient decrease \eqref{eq:nl:linesearchsuffdecrease} is only tested in type \texttt{bt}.} \label{tab:nl:linesearchoptions}
\end{table}

One chooses line search type \texttt{X} by option \texttt{-snes\_linesearch\_type X}.  Option \texttt{-snes\_linesearch\_monitor} shows runtime diagnostics.  Parameter $\alpha$ in sufficient decrease condition \eqref{eq:nl:linesearchsuffdecrease} can be set, in the rare case where this is needed, by \texttt{-snes\_linesearch\_alpha}.  The initial value for $\lambda_k$ can be set by \texttt{-snes\_linesearch\_damping}, with default of $1$.  Option \texttt{-snes\_linesearch\_max\_it} sets the maximum number of line search tries for type \texttt{bt}, but for types \texttt{l2} and \texttt{cp} it sets the fixed number of repetitions of the secant line calculation.  As usual, find all such options by
\begin{cline}
$ ./reaction -help | grep snes_linesearch
\end{cline}
%$

We have actually only been describing one type of \pSNES solver in the last section, namely the line-search \pSNES type chosen by \texttt{-snes\_type newtonls}.  There are also trust region methods \citep{NocedalWright2006} (\texttt{-snes\_type newtontr}) as well as variations on nonlinear preconditioning, but they are not used in this book.  See the \PETSc User's Manual \citep{petsc-user-ref}.


\section{Exercises}

\renewcommand{\labelenumi}{\arabic{chapter}.\arabic{enumi}\quad}
\renewcommand{\labelenumii}{(\alph{enumii})}
\begin{enumerate}
\item One needs to \emph{see} quadratic convergence to believe it.  Observe that for $x_k$ in both parts (b) and (c) below, the \emph{number of correct digits in $x_k$ doubles} at each iteration.
    \begin{enumerate}
    \item The sequence $x_k = 1-2^{-n}$ converges to $x^*=1$, but linearly instead of quadratically.  Find $k$ so that $|e_k| < 10^{-16}$.
    \item The sequence $x_k = 1-2^{(-2^n)}$ converges quadratically to $x^*=1$.  Find $k$ so that $|e_k| < 10^{-16}$.  Find the smallest $K$ so that $|e_{k+1}| \le K |e_k|^2$ for all $k$.
    \item Let $F(x) = \cos(x-1) - \exp(1-x)$ and $x_0=0.5$.  Using either \PETSc or a quick-and-dirty numerical tool, compute Newton iterates $x_k$ for $k=1,\dots,6$.  Estimate $K$ so that $|e_{k+1}| \le K |e_k|^2$ for large $k$.
    \end{enumerate}

\item Make a tiny modification to \texttt{ecjacobian.c} to set the initial vector to $\bx_0 = [10\,\, 10]^\top$.  Rerun it with option \texttt{-snes\_monitor} and note it does not converge to a solution.  Why does the \pSNES stop?  Add one runtime option so that it converges, thereby producing the data shown in Figure \ref{fig:newtonconvdelayed}.

\item \label{exer:nl:snestestdisplay}  Run and interpret the result:
\begin{cline}
$ ./ecjacobian -snes_type test -snes_test_display
\end{cline}
%$

\item  \label{exer:nl:newtonatan}
This exercise shows that the original Newton iteration can diverge or enter a limit cycle, according to the initial iterate $x_0$, while using linesearch gives essentially global convergence.
    \begin{enumerate}
    \item By straightforward modifications of \texttt{expcircle.c}, write a code \texttt{atan.c} which solves nonlinear equation \eqref{eq:nl:equation} with $F(x)=\arctan(x)$ and initial iterate $x_0=2.0$.  Observe that running it \texttt{./atan -snes\_fd -snes\_linesearch\_type basic} causes it to diverge, but the default choice \texttt{-snes\_linesearch\_type bt} gives convergence.
    \item Returning to paper calculation, require the Newton iteration \eqref{eq:nl:newton} to enter a sign-flipping limit cycle, i.e.~so that $x_{k+1} = - x_k$ for all $k$.  Thereby approximate the positive value $x_0$ so that the Newton iteration both makes no progress, and remains bounded, for all $k$.\sidenote{One should solve this problem with Newton's method!}  Sketch the situation.
    \item By modifying $x_0$ in \texttt{atan.c} and using linesearch type \texttt{basic}, confirm that one can make the Newton iteration ``stick'' in this actually-unstable limit cycle for quite a while.  Then confirm that linesearch types \texttt{bt,l2,cp} all get unstuck immediately.
% to solve:  - x = x - arctan(x) (1+x^2)
% octave:
%   G = @(x) 2*x - atan(x) * (1+x^2);
%   J = @(x) 1 - atan(x) * (2*x);
%   x = 1.4;
%   format long g
%   for j = 1:4, x = x - G(x) / J(x), end
% get:
%  x = 1.39174520027073
    \end{enumerate}

\item Option \texttt{-log\_view} can show whether function evaluations or linear solves are dominating the cost of Newton's method.  By looking at \texttt{-log\_view} output, compare where the work is done in these runs: \texttt{./reaction}, \texttt{./reaction -snes\_fd}, \texttt{./reaction -snes\_mf}.  Does the comparison change if you add \texttt{-da\_refine 5}?  Explain as much as you can.  (\emph{Hints}: Look at \texttt{SNESSolve} events for the total time spent in the Newton method, which is $\approx$100\% for these cases.  Within that event, event \texttt{SNESLineSearch} is where function and Jacobian evaluations happen, while the linear solve is the \texttt{KSPSolve} event.  Within the \texttt{SNESLineSearch} event, look at \texttt{SNESFunctionEval} and \texttt{SNESJacobianEval} events.)

\item \label{exer:nl:symmetrizeJ}  In Chapter \ref{chap:st} we gave some advice which we have not followed in the current Chapter, namely that one should implement Dirichlet conditions in a manner that preserves the symmetry of the matrix in question, which is the Jacobian here.  Modify \texttt{reaction.c} to preserve symmetry, noting that this requires modifying both the $\bF$ and $J$ evaluation routines.  Does this modification improve performance when using the CG iteration?  Does it change our conclusions about un-preconditioned JFNK?

\item The PDE problem in \texttt{reaction.c} has parameter $\rho$ which can be adjusted via \texttt{PetscOptionsReal()}.  Modify \texttt{reaction.c} to a new code ``\texttt{optreact.c}'' and add an option \texttt{-optr\_rho} to read $\rho$ at runtime.  (The prefix \texttt{optr\_} is added when you call \texttt{PetscOptionsBegin()}.)  By sampling, for what range of positive \texttt{X} does the run
\begin{cline}
$ ./optreact -snes_monitor -da_refine 4 -optr_rho X
\end{cline}
%$
converge?  Explain why this case, with this way of choosing the boundary conditions and initial iterate, in contrast to the next Exercise, is insensitive to the value of the major parameter $\rho$.
% because this option is used in bratu.c below, for simplicity of maintenance, the result optreact.c is NOT saved in c/ch4/solns/

\item \label{exer:nl:bratu} Modify \texttt{reaction.c} to create ``\texttt{bratu.c}'' which solves
\begin{equation}
    - u'' - \lambda e^u = 0 \label{eq:nl:bratuoned}
\end{equation}
% R(u) = lambda e^u increasing
with Dirichlet boundary conditions $u(0)=u(1)=0$.  Make $\lambda$ an option-adjustable real parameter---see the previous Exercise---and use the simplest function which satisfies the boundary conditions as an initial iterate.  Confirm by using a fine grid, and option \texttt{-snes\_converged\_reason} plus other options as needed, that around $\lambda=3.513$ there is a transition to non-convergence of the Newton iteration.  This \emph{critical value} is intrinsic to the nonlinear PDE problem and not a result of a Newton failure \citep{Doedeletal1991}.

\item To the author's knowledge an exact solution of \eqref{eq:nl:bratuoned} cannot be written in elementary functions for $\lambda>0$.  Because we started from \texttt{reaction.c}, writing the code in the last exercise is doable.  However, one should not walk over high wires too often without a net, and having an exact solution to check correctness is such a ``net.''  Fortunately, we can ``manufacture'' \citep{Wesseling2001} an exact solution to a generalized form of \eqref{eq:nl:bratuoned}, namely to equation \eqref{eq:nl:diffusionreaction} in the case where $R(u)=-\lambda e^u$.  For example we can choose $u(x) = \sin(\pi x)$ as the exact solution of \eqref{eq:nl:diffusionreaction}, and then determine $f(x)$ from the equation itself.  Here $u(0)=u(1)=0$ are the boundary conditions; they are satisfied by the exact solution.

The exercise itself: Add options and such a manufactured exact solution to the code \texttt{bratu.c} in the previous Exercise.  The new code will both solve the previous Exercise and be verified using the new exact solution, according to a new boolean member \texttt{manufactured} of \texttt{AppCtx}: when it is false then $f(x)=0$ and the code solves the previous Exercise, but if it is true then we report a numerical error based on the manufactured solution.  Implement option \texttt{-bratu\_manu} to set this boolean.  Observe that $f(x)\ne 0$ does not change the Jacobian.  Confirm that in the manufactured-solution case we have both quadratic convergence of the Newton iteration on a fixed grid \emph{and} $O(h^2)$ convergence of the numerical error under grid refinement.

\item As long as the number of MPI processes does not exceed the number of grid points, \texttt{reaction.c} will solve correctly, though perhaps not efficiently, in parallel.  Confirm this by running
\begin{cline}
$ mpiexec -n N ./reaction -snes_converged_reason -da_refine M
\end{cline}
%$
with a few values of \texttt{N} and \texttt{M}.  However, \texttt{ecjacobian.c} does not run correctly even on two MPI processes.  Modify so that it does.

% from petsc-users by Matt:  What happens if you use LU? When debugging a nonlinear solver always use a direct solver for the linear solver (and make sure the direct solver actually worked).

\end{enumerate}


\chapter{Example: $p$-Laplacian equation}
\label{chap:of}
\renewcommand{\CODELOC}{ch5/}

In the last three Chapters we have introduced six important \PETSc objects for solving PDEs:
\begin{itemize}
\item[\quad Chapter \ref{chap:ls}:] \pVec, \pMat, \pKSP, \pPC
\item[\quad Chapter \ref{chap:st}:] \pDM\sidenote{In the \pDMDA case only.}
\item[\quad Chapter \ref{chap:nl}:] \pSNES
\end{itemize}
Each example in the rest of the book will use \emph{all} of these object types.

In particular, from now on we will even solve \emph{linear} PDEs using a \pSNES object and Newton iteration.  Linear problems are now merely the case where we know, in advance, that one Newton iteration suffices.  Always using \pSNES gives us uniform code structure and much more flexibility when it comes to changing the PDE problem.

Though we continue with new ideas, in the current Chapter we take a break from using new \PETSc objects.  Instead we first introduce PDEs which arise from minimization in a function space, and then we introduce a structured-grid finite element method (FEM).  Our one example is from an important class of nonlinear PDEs, namely Poisson-like problems with solution-dependent diffusivity.


\section{$p$-Laplacian equation as minimization}

Let $\Omega$ be a domain (connected open subset) in $\RR^2$ or $\RR^3$ with well-behaved boundary.\sidenote{A Lipschitz boundary will suffice in theory.  In practice we use polygonal domains, and merely a rectangle in the current Chapter.}  Consider this nonlinear functional for $p \ge 1$,
\begin{equation}
    I[u] = \int_\Omega \frac{1}{p} |\grad u|^p - fu.  \label{eq:of:functional}
\end{equation}
This functional is well-defined on the Sobolev space \citep{Evans2010} of integrable functions which are defined on $\Omega$ and have integrable gradient,
\begin{equation}
    W^{1,p}(\Omega) = \left\{w \,:\, \int_\Omega |w|^p < \infty \,\, \& \, \int_\Omega |\grad w|^p < \infty\right\}, \label{eq:of:sobolevdefn}
\end{equation}
which is a Banach space with norm $\|w\|_{W^{1,p}} = \left(\int_\Omega |w|^p + \int_\Omega |\grad w|^p\right)^{1/p}$.  We will assume that the function $f$ in \eqref{eq:of:functional} is at least integrable, in the sense that $f\in L^q(\Omega)$ for $q$ dual to $p$ (i.e.~ $1/p+1/q=1$), but little is lost if we just consider continuous $f\in C(\Omega)$.

\begin{marginfigure}
\includegraphics[width=1.2\textwidth]{figs/minsurf} % generated by figs/minsurf.tex
\medskip
\caption{The functional $I[u]$ is analogous to the convex surface $z = \tfrac{1}{4}(x^4 + y^4) - 2x + 2y$ shown here, but with input from the $\infty$-dimensional space $W_g^{1,p}(\Omega)$ instead of the plane $\RR^2$.}
\label{fig:of:cartoonfunctional}
\end{marginfigure}

The reader may visualize $I[u]$ in \eqref{eq:of:functional} as drawn in cartoon form in Figure \ref{fig:of:cartoonfunctional}.  As suggested by our cartoon, not only is it well-defined, but it also has a unique minimum if we add boundary conditions.

We add Dirichlet conditions, as in the Poisson problem in Chapter \ref{chap:st}, by choosing a real-valued function $g$ defined along $\partial \Omega$ and define a subset (affine subspace) of $W^{1,p}(\Omega)$, namely
\begin{equation}
    W_g^{1,p}(\Omega) = \left\{w \,:\, w \in W^{1,p}(\Omega) \,\, \& \,\, w\big|_{\partial \Omega} = g\right\}.  \label{eq:of:affinedirichlet}
\end{equation}
For this to make sense we assume $g \in L^p(\partial \Omega)$ and note that ``$w\big|_{\partial \Omega} = g$'' has a precise ``trace operator'' meaning \citep{Evans2010}.  These considerations also define the vector subspace $W_0^{1,p}(\Omega) \subset W^{1,p}(\Omega)$ in the $g=0$ case; we will use this subspace for ``test functions'' below.

With such Dirichlet boundary conditions, then first of all $I[u]$ is \emph{coercive} in the sense that if the input function from $W_g^{1,p}(\Omega)$ is large in norm then the output is large:
\begin{equation}
\lim_{\|u\|_{W^{1,p}} \to +\infty} I[u] = +\infty.   \label{eq:of:coercivity}
\end{equation}
Secondly it is continuous enough to have a minimum on compact sets, namely it is \emph{weakly lower semi-continuous}, which means, by definition, that
\begin{equation}
\lim_{u\rightharpoonup v} I[u] \ge I[v],  \label{eq:of:lowersemicont}
\end{equation}
where the limit is in the weak topology on $W^{1,p}(\Omega)$ \citep{Evans2010,KinderlehrerStampacchia1980}.  Third it is \emph{convex}, meaning
\begin{equation}
I[\lambda u + (1-\lambda) v] \le \lambda I[u] + (1-\lambda) I[v]    \label{eq:of:convexity}
\end{equation}
if $u,v\in W^{1,p}(\Omega)$ and $0 \le \lambda \le 1$.  \citet{KinderlehrerStampacchia1980} show that the three properties \eqref{eq:of:coercivity}, \eqref{eq:of:lowersemicont}, \eqref{eq:of:convexity} imply that the problem
\begin{equation}
\min_{u \in W_g^{1,p}(\Omega)} I[u] \label{eq:of:plapmin}
\end{equation}
has a unique solution.

On the other hand, the solution of minimization problem \eqref{eq:of:plapmin} is also the solution to a nonlinear PDE.  Indeed, if $p>1$ then the functional $I[u]$ is smooth enough to have a gradient (derivative).  The statement that the gradient is zero at the minimum has several names, among them \emph{variational equation}, \emph{Euler-Lagrange equation}, and the \emph{weak form} of the PDE.  These multiple names hint that one sees the following calculation in many parts of applied mathematics.

Assume $\eps$ is a real number and $u,v \in W^{1,p}(\Omega)$.  Then
\begin{align*}
I[u+\eps v] - I[u] &= \int_\Omega \frac{1}{p} |\grad u + \eps \grad v|^p + \frac{1}{p} |\grad u|^p - \eps f v \\
   &= \eps \left(\int_\Omega |\grad u|^{p-2} \grad u \cdot \grad v - f v\right) + O(\eps^2),
\end{align*}
and so the gradient exists,
\begin{equation}
\grad I[u](v) = \lim_{\eps\to 0} \frac{I[u+\eps v] - I[u]}{\eps} = \int_\Omega |\grad u|^{p-2} \grad u \cdot \grad v - f v. \label{eq:of:plapfunctionalderivative}
\end{equation}
That is, for each $u \in W^{1,p}(\Omega)$ formula \eqref{eq:of:plapfunctionalderivative} defines a map
   $$\grad I[u] : W^{1,p}(\Omega) \to \RR.$$

If $u \in W_g^{1,p}(\Omega)$ and $v\in W_0^{1,p}(\Omega)$ then $u+\eps v\in W_g^{1,p}(\Omega)$.  Thus if $u \in W_g^{1,p}(\Omega)$ solves \eqref{eq:of:plapmin} then the above derivative calculation shows $\grad I[u](v)=0$ or
\begin{equation}
\int_\Omega |\grad u|^{p-2} \grad u \cdot \grad v - f v = 0 \label{eq:of:plapweakform}
\end{equation}
for all $v\in W_0^{1,p}(\Omega)$.  From now on we refer to \eqref{eq:of:plapweakform} as the \emph{weak form} of the $p$-\emph{Laplacian} equation.

Whether or not this is a surprise to the reader, we have nearly returned to the Poisson problem address in Chapter \ref{chap:st}.  In particular, if the solution $u \in W_g^{1,p}(\Omega)$ to problem \eqref{eq:of:plapmin} is actually smooth-enough to have continuous second derivatives, i.e.~$u \in C^2(\Omega)$, then we can derive the \emph{strong form} corresponding to equation \eqref{eq:of:plapweakform}.

We show this by an integration-by-parts \citep{Evans2010} on equation \eqref{eq:of:plapweakform}, which gives
    $$-\int_\Omega \Div\left(|\grad u|^{p-2} \grad u\right) v - \int_\Omega f v + \int_{\partial \Omega} v |\grad u|^{p-2} \grad u \cdot \bn = 0$$
The boundary integral is zero if $v\in W_0^{1,p}(\Omega)$.  It follows that
\begin{equation}
- \Div\left(|\grad u|^{p-2} \grad u\right) = f
\label{eq:of:plapstrongform}
\end{equation}
This is the strong form.  It is the traditional form of the $p$-Laplacian equation, and it reduces to equation \eqref{poissonsquare} if $p=2$.

Before proceeding to a numerical solution, the main idea of this extended diversion into theory is fairly simple:
\begin{quote}
Minimization problem \eqref{eq:of:plapmin} for functional $I[u]$ in \eqref{eq:of:functional} is equivalent to the weak form equation \eqref{eq:of:plapweakform}, and it becomes the strong form equation \eqref{eq:of:plapstrongform} in cases where the solution $u$ is smooth.  Thus the $p$-Laplacian equation \eqref{eq:of:plapstrongform}, which is a nonlinear generalization of the Poisson equation, arises from a minimization problem.
\end{quote}


\section{Structured $Q^1$ finite elements}

FIXME so what?  well, it turns out \PETSc is perfectly-happy to let us think of the $p$-Laplacian equation as a minimization problem.  all we need to implement is \eqref{eq:of:functional}.  we do so, and we take quite literally the fundamental idea that the unknowns represent a function in a function space and we need only implement a real-valued functional on such functions.

FIXME show grid with $(i,j)$ indexing on nodes, of thus of elements $\square_{ij}$

\begin{align}
I[u] &= \int_\Omega \frac{1}{p} |\grad u|^p - fu  \notag \\
     &= \sum_{i=0}^{M_x-1} \sum_{j=0}^{M_y-1} \int_{\square_{ij}} \frac{1}{p} |\grad u|^p - fu  \label{eq:of:sumoverelements}
\end{align}

FIXME show bilinear functions: $\ell=0,1,2,3$ indexing on corners  and their basis $\chi_\ell(\xi,\eta)$ on reference element $\square_\ast = [-1,1]\times[-1,1]$

\begin{equation}
\chi_\ell(\xi,\eta) = \frac{1}{4} \left(1 + \xi_\ell \xi\right) \left(1 + \eta_\ell \eta\right)   \label{eq:of:chidefn}
\end{equation}

\begin{equation}
x(\xi,\eta) = \sum_{\ell=0}^3 x_\ell \chi_\ell(\xi,\eta), \quad y(\xi,\eta) = \sum_{\ell=0}^3 y_\ell \chi_\ell(\xi,\eta)  \label{eq:of:referencemap}
\end{equation}

\begin{figure}
\begin{tikzpicture}[scale=3.0]
% (x,y) elements
  \draw[->,very thin] (-0.2,0.0) -- (1.2,0.0) node[below] {$x$};
  \draw[->,very thin] (0.0,-0.2) -- (0.0,1.2) node[left] {$y$};
  \draw[line width=1.0pt] (0.0,0.0) -- (0.0,1.0) -- (1.0,1.0) -- (1.0,0.0) -- cycle;
  \pgfmathsetmacro\fourth{1.0/4.0}
  \pgfmathsetmacro\third{1.0/3.0}
  \pgfmathsetmacro\twothird{2.0/3.0}
  \draw[xstep=\fourth,ystep=\third,black,thin] (0.0,0.0) grid (1.0,1.0);
  % outline an element
  \draw[line width=1.5pt] (0.5,\third) -- (0.75,\third) -- (0.75,\twothird) -- (0.5,\twothird) -- cycle;
  \node at (0.64,0.5) {$\square_{ij}$};
  \draw[<->] (0.5,1.1) -- (0.75,1.1) node[above,yshift=1mm,xshift=-4mm] {$\Delta x$};
  \draw[<->] (1.1,\third) -- (1.1,\twothird) node[right,yshift=-5mm] {$\Delta y$};
  \node at (0.15,0.8) {\large $\Omega$};

% (xi,eta) reference element
% origin of axes at (2.5,0.0) and square has half-width 0.5
  \draw[->,very thin] (1.7,0.0) -- (3.3,0.0) node[below] {$\xi$};
  \draw[->,very thin] (2.5,-0.8) -- (2.5,0.8) node[left] {$\eta$};
  \draw[line width=1.5pt] (2.0,-0.5) -- (3.0,-0.5) -- (3.0,0.5) -- (2.0,0.5) -- cycle;
  \node at (3.1,-0.1) {$1$};
  \node at (1.9,-0.1) {$-1$};
  \node at (2.4,0.6) {$1$};
  \node at (2.4,-0.6) {$-1$};
  \node at (2.8,0.3) {\large $\square_\ast$};
  \filldraw (2.0,-0.5) circle (0.8pt) node[xshift=-6mm,yshift=-2mm] {$\ell=0$};
  \filldraw (3.0,-0.5) circle (0.8pt) node[xshift=6mm,yshift=-2mm] {$\ell=1$};
  \filldraw (3.0,0.5) circle (0.8pt) node[xshift=6mm,yshift=2mm] {$\ell=2$};
  \filldraw (2.0,0.5) circle (0.8pt) node[xshift=-6mm,yshift=2mm] {$\ell=3$};

% arc
  \draw[->,line width=1.0pt] (2.3,0.55) .. controls (1.8,1.0) and (0.7,1.0) .. (0.6,0.72);
\end{tikzpicture}
\caption{Each element $\square_{ij}$ in the domain $\Omega$, a square in this case, is the image under a map $\left(x(\xi,\eta),y(\xi,\eta)\right)$ of a reference element $\square_\ast = [-1,1]\times[-1,1]$.  The corners of the reference element, and thus of $\square_{ij}$ as well, are locally-numbered $\ell=0,1,2,3$.}
\label{fig:q1grid}
\end{figure}

FIXME hat functions

\begin{equation}
  \psi_{i,j}(x_r,y_s) = \delta_{ir} \delta_{js}  \label{eq:of:phinodewise}
\end{equation}

\begin{equation}
  \psi_\ell(\xi,\eta) = \psi_\ell(x(\xi,\eta),y(\xi,\eta)) = \chi_\ell(\xi,\eta)  \label{eq:of:phionref}
\end{equation}

\begin{equation}
  (\grad \psi_\ell)(\xi,\eta) = \left<FIXME\right>  \label{eq:of:gradphionref}
\end{equation}

\begin{marginfigure}
\begin{tikzpicture}[scale=0.5]

  % strong grid around elements
  \draw[thick] (0,0) -- (8,0);
  \draw[thick] (2,2) -- (10,2);
  \draw[thick] (4,4) -- (12,4);
  \draw[thick] (0,0) -- (4,4);
  \draw[thick] (4,0) -- (8,4);
  \draw[thick] (8,0) -- (12,4);

  \def\ytop{7};

  % tent lines
  \draw[gray,thick] (6,\ytop) -- (4,0);
  \draw[gray,thick] (6,\ytop) -- (2,2);
  \draw[gray,thick] (6,\ytop) -- (10,2);
  \draw[gray,thick] (6,\ytop) -- (8,4);

  \def\dx{(10.0-6.0)/6};
  \def\dy{(2.0-\ytop)/6};
  \foreach \jj in {1,...,5}
  {
       \draw[gray,very thin] ({6+\jj*\dx},{\ytop+\jj*\dy}) -- ({4+(4/6)*\jj},0.0);
  }

  \def\dx{(4.0-6.0)/6};
  \def\dy{(0.0-\ytop)/6};
  \foreach \jj in {1,...,5}
  {
       \draw[gray,very thin] ({6+\jj*\dx},{\ytop+\jj*\dy}) -- ({10-(2/6)*\jj},{2-(2/6)*\jj});
  }

  \def\dx{(2.0-6.0)/6};
  \def\dy{(2.0-\ytop)/6};
  \foreach \jj in {1,...,5}
  {
       \draw[gray,thin] ({6+\jj*\dx},{\ytop+\jj*\dy}) -- ({4-(4/6)*\jj},0.0);
  }

  \def\dx{(4.0-6.0)/6};
  \def\dy{(0.0-\ytop)/6};
  \foreach \jj in {1,...,5}
  {
       \draw[gray,thin] ({6+\jj*\dx},{\ytop+\jj*\dy}) -- ({2-(2/6)*\jj},{2-(2/6)*\jj});
  }

  \def\dx{(10.0-6.0)/6};
  \def\dy{(2.0-\ytop)/6};
  \foreach \jj in {1,...,5}
  {
       \draw[gray,thin] ({6+\jj*\dx},{\ytop+\jj*\dy}) -- ({8+(4/6)*\jj},4.0);
  }

  \def\dx{(8.0-6.0)/6};
  \def\dy{(4.0-\ytop)/6};
  \foreach \jj in {1,...,5}
  {
       \draw[gray,thin] ({6+\jj*\dx},{\ytop+\jj*\dy}) -- ({10+(2/6)*\jj},{2+(2/6)*\jj});
  }

  \def\dx{(2.0-6.0)/3};
  \def\dy{(2.0-\ytop)/3};
  \foreach \jj in {1,...,2}  % reduce clutter
  {
       \draw[gray,thin] ({6+\jj*\dx},{\ytop+\jj*\dy}) -- ({8-(4/3)*\jj},4.0);
  }

  \def\dx{(8.0-6.0)/3};
  \def\dy{(4.0-\ytop)/3};
  \foreach \jj in {1,...,2}
  {
       \draw[gray,thin] ({6+\jj*\dx},{\ytop+\jj*\dy}) -- ({2+(2/3)*\jj},{2+(2/3)*\jj});
  }

  % nodes in base plane
  \filldraw (0,0) circle (4pt);
  \filldraw (4,0) circle (4pt);
  \filldraw (8,0) circle (4pt);
  \filldraw (2,2) circle (4pt);
  %\filldraw (6,2) circle (4pt);   % (x_j,y_k) is at (6,2)
  \filldraw (10,2) circle (4pt);
  \filldraw (4,4) circle (4pt);
  \filldraw (8,4) circle (4pt);
  \filldraw (12,4) circle (4pt);

  % node at tent top
  \filldraw (6,\ytop) circle (4pt);

  % annotate
  \draw (10,\ytop+1.0) node {$\psi_{p,q}(x_p,y_q)=1$};
  \draw[-latex] (7.2,\ytop+0.8) -- (6.2,\ytop+0.2);
  \draw (8.5,-1.2) node {$\psi_{p,q}(x_r,y_s)=0$ (other nodes)};
  \draw[-latex] (10,-0.3) -- (10,1.8);

  % label center point
  \draw (4,-0.6) node {$x_p$};
  \draw (1.2,2) node {$y_q$};

\end{tikzpicture}

\caption{FIXME}
\label{fig:q1hat}
\end{marginfigure}


\section{Implementation with objective only}

FIXME code uses \texttt{SNESSetObjective()} only, though also \texttt{SNESSetFunction()}; no hand-made Jacobian at all

FIXME try NCG

\cinputpart{plap.c}{\CODELOC}{FIXME I}{I}{//STARTCONFIGURE}{//ENDCONFIGURE}{code:plapI}

\cinputpart{plap.c}{\CODELOC}{FIXME II}{II}{//STARTEXACTF}{//ENDEXACTF}{code:plapII}

\cinputpart{plap.c}{\CODELOC}{FIXME III}{III}{//STARTOBJECTIVE}{//ENDOBJECTIVE}{code:plapIII}

\cinputpart{plap.c}{\CODELOC}{FIXME IV}{IV}{//STARTMAIN}{//ENDMAIN}{code:plapIV}


\section{Residual function $=$ gradient}

\cinputpart{plap.c}{\CODELOC}{FIXME V}{V}{//STARTFUNCTION}{//ENDFUNCTION}{code:plapV}


\section{Exercises}

\renewcommand{\labelenumi}{\arabic{chapter}.\arabic{enumi}\quad}
\renewcommand{\labelenumii}{(\alph{enumii})}
\begin{enumerate}
\item Prove coercivity \eqref{eq:of:coercivity} and convexity \eqref{eq:of:convexity} of functional $I[u]$ defined in \eqref{eq:of:functional}.  The finite dimensional facts that FIXME for $\xi,\eta\in\RR^n$ will be useful.
\item FIXME
\end{enumerate}

\chapter{Example: Advection-diffusion equation}
\label{chap:ad}
\renewcommand{\CODELOC}{ch6/}

As in the previous Chapter, this is a big example.


\section{Linear advection-diffusion equations}

FIXME

\section{3D problem with manufactured solution}

\section{Implementation}

\cinputpart{ad3.c}{\CODELOC}{Create a 3D \pDMDA for a star stencil and a combination of Dirichlet and periodic boundary conditions.}{I}{//STARTDMDA}{//ENDDMDA}{code:ad3I}

\cinputpart{ad3.c}{\CODELOC}{Declare context and wind-vector \texttt{struct}s.}{II}{//STARTSETUP}{//ENDSETUP}{code:ad3II}

\cinputpart{ad3.c}{\CODELOC}{The main part of the implementation is to define the linear equations we are solving, for use as a residual by \pSNES.}{II}{//START}{//END}{code:ad3III}

\section{Convergence and upwinding}

\section{Double-glazing problem}

\section{Preconditioning for advection}



\chapter{Preconditioners}
\label{chap:pr}
\renewcommand{\CODELOC}{ch7/}

When we combine \PETSc \pDMDA and \pSNES objects we get the abilily to do \emph{geometric multigrid} \citep{Briggsetal2000,Trottenbergetal2001} and \emph{domain decomposition} \citep{Smithetal1996,Doleanetal2015} methods.  We will show how these methods act as preconditioners on the linear systems which arise in Newton's method, so this section is about \emph{preconditioned Newton-Krylov} methods.  To get started we introduce the basic ideas of multigrid.

FIXME cite survey \citep{Wathen2015}

\section{Multigrid basics}

FIXME: setup: suppose we are solving 1D Poisson equation on $\RR$, i.e.
    $$- u_{xx} = f(x)$$
with scheme
    $$- \frac{u_{j+1} - 2 u_j + u_{j-1}}{h^2} = f_j,$$
where $f_j = f(x_j)$, or equivalently
    $$- u_{j-1} + 2 u_j - u_{j+1} = h^2 f_j.$$

FIXME: idea (1) in 1D, on unbounded grid with spacing $h$, restriction to every other point takes lowish frequency wave to higher frequency wave on (still unbounded) grid with spacing $2h$

FIXME: idea (2) pointwise iteration version of Poisson FD scheme, e.g.~Jacobi
   $$u_j^{(m+1)} = \frac{1}{2} \left(u_{j-1}^{(m)} + u_{j+1}^{(m)} + h^2 f_j\right) $$
or Gauss-Seidel (where we update in increasing order on $j$)
   $$u_j^{(m+1)} = \frac{1}{2} \left(u_{j-1}^{(m+1)} + u_{j+1}^{(m)} + h^2 f_j\right) $$
will serve as low-pass filter on a given grid

FIXME: idea (3) given $u^{(0)}$ on fine $h$ grid, it is the residual FIXME

FIXME: for more, read \citep{Briggsetal2000} which is at same level as this book


\section{A multigrid-capable Poisson problem code}

The code \texttt{c/ch7/fish2.c} (not shown) solves the same 2D Poisson problem
    $$-\grad^2 u = f,$$
with homogeneous boundary conditions on the square $\Omega=[0,1]\times[0,1]$, as does \texttt{c/ch3/poisson.c} in Chapter \ref{chap:st}.  In fact there is much overlap between the two codes, with similar C functions which set the exact solution, set the \pVec corresponding to $f$, and set up a \pMat $A$ corresponding to the finite-difference discretized operator $-\grad^2$.

This time, however, we adopt the approach common to the second half of this book, which is to say we set up a \PETSc \pSNES object.  It does Newton's method, even though the problem is linear.  In \texttt{c/ch7/fish2.c} there is a residual-evaluation function, corresponding to the discretized form of the PDE; heuristically it computes $\bF(\bu) = -\grad^2 u - f$.  The Jacobian of this residual function gives precisely the same matrix as was constructed by \texttt{c/ch3/poisson.c}.

The two codes compute the same solution as long as we tighten the Krylov solver tolerance:\sidenote{Without this tightening \texttt{fish2.c} computes the more accurate solution because the \pSNES asks for iteration so as to ``clean up'' the small residual; note the default tolerances for \texttt{-snes\_rtol} of $10^{-8}$ and for \texttt{-ksp\_rtol} of $10^{-5}$.}
\begin{cline}
$ cd c/ch7/
$ make fish2
...
$ ./fish2 -ksp_rtol 1.0e-10
on 9 x 9 grid:  error |u-uexact|_inf = 0.000763883
$ cd ../ch3/
$ make poisson
...
$ ./poisson -ksp_rtol 1.0e-10
on 9 x 9 grid:  error |u-uexact|_inf = 0.000763883
\end{cline}

FIXME

\section{Grid sequencing on a nonlinear problem}

FIXME

\section{Domain decomposition preconditioners}

FIXME

% question by Gideon Simpson:
% ...
%> Nope, I'm not doing any grid sequencing. Clearly that makes a lot of
%> sense, to solve on a spatially coarse mesh for the field variables,
%> interpolate onto the finer mesh, and then solve again.  I'm not entirely
%> clear on the practical implementation
%
% SNES should do this automatically using -snes_grid_sequence <k>.  If this
% does not work, complain. Loudly.
%   Matt
% ... later in correspondence ...
%> 4.  When I do SNESSolve(snes, NULL, U) with grid sequencing, U is not
%> the solution on the fine mesh.  But what is it?  Is it still the starting
%> guess, or is it the solution on the coarse mesh?
%
% It will contain the solution on the coarse mesh. After SNESSolve() call
% SNESGetSolution() and it will give back the solution on the fine mesh.
%  Barry

\section{Composing preconditioners}

FIXME:  block Jacobi and GS/SOR?

FIXME: somewhere describe Nachtigal et al 1992 result that KSP are strictly inequivalent

\begin{comment}
> Hi all,
>
> Is weighted Jacobi available as a preconditioner ? I can't find it in the
> list of preconditioners. If not, what is the rationale between this choice
> ? It is pretty straightforward to code, so if it is not available I can do
> it without problem I guess, but I am just wondering. In the matrix-free
> case where SOR is not available by default, it may be better than pure
> Jacobi, and much easier to parallelize than SOR.
>
>   Timothee Nicolas

I believe what you are looking for is defined by the following options
  -ksp_type richardson
  -ksp_richardson_scale <value>
  -pc_type jacobi

Thanks,
  Dave May
\end{comment}

\begin{comment}
DAVE MAY petsc-users 2/9/2016:
...
To be sure your code is working correctly, test it using a preconditioner
which isn't dependent on the partitioning of the matrix. I would use these
options:
  -pc_type jacobi
  -ksp_monitor_true_residual

The last option is useful as it will report both the preconditioned
residual and the true residual. If the operator is singular, or close to
singular, gmres-ILU or gmres-BJacobi-ILU can report a preconditioned
residual which is small, but is orders of magnitude different from the true
residual.

Thanks,
  Dave
\end{comment}

%EXERCISE: make fish2.c more efficient by NOT doing "divide by hx^2 then mult by vol"  and instead "mult by hy*hz/hx"; measure performance difference; note this is already done by fish3.c


\chapter{Parallel scaling and performance}
\label{chap:sc}
\renewcommand{\CODELOC}{ch8/}
\stubinput{scaling.tex}{5}

\chapter{Example: An unstructured finite element method}
\label{chap:un}
\renewcommand{\CODELOC}{ch9/}
%\stubinput{unstructured.tex}{40}

The cliched Poisson problem can be exploited some more.  It gives us the opportunity to use \PETSc for important tasks we have not yet seen, including reading an unstructured mesh into \PETSc \pVecs, symmetric implementation of boundary conditions, and explicit preallocation of a \pMat in parallel.

\section{Example: The Poisson problem}

Let $\Omega \subset \RR^d$ be a bounded (open) region.  Suppose its boundary $\partial\Omega$ is well-behaved, for instance that it is Lipschitz-continuous \citep[section 1.2]{Ciarlet2002} or even polygonal.  Suppose $\partial\Omega$ is decomposed into (measurable) disjoint subsets $\partial_D \Omega$ and $\partial_N \Omega$ whose union is the entire boundary $\partial \Omega$.  The \emph{Poisson problem}, in strong form and including nonhomogeneous Dirichlet and Neumann boundary conditions, is\marginnote{%
\begin{tikzpicture}[scale=0.7]
%\draw[gray,very thin] (-2,-6) grid (8,3);
\draw[line width=2pt] (0,0) .. controls (0,3) and (5,3) .. node[sloped,above] {$u=g$} node[sloped,below] {$\partial_D\Omega$} (7,0);
\draw[line width=0.75pt] (7,0) .. controls (9,-2) and (-1,-7) .. node[sloped,above] {$\partial_N\Omega$} node[sloped,below] {$\partial u/\partial n = \gamma$} (-1,-5);
\draw[line width=0.75pt] (-1,-5) .. controls (-1,-4) and (2,-5) .. (2,-3);
\draw[line width=0.75pt] (2,-3) .. controls (2,-1) and (0,-3) .. (0,0);
\draw[->,line width=1.0pt] (2,-3) -- (1.2,-3) node[below] {$\bn$}; % normal vector
\draw (4,-1) node {$- \grad^2 u = f$ in $\Omega$};
\end{tikzpicture}}
\begin{align}
- \grad^2 u &= f \quad \text{ on } \Omega, \label{poissonstrong} \\
u &= g \quad \text{ on } \partial_D \Omega, \notag \\
\frac{\partial u}{\partial n} &= \gamma \quad \text{ on } \partial_N \Omega \notag
\end{align}
where $\bn$ is the outward unit normal on $\partial \Omega$ and $\partial u/\partial n = \bn \cdot \grad u$.  The data of problem \eqref{poissonstrong}, besides the region $\Omega$ and its boundary, includes a \emph{source term} $f\in L^2(\Omega)$, \emph{Dirichlet data} $g\in L^2(\partial_N \Omega)$, and \emph{Neumann data} $\gamma\in L^2(\partial_N \Omega)$.

As \eqref{poissonstrong} is stated there may be no solution where ``$\grad^2 u$'' makes sense as a continuous function, even for polygonal regions, continuous boundary values, and continuous source functions.  In particular, there may be no $u\in C^2(\Omega)$ which is continuous up to the boundary (i.e.~$u\in C(\bar\Omega)$) and so that $\grad^2 u = f$.  There is, however, a solution if we change to a \emph{weak formulation}.\sidenote{A proof of this well-posedness claim is in \citet{Ciarlet2002} and in \citet{Evans2010}.  These are technicalities for us, however, as our goal is computational performance in cases where the Poisson problem is mathematically well-behaved and easily approximated.}  Furthermore, if $\partial_D \Omega$ has positive size, in an appropriate sense \citep[Theorem 1.2.1]{Ciarlet2002}, then the solution of the weak formulation is unique.  We will state the weak formulation after glossing the definitions of the needed function spaces.

Recalling $L^2(\Omega)$ is the space of all square-integrable real functions on $\Omega$, define
    $$H^1(\Omega) = \{u\in L^2(\Omega) \big| \grad u \text{ exists a.e.~and } \grad u \in L^2(\Omega)\},$$
which is a Sobolev space \citep{Evans2010}.  This space has two subsets we use, namely functions with value $g$ on $\partial_D \Omega$ and those with value $0$ on $\partial_D \Omega$, respectively, which we denote $H_g^1(\Omega)$ and $H_0^1(\Omega)$.  Note that $H_0^1(\Omega)$ is a linear subspace of $H^1(\Omega)$.  While $H_g^1(\Omega)$ is generally not a subspace (e.g.~because the zero function is not in it), it is an affine subspace, and we refer to both $H_g^1(\Omega)$ and $H_0^1(\Omega)$ as ``subspaces''.

To get to the weak formulation of the Poisson problem we suppose we already have a classical solution $u$ of \eqref{poissonstrong}.  Then we choose any $v\in H_0^1(\Omega)$, multiply the first equation in \eqref{poissonstrong} by $v$, and integrate by parts:
\begin{equation*}
\int_\Omega \grad u \cdot \grad v - \int_{\partial\Omega} \frac{\partial u}{\partial n} v = \int_\Omega f v.
\end{equation*}
Next\marginnote{%
{\color{red}Main ideas} of strong and weak formulations:\begin{itemize}
\item If $u \in H_g^1(\Omega)$ solves the strong form \eqref{poissonstrong} then it solves \eqref{poissonweak} also.
\item If $u \in H_g^1(\Omega)$ solves the weak form \eqref{poissonweak} then we accept it, by definition, as a solution of the Poisson problem.\end{itemize}} 
we use the other data, namely that $v=0$ on $\partial_D\Omega$ and that there is Neumann data $\gamma$ on $\partial_N\Omega$:
\begin{equation}
\int_\Omega \grad u \cdot \grad v = \int_\Omega f v + \int_{\partial_N\Omega} \gamma v\quad \text{ for any } v\in H_0^1(\Omega). \label{poissonweak}
\end{equation}

Equation \eqref{poissonweak} is the weak formulation of the Poisson problem.  Any $u \in H_g^1(\Omega)$ satisfying \eqref{poissonweak} is called a \emph{weak solution}.  A key observation is that $u$ itself incorporates the Dirichlet boundary condition, because it lives in $H_g^1(\Omega)$, while both the Neumann boundary data $\gamma$ and the source function $f$ appear in equation \eqref{poissonweak}.


\section{A finite element method (FEM) for the Poisson problem in the plane}

An FEM for the Poisson problem comes from requiring the weak formulation \eqref{poissonweak} to be true for $u$ in a much smaller, indeed finite-dimensional, subspace of $H_g^1(\Omega)$, and for test functions $v$ ranging over a finite-dimensional subspace of $H_0^1(\Omega)$.  In the ``Galerkin'' method here, these subspaces will be essentially the same.  We will build these subspaces, in the current example, using an unstructured triangulation on $\Omega\subset \RR^2$; from now on in this Chapter we restrict to $d=2$ dimensions.

Furthermore, to make our finite-dimensional spaces true subspaces of $H^1(\Omega)$---to make our FEM \emph{conforming}---we require that $\Omega$ be polygonal, with $\partial\Omega$ a closed polygon.  Segments of $\partial\Omega$ must have positive length, and be either entirely in $\partial_D\Omega$ or entirely in $\partial_N\Omega$.  We also assume $\partial_D\Omega$ is a closed set so that, at vertices of $\partial \Omega$ where the Dirichlet boundary and Neumann boundary meet, the vertex is Dirichlet.

By definition, a \emph{triangulation} is a finite set of non-overlapping, non-empty open triangles $\triangle_k\subset \RR^2$ which tile $\Omega$:
\begin{equation*}
\Th = \left\{\triangle_k \quad\Big|\quad \cup_k \overline{\triangle}_k = \overline{\Omega} \quad \text{ and} \quad \Omega_k \cap \Omega_l = \emptyset \text{ if } k\ne l\right\}.
\end{equation*}
We index the $K$ triangles in $\Th$ by $k=0,\dots,K-1$.  The $N$ vertices (nodes) in $\Th$ are indexed by $j=0,1,\dots,N-1$, with locations
\begin{equation*}
\bx_j = (x_j,y_j).
\end{equation*}
An example triangulation is shown in Figure \ref{fig:number-elements}.

\begin{marginfigure}
\input{samplepoly.1.tikz}
\caption{A triangulation $\Th$ with $K=22$ triangles (elements) numbered $k=0,1,\dots,K-1$ ({\color{red} red}) and $N=16$ nodes numbered $j=0,1,\dots,N-1$  ({\color{blue} blue}).  Nodes $\bx_0$, $\bx_1$, $\bx_2$, $\bx_3$ are in the Dirichlet boundary $\partial_D\Omega$.}
\label{fig:number-elements}
\end{marginfigure}

For now the reader can regard the subscript ``$h$'' in ``$\Th$'' as merely-traditional notation.  It denotes the typical or maximum size $h$ (e.g.~diameter) of the triangles, and it serves as a reminder that we want to approximate the solution in the limit $h\to 0$.  Also note that, in contrast to \citet{Elmanetal2005}, which we generally follow, and other references on the FEM or its implementation in languages like \Matlab, our indexing is zero-based.  This is so because we implement in C and we want to avoid any confusion when comparing text and codes.  Breaking long mathematical traditions, rows and columns of vectors and matrices will also have numbering starting with zero in this book.

We informally call the triangles \emph{elements}, though there is more to the definition of ``element.''  We are going to approximate the Poisson problem with $\Pone$ finite elements, which means that our finite-dimensional subspace contains only piecewise-linear functions which are linear on each triangle $\triangle_k$.

For each node $j$ there is a $\Pone$ basis function, or ``hat'' function, $\phi_j(x,y)$ which is linear on each triangle, continuous on all of $\overline{\Omega}$, and equal to one on only one node $j$:\marginnote{%
\begin{tikzpicture}[scale=0.9, z={(.707,.3)}]
    % (2,2,1) is top
    \draw[style=dashed] (0,0,0) -- (2,2,1); % to top from left
    \draw[style=dashed] (4,0,0) -- (2,2,1); %   ...  from front
    \draw[style=dashed] (4,0,3) -- (2,2,1); %   ...  from right
    \draw[color=gray, style=dashed] (0.3,0,4) -- (2,2,1); % from back
    \draw[color=gray, style=dashed] (2,0.4,1) -- (2,2,1); % from middle
    % draw base
    \draw (0,0,0) -- (4,0,0);
    \draw (4,0,0) -- (4,0,3);
    \draw[color=gray] (0,0,0) -- (0.3,0,4);
    \draw[color=gray] (0.3,0,4) -- (4,0,3);
    \draw[color=gray] (0,0,0) -- (2,0.4,1);
    \draw[color=gray] (2,0.4,1) -- (4,0,3);
    \draw[color=gray] (4,0,0) -- (2,0.4,1);
    \draw[color=gray] (2,0.4,1) -- (0.3,0,4);
    % draw \phi_j at nodes
    \filldraw (2,2,1) circle (1.25pt);
    \draw (2,2.5,1) node {$\phi_j(\bx_j)=1$};
    \draw (2,0.15,1) node {$\bx_j$};
    \filldraw (0,0,0) circle (1.25pt);
    \filldraw (4,0,0) circle (1.25pt);
    \draw (4,-0.5,0) node {$\phi_j(\bx_i)=0$};
    \draw (4.1,0.3,0) node {$\bx_i$};
    \filldraw (4,0,3) circle (1.25pt);
    \filldraw[color=gray] (0.3,0,4) circle (1.25pt);
\end{tikzpicture}
}%
\begin{equation*}
\phi_j(\bx_i) = \delta_{ij}.
\end{equation*}
The functions $\phi_j$ are in $H^1(\Omega)$, with piecewise-constant partial derivatives $\partial\phi_j/\partial x$ and $\partial\phi_j/\partial y$.  Also, the set $\{\phi_j\}_{j=0,\dots,N-1}$ is linearly-independent.  On each triangle, $\phi_j$ has three degrees of freedom, because on $\triangle_k$ there exist coefficients $A_k,B_k,C_k\in\RR$ so that
\begin{equation*}
\phi_j(\bx) = A_k + B_k x + C_k y \quad \text{ on } \triangle_k,
\end{equation*}
where $\bx = (x,y)$.

We can immediately use these basis functions to approximate the Dirichlet data $g$ and extend it to the region $\Omega$.  We will assume from now on that the Dirichlet boundary $\partial_D\Omega$ is closed, and index the $L$ nodes which are in the Dirichlet boundary by $\bx_{j_l} \in \partial_D\Omega$ for $l=0,\dots,L-1$.  (Figure \ref{fig:number-elements} shows an example with $L=4$ and $j_l=l$ for $l=0,1,2,3$, but any subset of boundary points can be Dirichlet as long as the index values $j_l$ are well-defined.)  Now we can define an extended interpolant $\hat g$ of $g$ as the function which has the correct value on the Dirichlet boundary nodes and which extends to all of $\Omega$ in a continuous and piecewise-linear way:
\begin{equation}
\hat g(\bx) = \sum_{l=0,\dots,L} g(\bx_{j_l}) \phi_{j_l}(\bx). \label{hatgdefine}
\end{equation}

By using $\hat g$ and the basis functions $\phi_j$, we can now describe three finite-dimensional subspaces of $H^1(\Omega)$:\sidenote{Traditionally, basis functions for $S_g^h$ are called the \emph{trial} functions, and basis functions for $S_0^h$ are called \emph{test} functions.  We will generally just use the labels ``$S_g^h$'' and ``$S_0^h$.''}
\begin{align*}
S^h &= \Span\{\phi_j \,\big|\, \text{ all } j\,\}, \\
S_0^h &= \Span\{\phi_j \,\big|\, \bx_j \notin \partial_D \Omega\} \subset S^h, \\
S_g^h &= S_0^h + \hat g \subset S^h.
\end{align*}
Then $\dim(S^h)=N$ while $\dim(S_0^h)=\dim(S_g^h)=N-L$, with $S_g^h$ only an affine subspace of $S^h$.

Our FEM requires that the weak formulation  \eqref{poissonweak} be true of $u_h\in S_g^h$ for all $v\in S_0^h$.  Thus we first write $u_h$ in the basis for $S_g^h$ using $N-L$ unknown coefficients $u_j$:
\begin{equation}
u_h(\bx) = \hat g(\bx) + \sum_{\bx_j \notin \partial_D \Omega} u_j\, \phi_j(\bx). \label{uhexpand}
\end{equation}
Then we require that the weak formulation hold for all $\phi_i$ in the basis of $S_0^h$.  That is, using definition \eqref{hatgdefine} and expansion \eqref{uhexpand}, we require
\begin{align}
\sum_{\bx_j \notin \partial_D \Omega} u_j \int_\Omega \grad \phi_j \cdot \grad \phi_i &= \int_\Omega f \phi_i + \int_{\partial_N\Omega} \gamma \phi_i \label{poissonfem} \\
&\qquad - \sum_{l=0,\dots,L} g(\bx_{j_l})  \int_\Omega \grad \phi_{j_l} \cdot \grad \phi_i \notag
\end{align}
for all $i$ such that $\bx_i \notin \partial_D \Omega$.  The coefficients $u_j$, for all $j$ such that $\bx_j \notin \partial_D \Omega$, are the unknowns in this equation.

Note that the support (i.e.~nonzero set) of $\phi_j$ includes only the node $\bx_j$ and all triangles (elements) $\triangle_k$ for which $\bx_j$ is a node of $\triangle_k$.  Thus the integral ``$\int_\Omega \grad \phi_j \cdot \grad \phi_i$'' in \eqref{poissonfem} is usually zero.  Specifically, it is zero if $\bx_i$ and $\bx_j$ are not both nodes of at least one triangle in the triangulation.


\section{Triangular meshes from \Triangle}

\PETSc itself does not include any tools for triangulating regions of the plane, sowe use the widely-available and easy-to-use \Triangle\sidenote{See \href{http://www.cs.cmu.edu/~quake/triangle.html}{www.cs.cmu.edu/$\sim$quake/ triangle.html} for documentation and source code. \Triangle may be available as a package in your operating system.} software \citep{Shewchuk1996} for this task.  \Triangle is both limited to planar regions and only capable of writing ASCII files.  Thus it is not a choice for performance, but of convenience.

\Triangle uses a simply-formatted ASCII file (extension \texttt{.poly}) as input to describe a polygonal region $\Omega$, and to indicate Dirichlet and Neumann portions of the boundary $\partial \Omega$.  For example, consider the input file \texttt{bump.poly} shown in Code \ref{code:bumppoly}.  This example polygon, a rectangle with a triangular bump in the base, is shown in Figure \ref{fig:bump-poly}.  It will reappear several times in this book as we solve more interesting PDEs on it.  The two apparently-unnecessary vertices introduced along the bottom help identify the Neumann part of the boundary, but note that \texttt{bump.poly} includes a Dirichlet/Neumann flag along each boundary segment.

\inputfromline{../c/ch8/bump.poly}{\CODELOC bump.poly}{A description of the boundary polygon in Figure \ref{fig:triangulation}, suitable for reading by \Triangle.}{16}{code:bumppoly}

The triangulation shown in Figure \ref{fig:triangulation} came from a single command which asks \Triangle to take \texttt{bump.poly} and generate a triangulation which has a polygon output file (option \texttt{-p}), relatively-uniform triangles (option \texttt{-q} for ``quality-checked'' \citep{Shewchuk1996}), and triangles with maximum area of $1.0$ (option \texttt{-a1.0}):
\begin{marginfigure}
\input{bump.poly.tikz}
\caption{The polygon described by \texttt{bump.poly} in Code \ref{code:bumppoly}.  The bold part is the closed Dirichlet boundary.  The lower boundary is Neumann, and has ``extra'' nodes to identify it as such.}
\label{fig:bump-poly}
\end{marginfigure}
\begin{cline}
$ triangle -pqa1.0 bump
\end{cline}
%$
This command generates three ASCII files, \texttt{bump.1.poly}, \texttt{bump.1.node}, and  \texttt{bump.1.ele}.  These files define the new (i.e.~refined relative to \texttt{bump.poly}) polygonal boundary, the nodes locations, and the elements in the triangulation, respectively.  For example, \texttt{bump.1.node} has the numbering and node locations shown in blue in Figure \ref{fig:triangulation}.

\Triangle includes a minimal visualization tool which shows the triangulation graphically.  The command
\begin{cline}
$ showme bump
\end{cline}
%$
displays the boundary polygon (from \texttt{bump.poly} or \texttt{bump.1.poly}) and the triangulation itself (\texttt{bump.1.node} and \texttt{bump.1.ele}).

\begin{figure}
\bigskip
% created by script tri2tikz.py command line:
%   ./tri2tikz.py --labelnodes --scale 2.0 --labeloffset 0.25 bump.1 bump.1.tikz
%
\begin{tikzpicture}[scale=2.000000]
  \draw[gray,very thin] (-1.500000,0.000000) -- (-2.250000,0.558372);
  \draw[gray,very thin] (-2.250000,0.558372) -- (-3.000000,0.000000);
  \draw[gray,very thin] (-3.000000,0.000000) -- (-1.500000,0.000000);
  \draw[gray,very thin] (-1.500000,0.000000) -- (0.000000,0.000000);
  \draw[gray,very thin] (0.000000,0.000000) -- (-0.750000,0.583333);
  \draw[gray,very thin] (-0.750000,0.583333) -- (-1.500000,0.000000);
  \draw[gray,very thin] (1.481325,1.942169) -- (0.422665,2.025778);
  \draw[gray,very thin] (0.422665,2.025778) -- (1.000000,1.000000);
  \draw[gray,very thin] (1.000000,1.000000) -- (1.481325,1.942169);
  \draw[gray,very thin] (-0.500000,1.500000) -- (0.000000,0.000000);
  \draw[gray,very thin] (0.000000,0.000000) -- (1.000000,1.000000);
  \draw[gray,very thin] (1.000000,1.000000) -- (-0.500000,1.500000);
  \draw[gray,very thin] (0.000000,3.000000) -- (-1.500000,3.000000);
  \draw[gray,very thin] (-1.500000,3.000000) -- (-1.208263,2.111158);
  \draw[gray,very thin] (-1.208263,2.111158) -- (0.000000,3.000000);
  \draw[gray,very thin] (2.000000,0.000000) -- (2.050000,1.050000);
  \draw[gray,very thin] (2.050000,1.050000) -- (1.000000,1.000000);
  \draw[gray,very thin] (1.000000,1.000000) -- (2.000000,0.000000);
  \draw[gray,very thin] (1.000000,0.000000) -- (2.000000,0.000000);
  \draw[gray,very thin] (2.000000,0.000000) -- (1.000000,1.000000);
  \draw[gray,very thin] (1.000000,1.000000) -- (1.000000,0.000000);
  \draw[gray,very thin] (2.394659,2.250000) -- (3.000000,1.500000);
  \draw[gray,very thin] (3.000000,1.500000) -- (3.000000,3.000000);
  \draw[gray,very thin] (3.000000,3.000000) -- (2.394659,2.250000);
  \draw[gray,very thin] (1.500000,3.000000) -- (0.422665,2.025778);
  \draw[gray,very thin] (0.422665,2.025778) -- (1.481325,1.942169);
  \draw[gray,very thin] (1.481325,1.942169) -- (1.500000,3.000000);
  \draw[gray,very thin] (2.050000,1.050000) -- (3.000000,0.000000);
  \draw[gray,very thin] (3.000000,0.000000) -- (3.000000,1.500000);
  \draw[gray,very thin] (3.000000,1.500000) -- (2.050000,1.050000);
  \draw[gray,very thin] (2.394659,2.250000) -- (2.050000,1.050000);
  \draw[gray,very thin] (2.050000,1.050000) -- (3.000000,1.500000);
  \draw[gray,very thin] (3.000000,1.500000) -- (2.394659,2.250000);
  \draw[gray,very thin] (-1.500000,3.000000) -- (-3.000000,3.000000);
  \draw[gray,very thin] (-3.000000,3.000000) -- (-2.250000,2.268853);
  \draw[gray,very thin] (-2.250000,2.268853) -- (-1.500000,3.000000);
  \draw[gray,very thin] (1.481325,1.942169) -- (1.000000,1.000000);
  \draw[gray,very thin] (1.000000,1.000000) -- (2.050000,1.050000);
  \draw[gray,very thin] (2.050000,1.050000) -- (1.481325,1.942169);
  \draw[gray,very thin] (3.000000,0.000000) -- (2.050000,1.050000);
  \draw[gray,very thin] (2.050000,1.050000) -- (2.000000,0.000000);
  \draw[gray,very thin] (2.000000,0.000000) -- (3.000000,0.000000);
  \draw[gray,very thin] (0.422665,2.025778) -- (1.500000,3.000000);
  \draw[gray,very thin] (1.500000,3.000000) -- (0.000000,3.000000);
  \draw[gray,very thin] (0.000000,3.000000) -- (0.422665,2.025778);
  \draw[gray,very thin] (2.394659,2.250000) -- (1.500000,3.000000);
  \draw[gray,very thin] (1.500000,3.000000) -- (1.481325,1.942169);
  \draw[gray,very thin] (1.481325,1.942169) -- (2.394659,2.250000);
  \draw[gray,very thin] (-1.750000,1.348485) -- (-0.750000,0.583333);
  \draw[gray,very thin] (-0.750000,0.583333) -- (-0.500000,1.500000);
  \draw[gray,very thin] (-0.500000,1.500000) -- (-1.750000,1.348485);
  \draw[gray,very thin] (-1.500000,0.000000) -- (-0.750000,0.583333);
  \draw[gray,very thin] (-0.750000,0.583333) -- (-1.750000,1.348485);
  \draw[gray,very thin] (-1.750000,1.348485) -- (-1.500000,0.000000);
  \draw[gray,very thin] (1.500000,3.000000) -- (2.394659,2.250000);
  \draw[gray,very thin] (2.394659,2.250000) -- (3.000000,3.000000);
  \draw[gray,very thin] (3.000000,3.000000) -- (1.500000,3.000000);
  \draw[gray,very thin] (2.050000,1.050000) -- (2.394659,2.250000);
  \draw[gray,very thin] (2.394659,2.250000) -- (1.481325,1.942169);
  \draw[gray,very thin] (1.481325,1.942169) -- (2.050000,1.050000);
  \draw[gray,very thin] (0.000000,3.000000) -- (-0.500000,1.500000);
  \draw[gray,very thin] (-0.500000,1.500000) -- (0.422665,2.025778);
  \draw[gray,very thin] (0.422665,2.025778) -- (0.000000,3.000000);
  \draw[gray,very thin] (1.000000,1.000000) -- (0.422665,2.025778);
  \draw[gray,very thin] (0.422665,2.025778) -- (-0.500000,1.500000);
  \draw[gray,very thin] (-0.500000,1.500000) -- (1.000000,1.000000);
  \draw[gray,very thin] (0.000000,0.000000) -- (-0.500000,1.500000);
  \draw[gray,very thin] (-0.500000,1.500000) -- (-0.750000,0.583333);
  \draw[gray,very thin] (-0.750000,0.583333) -- (0.000000,0.000000);
  \draw[gray,very thin] (-1.750000,1.348485) -- (-2.250000,2.268853);
  \draw[gray,very thin] (-2.250000,2.268853) -- (-3.000000,1.500000);
  \draw[gray,very thin] (-3.000000,1.500000) -- (-1.750000,1.348485);
  \draw[gray,very thin] (-3.000000,1.500000) -- (-3.000000,0.000000);
  \draw[gray,very thin] (-3.000000,0.000000) -- (-2.250000,0.558372);
  \draw[gray,very thin] (-2.250000,0.558372) -- (-3.000000,1.500000);
  \draw[gray,very thin] (-1.500000,0.000000) -- (-1.750000,1.348485);
  \draw[gray,very thin] (-1.750000,1.348485) -- (-2.250000,0.558372);
  \draw[gray,very thin] (-2.250000,0.558372) -- (-1.500000,0.000000);
  \draw[gray,very thin] (-3.000000,1.500000) -- (-2.250000,0.558372);
  \draw[gray,very thin] (-2.250000,0.558372) -- (-1.750000,1.348485);
  \draw[gray,very thin] (-1.750000,1.348485) -- (-3.000000,1.500000);
  \draw[gray,very thin] (0.000000,3.000000) -- (-1.208263,2.111158);
  \draw[gray,very thin] (-1.208263,2.111158) -- (-0.500000,1.500000);
  \draw[gray,very thin] (-0.500000,1.500000) -- (0.000000,3.000000);
  \draw[gray,very thin] (-0.500000,1.500000) -- (-1.208263,2.111158);
  \draw[gray,very thin] (-1.208263,2.111158) -- (-1.750000,1.348485);
  \draw[gray,very thin] (-1.750000,1.348485) -- (-0.500000,1.500000);
  \draw[gray,very thin] (-2.250000,2.268853) -- (-1.208263,2.111158);
  \draw[gray,very thin] (-1.208263,2.111158) -- (-1.500000,3.000000);
  \draw[gray,very thin] (-1.500000,3.000000) -- (-2.250000,2.268853);
  \draw[gray,very thin] (-3.000000,1.500000) -- (-2.250000,2.268853);
  \draw[gray,very thin] (-2.250000,2.268853) -- (-3.000000,3.000000);
  \draw[gray,very thin] (-3.000000,3.000000) -- (-3.000000,1.500000);
  \draw[gray,very thin] (-1.208263,2.111158) -- (-2.250000,2.268853);
  \draw[gray,very thin] (-2.250000,2.268853) -- (-1.750000,1.348485);
  \draw[gray,very thin] (-1.750000,1.348485) -- (-1.208263,2.111158);
  \draw[line width=2.5pt] (-3.000000,0.000000) -- (-3.000000,1.500000);
  \draw[line width=2.5pt] (-3.000000,3.000000) -- (-1.500000,3.000000);
  \draw[line width=2.5pt] (3.000000,3.000000) -- (3.000000,1.500000);
  \draw[line width=0.75pt] (3.000000,0.000000) -- (2.000000,0.000000);
  \draw[line width=0.75pt] (2.000000,0.000000) -- (1.000000,0.000000);
  \draw[line width=2.5pt] (1.000000,1.000000) -- (1.000000,0.000000);
  \draw[line width=2.5pt] (0.000000,0.000000) -- (1.000000,1.000000);
  \draw[line width=0.75pt] (0.000000,0.000000) -- (-1.500000,0.000000);
  \draw[line width=0.75pt] (-1.500000,0.000000) -- (-3.000000,0.000000);
  \draw[line width=2.5pt] (3.000000,1.500000) -- (3.000000,0.000000);
  \draw[line width=2.5pt] (0.000000,3.000000) -- (1.500000,3.000000);
  \draw[line width=2.5pt] (1.500000,3.000000) -- (3.000000,3.000000);
  \draw[line width=2.5pt] (-3.000000,1.500000) -- (-3.000000,3.000000);
  \draw[line width=2.5pt] (-1.500000,3.000000) -- (0.000000,3.000000);
  \draw (-2.825000,-0.250000) node {$0$};
  \filldraw (-3.000000,0.000000) circle (1.25pt);
  \draw (-2.825000,2.750000) node {$1$};
  \filldraw (-3.000000,3.000000) circle (1.25pt);
  \draw (3.175000,2.750000) node {$2$};
  \filldraw (3.000000,3.000000) circle (1.25pt);
  \draw (3.175000,-0.250000) node {$3$};
  \filldraw (3.000000,0.000000) circle (1.25pt);
  \draw (1.175000,-0.250000) node {$4$};
  \filldraw (1.000000,0.000000) circle (1.25pt);
  \draw (1.175000,0.750000) node {$5$};
  \filldraw (1.000000,1.000000) circle (1.25pt);
  \draw (0.175000,-0.250000) node {$6$};
  \filldraw (0.000000,0.000000) circle (1.25pt);
  \draw (2.175000,-0.250000) node {$7$};
  \filldraw (2.000000,0.000000) circle (1.25pt);
  \draw (-1.325000,-0.250000) node {$8$};
  \filldraw (-1.500000,0.000000) circle (1.25pt);
  \draw (3.175000,1.250000) node {$9$};
  \filldraw (3.000000,1.500000) circle (1.25pt);
  \draw (2.225000,0.800000) node {$10$};
  \filldraw (2.050000,1.050000) circle (1.25pt);
  \draw (0.175000,2.750000) node {$11$};
  \filldraw (0.000000,3.000000) circle (1.25pt);
  \draw (1.675000,2.750000) node {$12$};
  \filldraw (1.500000,3.000000) circle (1.25pt);
  \draw (1.656325,1.692169) node {$13$};
  \filldraw (1.481325,1.942169) circle (1.25pt);
  \draw (-0.325000,1.250000) node {$14$};
  \filldraw (-0.500000,1.500000) circle (1.25pt);
  \draw (-2.825000,1.250000) node {$15$};
  \filldraw (-3.000000,1.500000) circle (1.25pt);
  \draw (2.569659,2.000000) node {$16$};
  \filldraw (2.394659,2.250000) circle (1.25pt);
  \draw (0.597665,1.775778) node {$17$};
  \filldraw (0.422665,2.025778) circle (1.25pt);
  \draw (-0.575000,0.333333) node {$18$};
  \filldraw (-0.750000,0.583333) circle (1.25pt);
  \draw (-1.575000,1.098485) node {$19$};
  \filldraw (-1.750000,1.348485) circle (1.25pt);
  \draw (-2.075000,0.308372) node {$20$};
  \filldraw (-2.250000,0.558372) circle (1.25pt);
  \draw (-1.325000,2.750000) node {$21$};
  \filldraw (-1.500000,3.000000) circle (1.25pt);
  \draw (-1.033263,1.861158) node {$22$};
  \filldraw (-1.208263,2.111158) circle (1.25pt);
  \draw (-2.075000,2.018853) node {$23$};
  \filldraw (-2.250000,2.268853) circle (1.25pt);
\end{tikzpicture}

\caption{A FEM triangulation generated by \Triangle.  {\color{blue} Nodes} are labeled by $j=0,\dots,N-1$ with $N=24$ and {\color{red} elements} are labeled by $k=0,\dots,K-1$ with $K=32$.}
\label{fig:triangulation}
\end{figure}


\section{Getting a triangular mesh from ASCII files into a \PETSc \pVec}

The ASCII files produced by \Triangle produces are not in a good format for large meshes, but we will accept this for portability and readability.  We will, however, demonstrate how files are read in parallel, based on first doing a mundane and un-scalable task in \PETSc, namely reading the ASCII \Triangle files serially onto a single processor and writing out a binary file in \PETSc format.\sidenote{Furthermore we will focus on the scalability of the FEM solution process, once the mesh is loaded.}  The binary file both describes the mesh element-by-element and can be read in parallel.

The code \texttt{c3convert.c} does the conversion.  It is invoked, for the present purposes, by
\begin{cline}
$ c3convert -f bump.1
\end{cline}
%$
This reads ASCII files \texttt{bump.1.\{node,ele,poly\}} and writes a \PETSc-formatted binary file \texttt{bump.1.petsc}.

We will not show \texttt{c3convert.c}, but we summarize the high points.  First \PETSc is initialized and we get the rank of the current (MPI) process.  We only ask the first process (``rank zero'') in the MPI communicator to do any work.\sidenote{This code can be invoked ``\texttt{mpiexec -n NN c3convert}'', but it behaves as a serial code.}  This part of the code first reads the header information in the \texttt{.node} file, and allocates \PETSc \pVecs according.  We use \texttt{VecCreateSeq} to allocate the a sequential \pVec \texttt{vx}, which contains the $x$-coordinate of the nodes, only on the rank zero process.  Then \texttt{VecDuplicate} is used to allocate two more \pVecs with the same layout, \texttt{vy} and \texttt{vBT}.  This last \pVec will contain a flag $\{0,2,3\}$ for each node, where $0$ is an interior node, $2$ is a Dirichlet boundary node, and $3$ is a Neumann boundary node.

Then we read the node locations from the \texttt{.node} file.  The reading itself is done with the standard C library call \texttt{fscanf}.  Then \texttt{VecSetValues} is used to set one entry at a time.  After setting these values, which stores a list of entrys into an internal \PETSc dynamic data structure, we ask \PETSc to assemble the \pVecs.

The next part of \texttt{c3convert.c} reads boundary polygon information from the \texttt{.poly} file.  Each segment of the boundary polygon corresponds to two node indices.  We store the segments in a \pVec with blocksize 2.  Then we read the header information in the \texttt{.ele} file and allocates a \pVec called \texttt{vE} for the elements.  This part of the code is an important transformation of the data structures.  In fact, \texttt{vE} has block size 15,\sidenote{A \PETSc \pVec is designed to hold \texttt{PetscScalar} data types, i.e.~\texttt{double}.  So we are being quite wasteful for integer indices and boolean flags.} and, in contrast to the format from \Triangle, it contains all the information about each element that we need to do assemble the matrix equation.  Each of its blocks is the C \texttt{struct} shown in Code \ref{code:elementtype}.

\cinputraw{../c/ch8/readmesh.h}{extract from \texttt{\CODELOC readmesh.h}}{The \texttt{elementtype} struct.}{}{//STARTSTRUCT}{//ENDSTRUCT}{code:elementtype}

In the next part of \texttt{c3convert.c} we fill the \pVec for elements with all of the information read so far, including the node indices for each element which we read from the \texttt{.ele} file.  This stage is fundamentally serial, because we must look at the entire mesh to find the node coordinates and node/segment boundary type for each node and edge of each element.  In this part there is an important detail about triangulations, which affects the data structure for elements.  Namely, we cannot tell if an edge of an element is in the boundary just by whether both endpoints are in the boundary.  For example, the element (triangle) labeled ``5'' in Figure \ref{fig:triangulation} has an edge from node 5 (on the Dirichlet boundary) to node 7 (on the Neumann boundary).  But triangle 5 is \emph{not} a boundary element.  Thus we need to have list of flags for the boundary segments themselves.  Thus the \texttt{elementtype} structure above has both a boundary type for each node of each element and a boundary type for each edge of each element.

At this point we have the whole triangulation into \PETSc \pVecs.  The almost-last part of \texttt{c3convert.c} simply creates a \PETSc ``viewer'' and ``view'' all of the \pVecs which contain the mesh.  We will be able to reread these \pVecs in parallel, as long as we re-read them in the same order.  The final bit of \texttt{c3convert.c} checks if option \texttt{-check} is given, and if so we read back the binary file in parallel.  This part of \texttt{c3convert.c} calls two methods from a separate code (and re-used component) \texttt{readmesh.c} (Codes \ref{code:readmeshpartone} and \ref{code:readmeshparttwo}).

The first method \texttt{getmeshfile()} finds a \PETSc binary file from the \texttt{-f} option.  The other major method \texttt{readmesh()}, in Code \ref{code:readmeshparttwo}, creates and reads three \pVecs in parallel from it, using utility methods \texttt{createloadname()} and \texttt{getcheckmeshsizes()}.  Note that prefixes are set on each \pVec so that the block size is correctly read.

\cinputpart{readmesh.c}{\CODELOC}{Determine the filename of a \PETSc binary file that has a mesh.}{I}{//STARTGET}{//ENDGET}{code:readmeshpartone}

\cinputpart{readmesh.c}{\CODELOC}{Read the mesh in parallel from the file.}{II}{//STARTREADMESH}{//ENDREADMESH}{code:readmeshparttwo}


\section{Constructing the FEM linear system}

Now that we can get a triangulation into \PETSc, we can return to the finite-dimensional weak formulation \eqref{poissonfem}.  This linear system
\begin{equation}
A \bu = \bb, \label{poissonmatrix}
\end{equation}
has $A\in\RR^{N\times N}$ and $\bu,\bb\in\RR^N$, where $N$ is the number of nodes.  We will write a code which assembles $A$ and $\bb$ and solves for $\bu$.  \PETSc \pMat and \pVec objects store the problem, and we use a \pKSP object to solve it.

For ease of construction we include the node-wise Dirichlet conditions $u_j=g(\bx_j)$ as equations in this linear system, treating such $u_j$ as unknowns, so that the matrix has $N$ rows and columns if there are $N$ nodes in total.  Thus we define $A$ to have entries
\begin{equation}
a_{ij} = \int_\Omega \grad \phi_j \cdot \grad \phi_i \label{Aentryfem}
\end{equation}
if $\bx_i \notin \partial_D \Omega$ and $\bx_j \notin \partial_D \Omega$, while otherwise
\begin{equation*}
a_{ij} = \delta_{ij}.
\end{equation*}
Notice that we index rows and columns of matrices and vectors starting with zero,\sidenote{This follows C and \PETSc conventions, but is opposed to long-standing traditions about linear algebra!} and that $j=0,1,\dots,N-1$ in particular.

Observe that $A$ is \emph{symmetric}, $a_{ij}=a_{ji}$.  Furthermore $A$ is \emph{sparse}, that is, most entries are zero, at least for triangulations with more than a handful of triangles.  These facts affect the algorithms we choose when solving \eqref{poissonmatrix}.

For the right side of \eqref{poissonfem}, define the entries
\begin{equation}
    b_i = \int_\Omega f \phi_i + \int_{\partial_N\Omega} \gamma \phi_i - \sum_{l=0,\dots,L} g(\bx_{j_l})  \int_\Omega \grad \phi_{j_l} \cdot \grad \phi_i  \label{bentryfem}
\end{equation}
if $\bx_i \notin \partial_D \Omega$, while if $\bx_i \in \partial_D \Omega$ then
    $$b_i = g(\bx_{i}).$$

In practice we \emph{don't} initially build $A$ or $\bb$ as described above.  We start with a matrix $\tilde A$ and vector $\tilde \bb$ which have the same size as $A$ and $\bb$, respectively, but which ignore the Dirichlet boundary.  We don't even evaluate the function $g$.  All entries of $\tilde A$ are computed by the first formula above for $a_{ij}$, that is, the entries are
\begin{equation*}
\tilde a_{ij} = \int_\Omega \grad \phi_i \cdot \grad \phi_j
\end{equation*}
for all $i,j=0,1,\dots,N$.  Also the simpler right-hand side $\tilde \bb\in\RR^N$ has entries
    $$\tilde b_i = \int_\Omega f \phi_i + \int_{\partial_N\Omega} \gamma \phi_i$$
for all $i=0,1,\dots,N$.

This initial linear system $\tilde A\bu=\tilde \bb$ is (obviously) final in the case where there is no Dirichlet boundary (i.e.~$\partial_D \Omega=\emptyset$).  However, in this case $u=C$, where $C\in\RR$ is any constant value, solves $-\grad^2 u=0$ with $\partial u/\partial n = 0$ on the whole boundary $\partial_N \Omega = \partial \Omega$.  The Poisson problem then only determines the solution up to such an additive constant.  Because it does not have a unique solution, when solving \eqref{poissonmatrix} in this Neumann case we will have to inform \PETSc about the null space of constant functions.

In general, the next step is to edit $\tilde A$ in each Dirichlet row, that is, for each $i$ where $\bx_i \in \partial_D \Omega$.  For each such row we replace the whole row with the corresponding row of the identity, and also we replace $\tilde b_i$ with $b_i = g(\bx_i)$.  Furthermore, in each column $j$ for which $\bx_j \in \partial_D \Omega$, we move all entries $\tilde a_{ij}$ where $i$ is \emph{not} a Dirichlet row index over to the right-hand side, multiplied by the negative of the boundary value $g(\bx_{j})$.  We can write
\begin{equation}
    \tilde b_i \to \tilde b_i - g(\bx_{j}) \tilde a_{ij} \label{btransform}
\end{equation}
for these transformations.  After the completion of this ``editing'' stage we get $A$ and $\bb$.  Since this part of the matrix assembly is a key stage in our FEM codes, we now give a concrete example.

\medskip\noindent\hrulefill
\begin{example} Figure \ref{fig:squarefour} shows a triangulation of the unit square with five nodes.  The matrix $\tilde A$ has the following nonzero pattern; the zero entries are shown as spaces:\begin{marginfigure}
\input{squarefour.tikz}
\caption{A triangulation of a square with five nodes.  The top segment is the Dirichlet boundary.}
\label{fig:squarefour}
\end{marginfigure}%
\begin{equation*}
\tilde A = \begin{bmatrix}
\X & \X &    & \X & \X \\
\X & \X & \X &    & \X \\
   & \X & \X & \X & \X \\
\X &    & \X & \X & \X \\
\X & \X & \X & \X & \X
\end{bmatrix}.
\end{equation*}
Note $\tilde a_{ij}=0$ only where the integral $\int_\Omega \grad \phi_j \cdot \grad \phi_i$ is zero, a rare event in this small (coarse)-mesh case.  Now, because the $i=1,2$ nodes live in the (closed) Dirichlet boundary $\partial_D \Omega$, we highlight the $i=1,2$ rows and $j=1,2$ columns which will be edited:\marginnote{{\color{red} \textbf{Caution}}:  The first row or column of any matrix in this book is numbered ``$0$.''}
\begin{equation*}
\tilde A = \begin{bmatrix}
\X & \redX &    & \X & \X \\
\blueX & \blueX & \blueX &    & \blueX \\
   & \blueX & \blueX & \blueX & \blueX \\
\X &    & \redX & \X & \X \\
\X & \redX & \redX & \X & \X
\end{bmatrix}.
\end{equation*}
The underlined {\color{blue} blue} entries are changed to $0$ or $1$ so that these rows become rows of the identity; the old computed values $\tilde a_{ij}$ are tossed out.  The underlined {\color{red} red} entries, specifically the four entries $\tilde a_{01}$, $\tilde a_{32}$, $\tilde a_{41}$, and $\tilde a_{42}$, are moved over to the right side using transformation \eqref{btransform}; these $\tilde a_{ij}$ values get used.  The final linear system $A \bu = \bb$, is
\begin{equation*}
\begin{bmatrix}
\X & & & \X & \X \\
 & \,1\, & & & \\
 & & \,1\, & & \\
\X & & & \X & \X \\
\X & & & \X & \X
\end{bmatrix}
\begin{bmatrix}
u_0 \\
u_1 \\
u_2 \\
u_3 \\
u_4
\end{bmatrix}
=
\begin{bmatrix}
\tilde b_0 - g(\bx_1) \tilde a_{01} \\
g(\bx_1) \\
g(\bx_2) \\
\tilde b_3 - g(\bx_2) \tilde a_{32} \\
\tilde b_4 - g(\bx_1) \tilde a_{41} - g(\bx_2) \tilde a_{42}
\end{bmatrix}
\end{equation*}
\end{example}
\noindent\hrulefill

\bigskip


\section{Assembling the matrix equation, element by element}

An entry $\tilde a_{ij}$ of the initial matrix $\tilde A$ is not computed in one step.  Rather, we compute contributions to these entries from each element in turn; we ``loop over the elements.''  Furthermore we do it in parallel.  Recall that \texttt{c3convert.c} above creates a distributed \pVec called \texttt{E}, which is an array of elements of type \texttt{elementtype}, and \texttt{readmesh.c} reads it.

For the element-by-element assembly procedure we first define the integral over one triangle $\triangle_k$,
\begin{equation}
\text{\texttt{a(k,i,j)}} = \int_{\triangle_k} \grad \phi_i \cdot \grad \phi_. \label{aelementintegral}
\end{equation}
so that
    $$\tilde a_{ij} = \sum_k \phantom{.}\text{\texttt{a(k,i,j)}}.$$
Likewise we will compute
\begin{equation}
\text{\texttt{b(k,i)}} = \int_{\triangle_k} f \phi_i + \int_{\overline{\triangle}_k \cap \partial_N\Omega} \gamma \phi_i, \label{belementintegral}
\end{equation}
so that $\tilde b_i = \sum_{k} \;\text{\texttt{b(k,i)}}$.

If \texttt{E} denotes an array of \texttt{elementtype} then for each triangle $\triangle_k$ we can get the node index for each of its three vertices.  Using $q=0,1,2$ for these vertices,
    $$\text{\texttt{E[k].j[q]}} \in \{0,1,\dots,N-1\}.$$
Thus the element-by-element assembly of $\tilde A$ follows this pseudocode:
\begin{code}
A = 0                           // N x N sparse matrix with entries a(i,j)
b = 0                           // N x 1 column vector with entries b(i)
for k = 0 to K-1                // loop through all elements
    for q = 0 to 2
        i = E[k].j[q]           // row index
        for r = 0 to 2
            j = E[k].j[r]       // column index
            a(i,j) += a(k,q,r)  // add contribution from element k
            b(i)   += b(k,q)    // ditto
\end{code}
\medskip\noindent
Observe that on each triangle $\triangle_k$ we write ``\texttt{a(k,q,r)}'' for \texttt{a(k,i,j)}, by replacing the global indices $i,j$ by corresponding local node indices $q,r\in\{0,1,2\}$ for triangle $k$.

A standard approach to computing the element-wise integrals \texttt{a(k,q,r)} is to refer triangle $\triangle_k$ to a reference triangle $\triangle_\ast$ with vertices $(0,0),\,(1,0),\,(0,1)$, as shown in Figure \ref{fig:isoparametric}.  The linear map from $\triangle_\ast$ to $\triangle_k$ with vertices $(x_0,y_0),\,(x_1,y_1),\,(x_2,y_2)$, as shown in the Figure, is
\begin{align}
x(\xi,\eta) &= x_0 + (x_1-x_0) \xi + (x_2-x_0) \eta, \label{trianglemap} \\
y(\xi,\eta) &= y_0 + (y_1-y_0) \xi + (y_2-y_0) \eta. \notag
\end{align}

\begin{marginfigure}
\begin{tikzpicture}[scale=1.1,
    decoration={
      markings,
      mark=at position 1 with {\arrow[scale=1.8,gray]{latex}};
    }]
% left x,y axes
    \draw[->, gray, very thin] (1.5,0) -- (4.0,0);
    \draw[->, gray, very thin] (2,-0.5) -- (2,2.4);
    \draw (4.1,-0.1) node {$x$};
    \draw (1.9,2.4) node {$y$};
    \filldraw (1.7,1) circle (1.25pt);    % (x_0,y_0)
    \filldraw (3.5,-0.3) circle (1.25pt); % (x_1,y_1)
    \filldraw (3.0,2.0) circle (1.25pt);  % (x_2,y_2)
    \draw (1.4,1.3) node {$(x_0,y_0)$};
    \draw (3.5,-0.6) node {$(x_1,y_1)$};
    \draw (3.0,2.3) node {$(x_2,y_2)$};
    \draw[line width=1pt] (1.7,1) -- (3.5,-0.3) -- (3.0,2.0) -- cycle;
    \draw (2.7,1.0) node {$\triangle_k$};
% right xi,eta axes
    \draw[->, gray, very thin] (4.6,0) -- (6.6,0);
    \draw[->, gray, very thin] (5,-0.4) -- (5,2.0);
    \draw (6.7,-0.1) node {$\xi$};
    \draw (4.9,2.05) node {$\eta$};
    \filldraw (5,0) circle (1.25pt);  % (0,0)
    \filldraw (6,0) circle (1.25pt);  % (1,0)
    \filldraw (5,1) circle (1.25pt);  % (0,1)
    \draw (5.3,-0.3) node {$(0,0)$};
    \draw (6.3,0.2) node {$(1,0)$};
    \draw (5.4,1.1) node {$(0,1)$};
    \draw[line width=1pt] (5,0) -- (6,0) -- (5,1) -- cycle;
    \draw (5.3,0.3) node {$\triangle_\ast$};
% arrows connecting nodes
    \draw[gray, postaction={decorate}] (5,0) -- (1.7,1.03);
    \draw[gray, postaction={decorate}] (6,0) -- (3.5,-0.3);
    \draw[gray, postaction={decorate}] (5,1) -- (3.0,2.0);
\end{tikzpicture}
\smallskip
\caption{Mapping of a triangle $\triangle_k$ from the reference triangle $\triangle_\ast$.}
\label{fig:isoparametric}
\end{marginfigure}

\noindent Furthermore, on $\triangle_\ast$ any linear function is a linear combination of these three local basis functions:
\begin{equation}
\chi_0(\xi,\eta) = 1-\xi-\eta, \qquad \chi_1(\xi,\eta) = \xi, \qquad \chi_2(\xi,\eta) = \eta. \label{chiformulas}
\end{equation}
On $\triangle_k$, each of the basis functions $\phi_q$ is the mapped version of the corresponding $\chi_q$:
\begin{equation}
\phi_q(x(\xi,\eta),y(\xi,\eta)) = \chi_q(\xi,\eta). \label{phichimap}
\end{equation}
From \eqref{trianglemap} and \eqref{phichimap} one can confirm that $\phi_q(x_r,y_r) = \delta_{qr}$.

Now is a good point at which to observe that the set of $\phi_j$, over all nodes $j=0,1,\dots,N$, is a partition of unity.  We can see this by switching to local node coordinate $q$ and using $\chi_0+\chi_1+\chi_2=1$, which is obvious from \eqref{chiformulas}.  That is, if $\bx \in \triangle_k$ then
\begin{equation}
   \sum_j \phi_j(\bx) = \sum_{q=0,1,2} \phi_q(\bx) = \sum_{q=0,1,2} \chi_q =1.  \label{partitionofunity}
\end{equation}

The Jacobian\sidenote{By definition, the \emph{Jacobian} $J=J(\bx)$ of a smooth map $F$ at a point $\bx$ is its linearization at $\bx$.  That is, if $\by=F(\bx)$ and $\by+\Delta\by = F(\bx+\Delta\bx)$ then $J(\bx)$ satisfies $\Delta \by = J(\bx) \Delta\bx + o(|\Delta\bx|)$.} of map \eqref{trianglemap} is
% CAUTION: Elman (1.37) uses "Jacobian" for the transpose of this, the true Jacobian (e.g. in Newton later)
\begin{equation}
J = \begin{bmatrix}
    \dfrac{\partial x}{\partial \xi} & \dfrac{\partial x}{\partial \eta} \\[1.0em]
    \dfrac{\partial y}{\partial \xi} & \dfrac{\partial y}{\partial \eta}
    \end{bmatrix}
    =
    \begin{bmatrix}
    x_1-x_0 & x_2-x_0 \\
    y_1-y_0 & y_2-y_0
    \end{bmatrix}.  \label{trianglejacobian}
\end{equation}
Recalling both the change-of-variables formula for integrals\sidenote{If $F$ is a smooth map from $\bx \in U$ to $\by \in F(U)$, $J$ is the Jacobian of $F$, and $g$ is integrable on $F(U)$, then $$\int_{F(U)} g(\by)\,d\by = \int_U g(F(\bx))\,|\det(J)|\,d\bx.$$} and the chain rule, by \eqref{phichimap} we can write
\begin{align}
\text{\texttt{a(k,q,r)}} &= \int_{\triangle_k} \grad\phi_q \cdot \grad\phi_r \label{elementcontrib} \\
   &= \int_{\triangle_k} \frac{\partial\phi_q}{\partial x} \frac{\partial\phi_r}{\partial x} + \frac{\partial\phi_q}{\partial y} \frac{\partial\phi_r}{\partial y} \; dx \, dy  \notag \\
   &= \int_{\triangle_\ast} \left(\frac{\partial \chi_q}{\partial \xi} \frac{\partial \xi}{\partial x} + \frac{\partial \chi_q}{\partial \eta} \frac{\partial \eta}{\partial x}\right) \left(\frac{\partial \chi_r}{\partial \xi} \frac{\partial \xi}{\partial x} + \frac{\partial \chi_r}{\partial \eta} \frac{\partial \eta}{\partial x}\right) \notag \\
   &\qquad\qquad + \left(\frac{\partial \chi_q}{\partial \xi} \frac{\partial \xi}{\partial y} + \frac{\partial \chi_q}{\partial \eta} \frac{\partial \eta}{\partial y}\right) \left(\frac{\partial \chi_r}{\partial \xi} \frac{\partial \xi}{\partial y} + \frac{\partial \chi_r}{\partial \eta} \frac{\partial \eta}{\partial y}\right) \; |\det(J)| \; d\xi \, d\eta. \notag
\end{align}
The last expression is the low point of this calculation.  From now on the formulas simplify because the integrand is \emph{constant} for the $\Pone$ elements here.  Indeed, by \eqref{trianglejacobian} we see $\det(J)$ is constant, but also the derivatives $\partial \chi_q/\partial\{\xi,\eta\}$ and $\partial\{\xi,\eta\}/\partial\{x,y\}$ are all constant because the functions in question are linear.

However, we need computable formulas for ``$\partial \xi/\partial x$'' and similar terms.  Again by the chain rule we can compute
\begin{equation*}
\begin{bmatrix}
    1 & 0 \\[0.2em]
    0 & 1
\end{bmatrix}
=\begin{bmatrix}
    \dfrac{\partial x}{\partial x} & \dfrac{\partial x}{\partial y} \\[1.0em]
    \dfrac{\partial y}{\partial x} & \dfrac{\partial y}{\partial y}
\end{bmatrix}
=
\begin{bmatrix}
    \dfrac{\partial x}{\partial \xi} & \dfrac{\partial x}{\partial \eta} \\[1.0em]
    \dfrac{\partial y}{\partial \xi} & \dfrac{\partial y}{\partial \eta}
\end{bmatrix}
\begin{bmatrix}
    \dfrac{\partial \xi}{\partial x} & \dfrac{\partial \xi}{\partial y} \\[1.0em]
    \dfrac{\partial \eta}{\partial x} & \dfrac{\partial \eta}{\partial y}
\end{bmatrix},
\end{equation*}
which is to say $I=J\; J^{-1}$.  On the other hand we can invert the $2\times 2$ matrix $J$ by hand to get
\begin{equation}
J^{-1}
= \frac{1}{\det(J)}
\begin{bmatrix}
    \dfrac{\partial y}{\partial \eta} & -\dfrac{\partial x}{\partial \eta} \\[1.0em]
    -\dfrac{\partial y}{\partial \xi} & \dfrac{\partial x}{\partial \xi}
\end{bmatrix}
= \frac{1}{\det(J)}
\begin{bmatrix}
    y_2-y_0 & x_0-x_2 \\
    y_0-y_1 & x_1-x_0
\end{bmatrix}.  \label{Jinverse}
\end{equation}
Now let $y_{20}=y_2-y_0$, $x_{02}=x_0-x_2$, $y_{01}=y_0-y_1$ and $x_{10}=x_1-x_0$.  In terms of these differences,
\begin{equation}
\begin{bmatrix}
    \dfrac{\partial \xi}{\partial x} & \dfrac{\partial \xi}{\partial y} \\[1.0em]
    \dfrac{\partial \eta}{\partial x} & \dfrac{\partial \eta}{\partial y}
\end{bmatrix}
= \frac{1}{\det(J)}
\begin{bmatrix}
    y_{20} & x_{02} \\
    y_{01} & x_{10}
\end{bmatrix} \label{dxietadxy}
\end{equation}
and
\begin{equation}
\det(J) = x_{10} y_{20} - y_{01} x_{02}. \label{detJ}
\end{equation}
Thus, by \eqref{elementcontrib}, \eqref{dxietadxy}, and \eqref{detJ}, and because the area of the reference triangle $\triangle_\ast$ is $1/2$\,, our element-wise contribution to $\tilde a_{ij}$ simplifies to
\begin{align}
\text{\texttt{a(k,q,r)}} = \frac{1}{2\; |\det(J)|} &\bigg[\left(\frac{\partial \chi_q}{\partial \xi} y_{20} + \frac{\partial \chi_q}{\partial \eta} y_{01}\right) \left(\frac{\partial \chi_r}{\partial \xi} y_{20} + \frac{\partial \chi_r}{\partial \eta} y_{01}\right) \label{elementcontribFINAL} \\
&\quad + \left(\frac{\partial \chi_q}{\partial \xi} x_{02} + \frac{\partial \chi_q}{\partial \eta} x_{10}\right) \left(\frac{\partial \chi_r}{\partial \xi} x_{02} + \frac{\partial \chi_r}{\partial \eta} x_{10}\right)\bigg]. \notag
\end{align}
Also note that
\begin{align}
\frac{\partial \chi_0}{\partial \xi} &= -1, & \frac{\partial \chi_1}{\partial \xi} &= 1, & \frac{\partial \chi_2}{\partial \xi} &= 0, \label{chiderivs} \\
\frac{\partial \chi_0}{\partial \eta} &= -1, & \frac{\partial \chi_1}{\partial \eta} &= 0, & \frac{\partial \chi_2}{\partial \eta} &= 1. \notag
\end{align}
Combining \eqref{elementcontribFINAL} and \eqref{chiderivs}, we know enough to write code to compute \texttt{a(k,q,r)}, and thus the entries in $\tilde A$.

To complete the code for the initial linear system $\tilde A \bu = \tilde\bb$ we also need to address the right-hand side $\tilde\bb$, that is, we need to compute the element-wise contribution \texttt{b(k,i)} in \eqref{belementintegral}.  We do these integrals by changing variables to the reference element, but we also only do the integrals approximately by quadrature.  This is because the data $f$ and $\gamma$ may have no analytically-computable integral, and because a modest error in evaluating the right side of the linear system can have only a small effect on the solution of the FEM linear system, which is, after all, only an approximation of the solution of the Poisson problem.

FIXME: in the homogeneous Neumann case ($\gamma=0$) we use midpoints of the edges of the triangles in our quadrature rule from \citet{Ciarlet2002}
\begin{align*}
\text{\texttt{b(k,q)}} &= \int_{\triangle_\ast} f\, \chi_q \; |\det(J)| \; d\xi \, d\eta \\
  &\approx \frac{|\det(J)|}{6} \left(\omega(0.5,0) + \omega(0.5,0.5) + \omega(0,0.5)\right)
\end{align*}
where $\omega(\xi,\eta) = f(x(\xi,\eta),y(\xi,\eta)) \chi_q(\xi,\eta)$.

%\section{Preallocate a \pMat}
%FIXME: figure showing parallel partition of elements
%\cinputpart{testprealloc.c}{Read mesh \pVecs from file.  Get row ownership range.}{I}{//STARTLOAD}{//ENDLOAD}{code:testpreallocpartone}
%\cinputpart{testprealloc.c}{Set up \pMat $A$ and actually preallocate it.  Fill it with junk entries so the pattern can be visualized.}{II}{//ENDLOAD}{//ENDTEST}{code:testpreallocparttwo}

\begin{comment}
From: Fande Kong <fdkong.jd@gmail.com>
To: PETSc users list <petsc-users@mcs.anl.gov>
Subject: Re: [petsc-users] Status on parallel mesh reader in DMPlex
> Message: 2
>Date: Fri, 18 Dec 2015 08:21:04 -0800
>From: Justin Chang <jychang48@gmail.com>
>To: petsc-users <petsc-users@mcs.anl.gov>
>Subject: [petsc-users] Status on parallel mesh reader in DMPlex

> Hi all,

>What's the status on the implementation of the parallel

I am actually developing a parallel loader. The loader has two steps:

(1) Use 1 core or several cores to read a sequential  mesh, and partition
it into $np$ parts using a partitioner, $np$ is the number of cores you
want to use to do a simulation. Usually, $np$ is large, for example, 10,000
cores. And then we write the partitioned data into the file system as a
HDF5 file.

(2) Load the partitioned data (HDF5 file) into $np$ cores in parallel.

This idea works pretty well for me at this point.  But, this loader is
not compatible with DMPlex now. However, I could change the code somehow,
If this idea is acceptable.

You could also implement it by yourself. It is not too bad.

>mesh reader/generator for DMPlex meshes? Is anyone actively working on
> this? If so is there a branch that I can peek into?

Parallel generator is highly nontrivial, I think. It is a very hard topic.

> Thanks,
> Justin
\end{comment}


\section{Solve the Poisson problem}

\cinputpart{poissontools.c}{\CODELOC}{Find Dirichlet rows, \texttt{INSERT\_VALUES} these, and call \texttt{MatAssemblyBegin/End} and \texttt{VecAssemblyBegin/End}.}{I}{//DIRICHLETROWS}{//ENDDIRICHLETROWS}{code:poissontoolsdirichlet}

\cinputpart{poissontools.c}{\CODELOC}{Start computing all other contributions element-by-element: Precompute all geometry of element and quadrature points.}{II}{//ASSEMBLEADDONE}{//ENDONE}{code:poissontoolsaddone}

\cinputpart{poissontools.c}{\CODELOC}{Compute all other contributions element-by-element including source $f$, Dirichlet columns using data $g$, and Neumann data $\gamma$.  Use \texttt{ADD\_VALUES} for these, and call \texttt{MatAssemblyBegin/End} and \texttt{VecAssemblyBegin/End}.}{III}{//ENDONE}{//ENDASSEMBLEADD}{code:poissontoolsaddtwo}

\cinputpart{poissontools.c}{\CODELOC}{Call \texttt{dirichletrows()} and \texttt{assembleadd()}.}{IV}{//FULLASSEMBLE}{//ENDFULLASSEMBLE}{code:poissontoolsassemble}


\cinputpart{poissonfem.c}{\CODELOC}{Read in mesh in parallel (i.e.~\pVec \texttt{E}) and create \pMat \texttt{A} and \pVec \texttt{b}.}{I}{//GETMESH}{//ENDMATVECCREATE}{code:cthreepoissongetmesh}

In the next part of the code (Code \ref{code:cthreepoissonchecks}) we do two basic checks on our assembly procedure, without solving a linear system.  First we check the construction of \texttt{A} by checking if constant functions are in the kernel when $\partial_D\Omega=\emptyset$.  In this check the data $f$, $g$, and $\gamma$ are all not used, and the boundary information in \texttt{E} is not used either.

\cinputpart{poissonfem.c}{\CODELOC}{Do two checks on the construction of the linear system.}{II}{//TWOCHECKS}{//ENDTWOCHECKS}{code:cthreepoissonchecks}

In the second check we verify the construction of $\bb$ in the case where $f\equiv 1$, $\gamma=0$, and $\partial_D\Omega=\emptyset$.  In this case, by formula \eqref{bentryfem} and observation \eqref{partitionofunity}, the sum of the entries of $\bb$ should be the area of the region:
   $$\sum_j b_j = \sum_j \int_\Omega 1\, \phi_j = \int_\Omega \sum_j \phi_j = |\Omega|.$$

\cinputpart{poissonfem.c}{\CODELOC}{Solve a Dirichlet problem for which we know the solution.}{III}{//SOLVEMANU}{//ENDSOLVEMANU}{code:cthreepoissonsolvemanu}


%\caveat{It works, but does it work well?}


\chapter{Systems: Shallow water and Stokes equations}
\label{chap:sy}
\renewcommand{\CODELOC}{ch10/}
\stubinput{stokes.tex}{5}

\chapter{Unstructured grids}
\label{chap:dp}
\renewcommand{\CODELOC}{ch11/}
\stubinput{dmplex.tex}{0}

\chapter{Constraints}
\label{chap:co}
\renewcommand{\CODELOC}{ch12/}
\stubinput{constrained.tex}{20}

%%%%%%%%%%

\backmatter

\bibliography{book}
\bibliographystyle{plainnat}

\clearpage

\newcommand{\tblockeqncode}[3]{
\begin{tabular}[t]{l} #1 \\ \quad {\normalsize \texttt{#3}} \\ \qquad \fbox{\small #2} \end{tabular}
}
\newcommand{\tblockcode}[2]{
\begin{tabular}[t]{l} #1 \\ \quad {\normalsize \texttt{#2}} \end{tabular}
}
\newcommand{\tblock}[1]{
\begin{tabular}[t]{l} #1 \end{tabular}
}

\clearpage
\thispagestyle{empty}
\noindent \textsc{inside front cover:}

\vfill
{\large \noindent \textsc{PDE examples}.} \quad Chapter $N$ codes are in directory \texttt{c/ch}$N$\texttt{/}.

\begin{center}
\small
\hspace{-10mm}\begin{tabular}{lllll}
\toprule
Chap.
    &linear
          &nonlinear
                &nonlinear time-dependent \\
\midrule  \bigskip
\ref{chap:st}
    & \tblockeqncode{Poisson (2D)}{$-\grad^2 u = f$}{poisson.c}
          &     &      \\ \bigskip
\ref{chap:nl}
    &     & \tblockeqncode{diffusion-reaction (1D)}{$- u''-R(u)=f$}{reaction.c} &      \\ \bigskip
\ref{chap:of}
    &     & \tblockeqncode{$p$-Laplace (2D)}{$\begin{matrix} -\Div\left(D \grad u\right) = f \\ D = |\grad u|^{p-2} \end{matrix}$}{plap.c}
                      &  \\ \bigskip
\ref{chap:mg}
    & \begin{minipage}[t]{35mm}
 \tblockcode{Poisson (2D, 3D)}{fish2.c, fish3.c}

 \tblockeqncode{advection-diffusion (2D)}{$\bv \cdot \grad u - \grad^2 u = f$}{ad.c}
\end{minipage}
          &     &      \\ \bigskip
\ref{chap:sc}
    & \tblockeqncode{time-dependent heat (2D)}{$u_t = \grad^2 u$}{heat.c}
          &     & \tblockeqncode{porous (3D)}{$\begin{matrix} u_t = \Div\left(D \grad u\right) \\ D = u^{\gamma-1} \end{matrix}$}{porous.c} \\ \bigskip
\ref{chap:un}
    & \tblockcode{Poisson (2D)}{femfish.c}    &     &     &      \\ \bigskip
\ref{chap:sy}
    & \tblockeqncode{Stokes (2D)}{$\begin{matrix} -\grad^2 \bu + \grad p = 0 \\ \Div \bu = 0 \end{matrix}$}{stokes.c}
          &     & \tblockeqncode{shallow water (1D)}{$\begin{matrix} h_t + (hv)_x = 0 \\ (hv)_t + (hv^2 + \tfrac{1}{2} g h^2)_x = 0 \end{matrix}$}{water.c}     \\ \bigskip
\ref{chap:dp}
    & \tblockcode{Poisson (2D, 3D)}{plexfish.c}
          &     &      \\ \bigskip
\ref{chap:co}
    &     & \tblockeqncode{obstacle}{$\begin{matrix} -\grad^2 u = f \\ u\ge \psi \end{matrix}$}{obstacle.c}
                & \tblockeqncode{ice sheet}{$\begin{matrix} H_t = \Div\left(D \grad H\right) + f \\ D \text{ nonlinear},\, H \ge 0\end{matrix}$}{ice.c} \\
\bottomrule
\end{tabular}
\end{center}
\vfill


\newpage\thispagestyle{empty}
\noindent \textsc{inside back cover:}

\vfill
{\large \noindent \textsc{\PETSc component coverage}}

\begin{center}
\begin{tabular}{rccccccc}
\toprule
Chapter 
    &\;\pVec\;
          &\;\pMat\;
                &\;\texttt{KSP}${+}$\pPC\;
                      &\pSNES
                            &\pDMPlex
                                  &\pDMDA
                                        &\;\pTS\; \\
\midrule
2   & \XX & \XX & \XX &     &     &     &      \\
3   & \XX & \XX & \XX &     &     & \XX &      \\
4   & \gX & \XX & \gX & \XX &     & \XX &      \\
5   & \gX & \gX & \gX & \XX &     & \XX &      \\
6   & \gX & \gX & \XX & \XX &     & \XX &      \\
7   & \gX & \gX & \XX & \XX &     & \XX & \XX  \\
8   & \XX & \XX & \gX & \XX &     &     &      \\
9   & \gX & \gX & \XX & \XX &     & \XX & \XX  \\
10  & \gX & \gX & \XX & \XX & \XX &     &      \\
11  & \gX & \gX & \gX & \XX &     & \XX &      \\
\bottomrule
\end{tabular}
\end{center}
\vfill

\end{document}
