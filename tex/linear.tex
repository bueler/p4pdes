
\chapter{2. Linear PDEs, naively}

We start with a cliched example because it is the right place to start.  The Poisson problem models the distribution of temperature in a conducting object at steady state, the electrostatic potential, the equilibrium distribution from certain random walks, and many other other physical phenomena.

Let $\Omega \subset \RR^d$

\section{Getting a triangular mesh into \PETSc}

\cinputpart{c2triangle.c}{A help string, some pre-processor defines, and the standard preamble.}{I}{}{//ENDPREAMBLE}

\cinputpart{c2triangle.c}{Determine filenames using \PETSc option processing.}{II}{//ENDPREAMBLE}{//ENDFILENAME}

\cinputpart{c2triangle.c}{Start to read ASCII files with \textsc{triangle}-generated mesh by going to rank zero processor, reading node header, and allocating accordingly.}{III}{//ENDFILENAME}{//ENDRANK0ALLOC}

\cinputpart{c2triangle.c}{On rank zero processor: Read node information.}{IV}{//ENDRANK0ALLOC}{//ENDREADNODES}

\cinputpart{c2triangle.c}{On rank zero processor: Read element information.}{V}{//STARTREADELEMENTS}{//ENDREADELEMENTS}

\cinputpart{c2triangle.c}{On rank zero processor: Write the mesh out in \PETSc binary format.}{VI}{//STARTBINARYWRITE}{//ENDRANK0}

\cinputpart{c2triangle.c}{Re-read mesh in parallel, and write in parallel.}{VII}{//ENDRANK0}{//END}

\section{FEM method, for the Poisson equation}

\section{Preallocate a \pMat}

%\cinputfull{c2prealloc.c}{demonstrates preallocation of parallel \pMat}{}{}

\section{Assembling Poisson}

\section{Performance: convergence and scaling}

\caveat{But real-world PDEs are nonlinear.}
