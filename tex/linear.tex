
\chapter{2. Linear PDEs, naively}

We start with a cliched example because it is the right place to start.  The Poisson problem models the distribution of temperature in a conducting object at steady state, the electrostatic potential, the equilibrium distribution from certain random walks, and many other other physical phenomena.

Let $\Omega \subset \RR^d$

\section{Getting a triangular mesh into \PETSc}

\cinput{c2triangle.c}{reads \textsc{triangle}-generated mesh, then re-reads in parallel}

\section{FEM method, for the Poisson equation}

\section{Preallocate a \pMat}

\cinput{c2prealloc.c}{demonstrates preallocation of parallel \pMat}

\section{Assembling Poisson}

\section{Performance: convergence and scaling}

\caveat{But real-world PDEs are nonlinear.}
