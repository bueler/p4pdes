\documentclass{tufte-book}

\hypersetup{colorlinks}% uncomment this line if you prefer colored hyperlinks (e.g., for onscreen viewing)

\title{PETSc for PDEs\thanks{Thanks to Jed Brown and Constantine Khroulev, my gurus.}}
\author{Ed Bueler}
\publisher{Publisher of This Book}

\date{\today}

% For nicely typeset tabular material
\usepackage{booktabs}

%\documentclass[11pt,reqno]{amsart}
%prepared in AMSLaTeX, under LaTeX2e
%\addtolength{\oddsidemargin}{-.65in}
%\addtolength{\evensidemargin}{-.65in}
%\addtolength{\topmargin}{-.3in}
%\addtolength{\textwidth}{1.5in}
%\addtolength{\textheight}{.6in}

%\renewcommand{\baselinestretch}{1.1}

\usepackage{verbatim} % for "comment" environment
\usepackage{xspace}
\usepackage{fancyvrb}
%\usepackage[final]{graphicx}
\usepackage{graphicx}
\usepackage{amssymb}
%\usepackage[T1, OT1]{fontenc}
%\usepackage[pdftex, colorlinks=true, plainpages=false, linkcolor=blue, citecolor=red, urlcolor=blue]{hyperref}

%\newcommand{\regfigure}[2]{\includegraphics[height=#2in,keepaspectratio=true]{#1}}

\newcommand{\cinput}[1]{
\vspace{0.8cm}
\VerbatimInput[frame=single,framesep=3mm,label=\fbox{\normalsize \textsl{\,#1.m\,}},fontfamily=courier,fontsize=\footnotesize]{../c/#1}
\vspace{0.5cm}
}

\newcommand{\caveat}[1]{
\vspace{1.0cm}{\Large
\begin{quote}
#1
\end{quote}
}
\vspace{0.8cm}
}

\newcommand{\bA}{\mathbf{A}}
\newcommand{\bB}{\mathbf{B}}
\newcommand{\bE}{\mathbf{E}}
\newcommand{\bF}{\mathbf{F}}
\newcommand{\bJ}{\mathbf{J}}
\newcommand{\br}{\mathbf{r}}
\newcommand{\bx}{\mathbf{x}}

\newcommand{\CC}{\mathbb{C}}
\newcommand{\RR}{\mathbb{R}}
\newcommand{\ZZ}{\mathbb{Z}}

\newcommand{\eps}{\epsilon}
\newcommand{\lam}{\lambda}
\newcommand{\lap}{\triangle}

\newcommand{\Div}{\ensuremath{\nabla\cdot}}
\newcommand{\Curl}{\ensuremath{\nabla\times}}
\newcommand{\grad}{\nabla}

\newcommand{\ip}[2]{\ensuremath{\left<#1,#2\right>}}
\newcommand{\Matlab}{\textsc{Matlab}\xspace}

\newcommand{\onull}{\operatorname{null}}
\newcommand{\rank}{\operatorname{rank}}
\newcommand{\range}{\operatorname{range}}

\renewcommand{\Re}{\operatorname{Re}}
\renewcommand{\Im}{\operatorname{Im}}

\usepackage{makeidx}
\makeindex

\begin{document}

\frontmatter

\maketitle

\tableofcontents


\chapter*{Why this book?}

This is a draft structure.  For sure I will cite \citep{Smithetal1996}.

\section{Introduction}

\section{What is PETSc, and how to use it?}

\newcommand{\pVec}{{\Large\texttt{Vec}}}
\newcommand{\pMat}{{\Large\texttt{Mat}}}

\section{\pVec s and \pMat s}  [introduce initialization/finalization, object Create+SetFromOptions paradigm, parallel layout of \pVec

\cinput{c1matvec.c}

\caveat{\emph{But} \Matlab is all you want if scale does not matter}

%%%%%%%%%%

\mainmatter

\chapter{PETSc for \textsc{linear} PDEs}

\section{Getting a triangular mesh into PETSc}

\cinput{c2triangle.c}

\section{FEM method, for the Poisson equation}

\section{Preallocate a \pMat}

\cinput{c2prealloc.c}

\section{Assembling Poisson}

\section{Performance: convergence and scaling}

\caveat{\emph{But} real PDEs are nonlinear.  And PETSc helps with meshes.}

%%%%%%%%%%

%\chapter{}

%\section{}

%\section{}

%%%%%%%%%%

\backmatter

\bibliography{ice-bib}
\bibliographystyle{plainnat}

\end{document}
