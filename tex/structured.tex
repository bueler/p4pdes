
\chapter{2. Linear PDEs: structured grids}

We start with a cliched example, the Poisson problem, because it is the right place to start.  Though the Poisson problem is a cliche in applied mathematics, it gives us the opportunity to use \PETSc several different ways, starting here with building a structured grid using \PETSc \pDMDA, and solving in parallel using a \pKSP object.

\section{Poisson-Dirichlet problem}

\cinput{structuredlaplacian.c}{Fill matrix entries using \texttt{MatSetValuesStencil}.}{//CREATEMATRIX}{//ENDCREATEMATRIX}{code:structuredlaplacian}

\cinputpart{c2poisson.c}{Set up \pDMDA \texttt{da} and \pMat \texttt{A} objects, and assemble the latter by calling \texttt{formlaplacian()}.}{I}{//CREATE}{//ENDCREATE}{code:ctwopoissoncreate}

\cinputpart{c2poisson.c}{Solve using \pKSP, and report on solution.}{II}{//SOLVE}{//ENDSOLVE}{code:ctwopoissonsolve}

\section{Runtime control of linear solver}

FIXME: basic Krylov theory

\section{Time-dependent heat equation}

FIXME: we WON'T do explicit, but it would look like ...

FIXME: use TS for backward-euler
